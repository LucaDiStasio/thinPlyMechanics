%\documentclass[first,firstsupp,handout,compress,notes,navigation]{ETHclass} 
%\documentclass[first,firstsupp,handout,lastsupp]{ETHclass} 
\documentclass[first,firstsupp,lastsupp,handout,last,hyperref,table]{ETHclass} 
%\documentclass[first,firstsupp]{ETHclass}
\usepackage{etex}

\usepackage{adjustbox}
\usepackage{amsmath}
\usepackage{amssymb}
\usepackage{animate}
\usepackage{booktabs}
\usepackage{charter}
\usepackage{etoolbox}
\usepackage{ifthen}
\usepackage{longtable}
\usepackage{mathrsfs}
\usepackage{multicol}
\usepackage{pgf}
\usepackage{ragged2e}
\usepackage{standalone}
\usepackage[caption=false]{subfig}
\usepackage{tabularx}
\usepackage{tikz}
\usepackage{verbatim}
\usepackage{xcolor}
\usepackage{hyperref}




\setbeamertemplate{navigation symbols}{}
\usetikzlibrary{arrows,decorations.pathreplacing,positioning,shapes,shadows}

%\usepackage[style=numeric-comp]{biblatex}

%\usepackage{lipsum}

%\usetikzlibrary{fit}
\usetikzlibrary{arrows}
\usetikzlibrary{trees}

% Options for beamer:
%
% 9,10,11,12,13,14,17pt  Fontsizes
% 
% compress: navigation bar becomes smaller
% t       : place contents of frames on top (alternative: b,c)
% handout : handoutversion
% notes   : show notes
% notes=onlyslideswithnotes
%
%hyperref={bookmarksopen,bookmarksnumbered} : Needed for menues in
%                                             acrobat. Also need
%                                             pdftex as option or 
%                                             compile with
% pdflatex '\PassOptionsToPackage{pdftex,bookmarksopen,bookmarksnumbered}{hyperref} \input{file}'

%\usepackage{beamerseminar}
%\usepackage[accumulated]{beamerseminar}
                                % remove ``accumulated'' option
                                % for original behaviour
%\usepackage{beamerbasenotes}
%\setbeamertemplate{note page}[plain] 
%\setbeameroption{notes on second screen}

%\setbeamertemplate{note page}[plain] 
\setbeamertemplate{note page}{\ \\[.3cm]
\textbf{\color{blue}Notes:}\\%[0.1cm]
{\footnotesize %\tiny
\insertnote}}
%\setbeameroption{notes on second screen}


%\setbeamertemplate{navigation symbols}{} % suppresses all navigation symbols:
 \setbeamertemplate{navigation symbols}[horizontal] % Organizes the navigation symbols horizontally.
% \setbeamertemplate{navigation symbols}[vertical] % Organizes the navigation symbols vertically.
% \setbeamertemplate{navigation symbols}[only frame symbol] % Shows only the navigational symbol for navigating frames.

\setlayoutscale{0.5}
\setparametertextfont{\scriptsize}
\setlabelfont{\scriptsize}

% \useoutertheme[subsection=false]{miniframes}
% \usepackage{etoolbox}
% \makeatletter
% \patchcmd{\slideentry}{\advance\beamer@xpos by1\relax}{}{}{}
% \def\beamer@subsectionentry#1#2#3#4#5{\advance\beamer@xpos by1\relax}%
% \makeatother

% \makeatletter
%     \newenvironment{withoutheadline}{
%        \setbeamertemplate{headline}{%
% \vspace{15pt}
% }
%     }{}
% \makeatother

\makeatletter
    \newenvironment{withoutheadline}{
         \setbeamertemplate{headline}{%
\vspace{35pt}
}
        %\def\beamer@entrycode{\vspace*{-1.5\headheight}}
    }{}
\makeatother

\newcommand{\Cross}{$\mathbin{\tikz [x=1.4ex,y=1.4ex,line width=.2ex, red] \draw (0,0) -- (1,1) (0,1) -- (1,0);}$}%

\newcommand{\Checkmark}{$\color{green}\checkmark$}

\setbeamerfont{subsection in toc}{size=\tiny}

\makeatletter
\patchcmd{\beamer@sectionintoc}
  {\vfill}
  {\vskip1.5\itemsep}
  {}
  {}
\makeatother  

\setbeamertemplate{frametitle continuation}{}

\setbeamertemplate{bibliography entry title}{}
\setbeamertemplate{bibliography entry author}{}
\setbeamertemplate{bibliography entry location}{}
\setbeamertemplate{bibliography entry note}{}

\setbeamercolor*{bibliography entry title}{fg=black}
\setbeamercolor*{bibliography entry author}{fg=black}
\setbeamercolor*{bibliography entry location}{fg=black}
\setbeamercolor*{bibliography entry note}{fg=black}
% and kill the abominable icon
%\setbeamertemplate{bibliography item}{\color{forestgreen}$\blacktriangleright$}
\setbeamertemplate{bibliography item}{\insertbiblabel}
%\setbeamertemplate{bibliography item}{\theenumiv}

\newcommand{\highlightred}[1]{%
  \colorbox{red!50}{$\displaystyle#1$}}
  
\newcommand{\highlightyellow}[1]{%
  \colorbox{yellow!50}{$\displaystyle#1$}}
  
\newcommand{\highlightgreen}[1]{%
  \colorbox{green!50}{$\displaystyle#1$}}

\AtBeginSection[]{
  \begin{frame}
  \vfill
  \centering
  \begin{beamercolorbox}[sep=8pt,center,shadow=true,rounded=true]{title}
    \usebeamerfont{frametitle}\includegraphics[width=2ex]{freccia_trasparente_verde_foresta.png}\hspace{.5ex}~{\LARGE \textsc{\bfseries \insertsectionhead}}\par%
  \end{beamercolorbox}
  \vfill
  \end{frame}
}

\hyphenpenalty=5000
\tolerance=1000

\graphicspath{{figures/}}

\newenvironment{system}%
{\left\lbrace\begin{array}{@{}l@{}}}%
{\end{array}\right.}

\newenvironment{subsystem}%
{\left\lgroup\begin{array}{@{}l@{}}}%
{\end{array}\right.}

\defbeamertemplate*{title page}{customized}[1][]
{
\usebeamerfont{subtitle}
\usebeamercolor[fg]{subtitle}

\vspace{-1.25cm}

{\flushleft
 \usebeamerfont{title}{\inserttitle}\par
}
\vspace{-.25cm}
{\flushleft
 \usebeamerfont{subtitle}{\small \insertsubtitle} \par
}

\vspace{-.5cm}

{\flushright
\setbeamercolor{author}{bg=white,fg=Red}
\usebeamerfont{author}{\footnotesize \insertauthor} \par}

\vspace{-.2cm}

{\flushright
\usebeamerfont{institute}{\tiny \insertinstitute}\par }

\vspace{.2cm}

{\centering
\usebeamerfont{date}{\scriptsize \insertdate} \par }

\vspace{0.2in}

%%----------------------------------------------------------------------------------------------%
%%                                          INPUT PARAMETERS
%%----------------------------------------------------------------------------------------------%
%
%\def\Rf{1}
%\def\Vff{0.5}
%\def\tratio{0.75}
%\def\meshfacone{0.2}
%\def\meshfactwo{0.75}
%\def\meshfacthree{0.5}
%\def\thetavalue{45}
%\def\deltatheta{15}
%
%%----------------------------------------------------------------------------------------------%
%%                               Definition of dependent parameters
%%----------------------------------------------------------------------------------------------%
%
%\def\pivalue{3.141592653589793238462643383279502884197169399375105820974944592307816406286}
%\pgfmathsetmacro\sqrttwo{sqrt(2)}
%
%\newcommand{\half}[1]{
%       0.5*#1
%       }
%
%\pgfmathsetmacro\l{0.5*\Rf*sqrt(\pivalue/\Vff)}
%
%\def\domlim{1.28*\l}
%\def\loadlim{1.197*\l}
%\def\loadlabel{0.2*\Rf}
%\def\cornerlabel{1.077*\l}
%
%\def\thetabot{\thetavalue-\deltatheta}
%\def\thetaup{\thetavalue+\deltatheta}
%
%\def\thetahalfbot{\thetavalue-0.5*\deltatheta}
%\def\thetahalfup{\thetavalue+0.5*\deltatheta}
%
%\def\thetaround{360+\thetavalue-\deltatheta}
%\def\thetadraw{0.25*\thetavalue}
%
%\def\xM{0.9*\costheta*\Rf}
%\def\yM{0.9*\sintheta*\Rf}
%
%\pgfmathsetmacro\cosfourtyfive{cos(45)}
%\pgfmathsetmacro\sinfourtyfive{sin(45)}
%
%\pgfmathsetmacro\costheta{cos(\thetavalue)}
%\pgfmathsetmacro\sintheta{sin(\thetavalue)}
%
%\pgfmathsetmacro\costhetabot{cos(\thetabot)}
%\pgfmathsetmacro\sinthetabot{sin(\thetabot)}
%
%\pgfmathsetmacro\costhetaup{cos(\thetaup)}
%\pgfmathsetmacro\sinthetaup{sin(\thetaup)}
%
%\pgfmathsetmacro\costhetahalfbot{cos(\thetahalfbot)}
%\pgfmathsetmacro\sinthetahalfbot{sin(\thetahalfbot)}
%
%\pgfmathsetmacro\costhetahalfup{cos(\thetahalfup)}
%\pgfmathsetmacro\sinthetahalfup{sin(\thetahalfup)}
%  
%\pgfmathsetmacro\yloadarrowone{\l+(\loadlim-\l)*0.2}
%\pgfmathsetmacro\yloadarrowtwo{\l+2*(\loadlim-\l)*0.2}
%\pgfmathsetmacro\yloadarrowthree{\l+3*(\loadlim-\l)*0.2}
%\pgfmathsetmacro\yloadarrowfour{\l+4*(\loadlim-\l)*0.2}
%
%\pgfmathsetmacro\ILsquared{(\costhetabot-\costhetaup)*(\costhetabot-\costhetaup)+(\sinthetabot-\sinthetaup)*(\sinthetabot-\sinthetaup))}
%\pgfmathsetmacro\IMsquared{(\costhetabot-0.9*\costheta)*(\costhetabot-0.9*\costheta)+(\sinthetabot-0.9*\sintheta)*(\sinthetabot-0.9*\sintheta)}
%\pgfmathsetmacro\IL{sqrt(\ILsquared)}
%\pgfmathsetmacro\IM{sqrt(\IMsquared)}
%\pgfmathsetmacro\angleM{asin(0.5*\IL/\IM)}
%
%\def\crackstartangle{\thetavalue-\angleM}
%\def\crackstopangle{\thetavalue+\angleM}
%
%\pgfmathsetmacro\meshradiusone{\meshfactwo*\Rf}
%\pgfmathsetmacro\meshradiustwo{\Rf+\meshfacthree*(\l-\Rf)}

%\begin{beamercolorbox}[sep=1em,wd=1.1\textwidth,ht=3cm,dp=3cm,right]{white}
%\usebeamerfont{title} {\huge \inserttitle}\par
%\usebeamerfont{subtitle} {\insertsubtitle}
%\end{beamercolorbox}
}

\begin{document}
\setbeamertemplate{caption}{\raggedright\insertcaption\par}

\title{\textsc{Mechanics of Extreme Thin Composite Layers for Aerospace Applications}}
\subtitle{\textsc{Background, Rationale \& Project's Start-Up}}
\author{ Luca Di Stasio}
\institute{ Science et Ing\'enierie des Mat\'eriaux et M\'etallurgie (SI2M), Institut Jean Lamour, Nancy, France\\Department of Engineering Sciences and Mathematics, Division of Materials Science, Lule\aa\ University of Technology, Lule\aa, Sweden}
\date{\today}

\begin{frame}[plain]
    \titlepage
\end{frame}

\begin{withoutheadline}
\begin{frame}
\frametitle{Outline}
\justifying
\vspace*{-0.5cm}
% \tableofcontents[hidesubsections]
% \begin{multicols}{2}
% \tableofcontents[hidesubsections]
% \end{multicols}
% \begin{columns}[t]
%         \begin{column}{.5\textwidth}
%             \tableofcontents[sections={1-2}]
%         \end{column}
%         \begin{column}{.5\textwidth}
%             \tableofcontents[sections={3-6}]
%         \end{column}
%     \end{columns}
% \end{frame}
\tableofcontents[hidesubsections]
\end{frame}
\end{withoutheadline}

%\note{}

%\begin{frame}
%\pagediagram
%\end{frame}
%% \note{}

\section{The new frontier in fiber composites}

\subsection{Spread tow technology}

\begin{frame}
\frametitle{Spread tow technology}
\vspace{-.5cm}
\centering
\begin{figure}
\centering
\includegraphics[width=0.8\textwidth]{spread_tow_technology_scheme}
%\caption{A schematic representation of the technology.}
\label{fig:design_workflow}
\end{figure}
\end{frame}

\subsection{Applications}

\begin{frame}
\frametitle{Advanced transportation concepts}
%\vspace{1cm}
\centering
\begin{figure}[!h]
\centering
\subfloat[\scriptsize By North Thin Ply Technology.\label{fig:solar_impulse}]{\includegraphics[height=0.45\textheight,width=0.45\textwidth]{ntpt_solar-impulse-1.jpg}}\quad
\subfloat[\scriptsize By TeXtreme.\label{fig:solar_car}]{\includegraphics[height=0.45\textheight,width=0.45\textwidth]{textreme_solar_car.jpg}}
  %\caption{Single RVE model.}
  \label{fig:transp1}
\end{figure}
\end{frame}

\begin{frame}
\frametitle{Sports}
%\vspace{1cm}
\centering
\begin{figure}[!h]
\centering
\subfloat[\scriptsize By North Thin Ply Technology.\label{fig:americascup}]{\includegraphics[height=0.45\textheight,width=0.45\textwidth]{ntpt_marine-yatching.jpg}}\quad
\subfloat[\scriptsize By TeXtreme.\label{fig:surf}]{\includegraphics[height=0.45\textheight,width=0.45\textwidth]{TeXtreme_Fanatic_FreeWave.jpg}}
  %\caption{Single RVE model.}
  \label{fig:sport1}
\end{figure}
\end{frame}

\begin{frame}
\frametitle{Sports}
%\vspace{1cm}
\centering
\begin{figure}[!h]
\centering
\subfloat[\scriptsize By North Thin Ply Technology.\label{fig:fishing}]{\includegraphics[height=0.45\textheight,width=0.45\textwidth]{ntpt_fishing-rod.jpg}}\quad
\subfloat[\scriptsize By TeXtreme.\label{fig:bicycle}]{\includegraphics[height=0.45\textheight,width=0.45\textwidth]{textreme_KE-R8-5KS-CARBON_WHITE.jpg}}
  %\caption{Single RVE model.}
  \label{fig:sport2}
\end{figure}
\end{frame}

\begin{frame}
\frametitle{Structural elements}
%\vspace{1cm}
\centering
\begin{figure}[!h]
\centering
  \includegraphics[width=\textwidth]{ntpt_tubes.jpg}%     without .tex extension
  % or use \input{mytikz}
  \caption{\scriptsize By North Thin Ply Technology.}
  \label{fig:ntpt_tubes}
  \end{figure}
\end{frame}



\begin{frame}
\frametitle{Luxury goods}
%\vspace{1cm}
\centering
\begin{figure}[!h]
\centering
    \includegraphics[width=\textwidth]{ntpt_luxurywatch.jpg}%     without .tex extension
  % or use \input{mytikz}
  \caption{\scriptsize By North Thin Ply Technology.}
  \label{fig:ntpt_luxury}
\end{figure}
\end{frame}

\section{The kinks in thin ply mechanics}

\subsection{Work-flow of composite structural design}

\begin{frame}
\frametitle{Work-flow of composite structural design}
\vspace{-0.75cm}
\centering
\begin{figure}
\centering
\includestandalone[width=\textwidth,height=0.8\textheight]{flux_diag_1}
%\caption{}
\label{fig:design_workflow}
\end{figure}
\end{frame}

\subsection{Failure load analysis}

\begin{frame}
\frametitle{Failure load analysis}
\vspace{-0.75cm}
\centering
\begin{figure}
\centering
\includestandalone[width=\textwidth,height=0.8\textheight]{flux_diag_2}
%\caption{}
\label{fig:failure_load_analysis}
\end{figure}
\centering

\end{frame}

\subsection{Progressive load analysis}

\begin{frame}
\frametitle{Progressive load analysis}
\vspace{-0.75cm}
\centering
\begin{figure}
\centering
\includestandalone[width=\textwidth,height=0.8\textheight]{flux_diag_3}
%\caption{}
\label{fig:progressive_load_analysis}
\end{figure}
\centering

\end{frame}

\section{The players in the field}

\begin{frame}
\frametitle{A map of the players in the field}
%\vspace{1cm}
\centering
\href{https://www.google.com/maps/d/edit?hl=en_US&app=mp&mid=zjFSmgZ1fbJQ.kbBAPW9bg6t4}{\beamergotobutton{Map}}
\end{frame}

\section{The model}

\subsection{Geometries, loads and boundary conditions}

\begin{frame}
\frametitle{Single RVE model}
\vspace{-0.5cm}
\centering
\begin{figure}[!h]
\centering
\subfloat[\scriptsize Crack closed in the radial direction.\label{fig:singleRVE_cc}]{\includestandalone[width=0.45\textwidth]{singleRVE_cc}}\quad
\subfloat[\scriptsize Crack open in the radial direction.\label{fig:singleRVE_oc}]{\includestandalone[width=0.45\textwidth]{singleRVE_oc}}
  %\caption{Single RVE model.}
  \label{fig:singleRVE_ccoc}
\end{figure}
\end{frame}

\begin{frame}
\frametitle{Single RVE model}
\vspace{-0.75cm}
\centering
\begin{figure}[!h]
\centering
    \includestandalone[height=0.7\textheight]{singleRVE_cc}%     without .tex extension
  % or use \input{mytikz}
  \caption{\scriptsize Initial state of single RVE model: crack closed in the radial direction.}
  \label{fig:singleRVE_onlycc}
\end{figure}
\end{frame}


\begin{frame}
\frametitle{Bounded RVE model}
\vspace{-0.75cm}
\centering
\begin{figure}[!h]
\centering
\subfloat[\scriptsize Crack closed in the radial direction.\label{fig:boundedRVE_cc}]{\includestandalone[width=0.425\textwidth]{boundedRVE_cc}}\quad
\subfloat[\scriptsize Crack open in the radial direction.\label{fig:boundedRVE_oc}]{\includestandalone[width=0.425\textwidth]{boundedRVE_oc}}
  % or use \input{mytikz}
  %\caption{Bounded RVE model.}
  \label{fig:boundedRVE_ccoc}
\end{figure}
\end{frame}

\begin{frame}
\frametitle{Bounded RVE model}
\vspace{-0.75cm}
\centering
\begin{figure}[!h]
\centering
    \includestandalone[height=0.7\textheight]{boundedRVE_cc}%     without .tex extension
  % or use \input{mytikz}
  \caption{\scriptsize Initial state of bounded RVE model: crack closed in the radial direction.}
  \label{fig:boundedRVE_onlycc}
\end{figure}
\end{frame}

\begin{frame}
\frametitle{Periodic RVE model}
\vspace{-0.5cm}
\centering
\begin{figure}[!h]
\centering
\subfloat[\scriptsize Crack closed in the radial direction.\label{fig:periodicRVE_cc}]{\includestandalone[width=0.45\textwidth]{periodicRVE_cc}}\quad
\subfloat[\scriptsize Crack open in the radial direction.\label{fig:periodicRVE_oc}]{\includestandalone[width=0.45\textwidth]{periodicRVE_oc}}
  % or use \input{mytikz}
  %\caption{Periodic \acrshort{rve} model.}
  \label{fig:periodicRVE_ccoc}
\end{figure}
\end{frame}

\begin{frame}
\frametitle{Periodic RVE model}
\vspace{-0.75cm}
\centering
\begin{figure}[!h]
\centering
    \includestandalone[height=0.7\textheight]{periodicRVE_cc}%     without .tex extension
  % or use \input{mytikz}
  \caption{\scriptsize Initial state of periodic RVE model: crack closed in the radial direction.}
  \label{fig:periodicRVE_onlycc}
\end{figure}
\end{frame}

\begin{frame}
\frametitle{Summary of designed geometries}
\vspace*{-0.25cm}
\centering
\begin{sidewaystable}[htbp]
  \centering
  \small
  \caption{Model geometries summary.}
    \begin{tabularx}{\textwidth}{lXp{0.08\textwidth}XXX}
    \toprule
  \textbf{Name} & \textbf{Description}&\textbf{Number of phases}&\textbf{Geometry of each phase}&\textbf{Boundary conditions}&\textbf{Imposed conditions} \\
    \midrule
   single-RVE &Circular fiber inside a square matrix domain.&2&Fiber: circular; matrix: square with circular inclusion at its center.&Constant strain at $z=\pm l$; in order to have constant strain, the displacement has a linear functional form, i.e. $u_{x}|_{z=\pm l}=\bar{u}_{x}\frac{x}{l}$.&Constant displacement $u_{x}|_{z=\pm l}=\bar{u}_{x}=\bar{\varepsilon}_{x}\cdot l$ at $x=\pm l$.\\
 \midrule
   bounded-RVE&Circular fiber inside a square matrix domain, bounded by two \acrshort{ud} rectangular domains on the upper and lower side.&3&Fiber: circular; matrix: square with circular inclusion at its center; \acrshort{ud}: rectangular.&Free surface at $z=\pm l$.&Constant displacement $u_{x}|_{z=\pm l}=\bar{u}_{x}=\bar{\varepsilon}_{x}\cdot l$ at $x=\pm l$.\\
 \midrule
   periodic-RVE&Periodically repeated unit cell, constituted by a circular fiber inside a square matrix domain.&2&Fiber: circular; matrix: square with circular inclusion at its center.&Periodic boundary conditions on all sides.&Constant displacement $u_{x}|_{z=\pm l}=\bar{u}_{x}=\bar{\varepsilon}_{x}\cdot l$ at $x=\pm l$.\\
    &&&&&\\
    \bottomrule
    \end{tabularx}%
  \label{tab:geom_tab}%
\end{sidewaystable}%

\end{frame}

\begin{frame}
\frametitle{Summary of designed geometries}
\vspace*{-0.25cm}
\centering
\begin{table}[htbp]
\footnotesize
  \centering
  \small
  %\caption{Model geometries summary.}
    \begin{tabularx}{\textwidth}{cc}
    \toprule
   \midrule
    \textbf{Name} & \textbf{Number of phases}\\
     bounded-RVE&3\\
    \midrule
    \multicolumn{2}{X}{\textbf{Description}}\\
    \multicolumn{2}{X}{Circular fiber inside a square matrix domain, bounded by two UD rectangular domains on the upper and lower side.}\\
    \midrule
    \multicolumn{2}{X}{\textbf{Geometry of each phase}}\\
    \multicolumn{2}{X}{Fiber: circular; matrix: square with circular inclusion at its center; UD: rectangular.}\\
    \midrule
    \multicolumn{2}{X}{\textbf{Boundary conditions}}\\
    \multicolumn{2}{X}{Free surface at $z=\pm l$.}\\
    \midrule
    \multicolumn{2}{X}{\textbf{Imposed conditions}}\\
    \multicolumn{2}{X}{Constant displacement $u_{x}|_{z=\pm l}=\bar{u}_{x}=\bar{\varepsilon}_{x}\cdot l$ at $x=\pm l$.}\\
    \midrule
    \bottomrule
    \end{tabularx}%
  \label{tab:geom_tab2}%
\end{table}%
\end{frame}

\begin{frame}
\frametitle{Summary of designed geometries}
\vspace*{-0.25cm}
\centering
\begin{table}[htbp]
\footnotesize
  \centering
  \small
  %\caption{Model geometries summary.}
    \begin{tabularx}{\textwidth}{cc}
    \toprule
    \midrule
    \textbf{Name} & \textbf{Number of phases}\\
     periodic-RVE&2\\
    \midrule
    \multicolumn{2}{X}{\textbf{Description}}\\
    \multicolumn{2}{X}{Periodically repeated unit cell, constituted by a circular fiber inside a square matrix domain.}\\
    \midrule
    \multicolumn{2}{X}{\textbf{Geometry of each phase}}\\
    \multicolumn{2}{X}{Fiber: circular; matrix: square with circular inclusion at its center.}\\
    \midrule
    \multicolumn{2}{X}{\textbf{Boundary conditions}}\\
    \multicolumn{2}{X}{Periodic boundary conditions on all sides.}\\
    \midrule
    \multicolumn{2}{X}{\textbf{Imposed conditions}}\\
    \multicolumn{2}{X}{Constant displacement $u_{x}|_{z=\pm l}=\bar{u}_{x}=\bar{\varepsilon}_{x}\cdot l$ at $x=\pm l$.}\\
    \midrule
    \bottomrule
    \end{tabularx}%
  \label{tab:geom_tab3}%
\end{table}%
\end{frame}

\subsection{Material properties}

\begin{frame}
\frametitle{Summary of single phase properties}
%\vspace*{-0.25cm}
\centering
\begin{table}[htbp]
\footnotesize
  \centering
  %\caption{Single phase properties summary.}
    \begin{tabularx}{\textwidth}{cccccccc}
    \toprule
    \textbf{Material} & \textbf{$E_{1}$}\ & \textbf{$E_{2}$}\ & \textbf{$G_{12}$} & \textbf{$\nu_{12}$} & \textbf{$\nu_{23}$} & \textbf{$a_{1}$} & \textbf{$a_{2}$} \\
   & \textbf{$\left[GPa\right]$}\ & \textbf{$\left[GPa\right]$}\ & \textbf{$\left[GPa\right]$} & \textbf{$\ \left[-\right]$} & \textbf{$\left[-\right]$} & \textbf{$\left[10^{-6}\  \frac{m}{mK}\right]$} & \textbf{$\left[10^{-6}\  \frac{m}{mK}\right]$} \\
    \midrule
    CF    & 500,0 & 30,0  & 20,0  & 0,2   & 0,5   & -1,0  & 7,8 \\
    GF    & 70,0  & 70,0  & 29,2  & 0,2   & 0,2   & 4,7   & 4,7 \\
    EP    & 3,5   & 3,5   & 1,3   & 0,4   & 0,4   & 60,0  & 60,0 \\
    \bottomrule
    \end{tabularx}%
  \label{tab:phaseprop}%
\end{table}%

\end{frame}

\begin{frame}
\frametitle{Summary of UD ply properties}
%\vspace*{-0.25cm}
\centering
\begin{table}[htbp]
  \centering
  \caption{\acrshort{ud} ply properties summary.}
    \begin{tabular}{cccccccc}
    \toprule
  \textbf{Material} & \textbf{$V_{f}$}\ & \textbf{$E_{1}$}\ & \textbf{$E_{2}$}\ & \textbf{$G_{12}$} &  \textbf{$G_{23}$} &\textbf{$\nu_{12}$} & \textbf{$\nu_{23}$} \\
   & \textbf{$\left[-\right]$} &\textbf{$\left[GPa\right]$}& \textbf{$\left[GPa\right]$}\ & \textbf{$\left[GPa\right]$}& \textbf{$\left[GPa\right]$} & \textbf{$\ \left[-\right]$} & \textbf{$\left[-\right]$} \\
    \midrule
    \acrshort{cf}/\acrshort{ep} & 0,6   & 301,4422 & 11,0389 & 4,0625 & 3,5767 & 0,2734 & 0,5432 \\
    \acrshort{cf}/\acrshort{ep} & 0,4   & 202,1433 & 7,5694 & 2,6136 & 2,3803 & 0,3133 & 0,5899 \\
    \acrshort{gf}/\acrshort{ep} & 0,6   & 43,4425 & 13,7145 & 4,3140 & 4,6808 & 0,2726 & 0,4650 \\
    \bottomrule
    \end{tabular}%
  \label{tab:UDprop}%
\end{table}%

\end{frame}

\subsection{Mesh characteristics}


\begin{frame}
\frametitle{Mesh regions}
\vspace{-0.75cm}
\centering
\begin{figure}[!h]
\centering
    \includestandalone[height=0.8\textheight]{mesh_regions}%     without .tex extension
  % or use \input{mytikz}
  %\caption{Block regions of the \acrshort{rve} geometry.}
  \label{fig:mesh_regions}
\end{figure}
\end{frame}


\begin{frame}
\frametitle{Mesh parameters}
\vspace{-0.75cm}
\centering
\begin{figure}[!h]
\centering
    \includestandalone[height=\textheight]{mesh_parameters_single}%     without .tex extension
  % or use \input{mytikz}
  %\caption{Parameters for mesh generation for the single and periodic \acrshort{rve}.}
  \label{fig:mesh_param_single}
\end{figure}
\end{frame}

\begin{frame}
\frametitle{Mesh parameters}
\vspace{-0.75cm}
\centering
\begin{figure}[!h]
\centering
    \includestandalone[height=\textheight]{mesh_parameters_bounded}%     without .tex extension
  % or use \input{mytikz}
  %\caption{Parameters for mesh generation for the bounded \acrshort{rve}.}
  \label{fig:mesh_param_bounded}
\end{figure}
\end{frame}

\begin{frame}
\frametitle{Helical numbering}
\vspace{-0.75cm}
\centering
\begin{figure}[!h]
\centering
    \includestandalone[height=0.7\textheight]{scheme_model3}%     without .tex extension
  % or use \input{mytikz}
  %\caption{\scriptsize CPE4.}
  \label{fig:helix}
\end{figure}
\end{frame}

\begin{frame}
\frametitle{Topological transformation}
\vspace{-0.75cm}
\centering
\begin{figure}[!h]
\centering
    \includestandalone[height=0.7\textheight]{scheme_model2}%     without .tex extension
  % or use \input{mytikz}
  %\caption{\scriptsize CPE4.}
  \label{fig:topo_transf}
\end{figure}
\end{frame}

\begin{frame}
\frametitle{Elements}
\vspace{-0.75cm}
\centering
\begin{figure}[!h]
\centering
    \includestandalone[height=0.7\textheight]{abaqus_cpe4}%     without .tex extension
  % or use \input{mytikz}
  \caption{\scriptsize CPE4.}
  \label{fig:cpe4}
\end{figure}
\end{frame}

\begin{frame}
\frametitle{Elements}
\vspace{-0.75cm}
\centering
\begin{figure}[!h]
\centering
    \includestandalone[height=0.7\textheight]{abaqus_cpe8}%     without .tex extension
  % or use \input{mytikz}
  \caption{\scriptsize CPE8.}
  \label{fig:cpe8}
\end{figure}
\end{frame}

\subsection{Types of analysis}

\begin{frame}
\frametitle{Features}
\vspace*{-0.25cm}
\centering
\begin{table}[h!]
\scriptsize
  \centering
  %\caption{Analysis methods summary.}
    \begin{tabularx}{\textwidth}{X}
    \toprule
    \midrule    
    \textbf{Method}\\
    ABAQUS/STD static analysis + VCCT + J-integral.\\
    \midrule
    \textbf{Type}\\
    Static, i.e. no inertial effects. Relaxation until equilibrium.\\
    \midrule
    \textbf{Elements}\\
    CPE4/CPE8\\
    \midrule
    \textbf{Interface}\\
    Tied surface constraint \& contact mechanics\\
    \midrule
   \textbf{Input variables}\\
    $R_{f}$, $V_{f}$, material properties, interface properties.\\
    \midrule
    \textbf{Control variables}\\
    $\theta$, $\Delta\theta$, $\bar{\varepsilon}_{x}$.\\
    \midrule
 \textbf{Output variables} \\
  Stress field, crack tip stress, stress intensity factors, energy release rates, $a$.\\
  \midrule
    \bottomrule
    \end{tabularx}%
  \label{tab:analysis_tab}%
\end{table}%

\end{frame}

\begin{frame}
\frametitle{Features}
\vspace*{-0.5cm}
\centering
\begin{table}[h!]
\scriptsize
  \centering
  %\caption{Analysis methods summary.}
    \begin{tabularx}{\textwidth}{X}
    \toprule
    \midrule
    \textbf{Method}\\
    ABAQUS/STD static analysis + CZM.\\
    \midrule
    \textbf{Type}\\
    Static, i.e. no inertial effects. Relaxation until equilibrium.\\
    \midrule
    \textbf{Elements}\\
    CPE4/CPE8 + COH2D4\\
    \midrule
    \textbf{Interface}\\
    Cohesive elements.\\
    \midrule
   \textbf{Input variables}\\
    $R_{f}$, $V_{f}$, material properties.\\
    \midrule
    \textbf{Control variables}\\
    Interface properties, maximum stresses at crack onset, energy release rates, applied strain.\\
    \midrule
 \textbf{Output variables} \\
  $\theta$, $\Delta\theta$, $a$, stress field, peak crack boundary stresses.\\
  \midrule
    \bottomrule
    \end{tabularx}%
  \label{tab:analysis_tab2}%
\end{table}%
\end{frame}

\begin{frame}
\frametitle{Abaqus commands}
%\vspace{1cm}
\centering
\begin{table}[htbp]
\footnotesize
  \centering
  %\caption{\gls{abaqusstd} commands summary.}
    \begin{tabularx}{\textwidth}{cc}
    \toprule
  \textbf{Method}&  \textbf{ABAQUS command}\\
    \midrule
static analysis&*STATIC\\
VCCT&*FRACTURE CRITERION, TYPE=VCCT\\
J-integral method&*CONTOUR INTEGRAL\\
cohesive element method&*COHESIVE SECTION\\
tied surface constraint&*TIE\\
contact mechanics&*CONTACT\\
stress intensity factors&*CONTOUR INTEGRAL, TYPE=K FACTORS\\
    \bottomrule
    \end{tabularx}%
  \label{tab:command_tab}%
\end{table}%
\end{frame}

\section{Project's on-line presence}

\begin{frame}
\frametitle{Project's on-line presence}
\centering
\href{http://tpm.docmase.lucadistasioengineering.com/}{\beamergotobutton{Website}}\\
\vspace*{1cm}
\href{https://www.facebook.com/tpm.docmase?ref=hl}{\beamergotobutton{Facebook}}
\end{frame}

\begin{frame}[plain]
\frametitle{}
\vspace{1cm}
\centering
{\LARGE
\textsc{Thank you!}
}
\end{frame}

%\section{Appendices}

%\begin{frame}[label=]
%\frametitle{}

%\end{frame}



%\section{References}

%\begin{frame}[t,label=references,allowframebreaks]
%       \frametitle{References}
%	\begin{itemize}
%%	\item Loading rate effects on delamination:\\[10pt] \textit{Loading\_rate\_effects\_on\_CFRP.bib}\\[30pt]
%	\item Body-fitted grids for FSI modeling with LBM:\\[10pt] %\textit{Fluid\_structure\_interaction\_on\_deformable\_surfaces.bib}
%	\end{itemize}
%        \bibliographystyle{amsalpha}
%        {\footnotesize
%          \bibliography{PSI_talk.bib}
%        }
        %\bibliography{/auto.mounter/home/lucadistasio/Documents/ETH/Research_material/References/fsi_references_kbib.bib}
%\end{frame}

\begin{frame}[plain]
\frametitle{}
\end{frame}

\end{document}

