\begin{table}
\centering
\scriptsize
\begin{tabularx}{\textwidth}{cccccc}
\toprule
\midrule
  \multicolumn{3}{l}{\textbf{Title}}&\textbf{Code} &\textbf{ECTS}&\textbf{Hours} \\
   &&&& \textbf{credits}&\\
    \midrule
    &&&&&\\
    \multicolumn{3}{p{0.5\textwidth}}{\textbf{Aerospace Materials}}&T7005T&7.5&\\
    \midrule
    &\textit{Institution}&\multicolumn{4}{p{0.7\textwidth}}{Lule\aa\ University of Technology.}\\
    &\textit{Organization}&\multicolumn{4}{p{0.7\textwidth}}{The course will take place from April 4, 2016 (week 14) to June 19, 2016 (week 24).}\\
    \iffalse
    &\textit{Objective}&\multicolumn{4}{p{0.7\textwidth}}{After the end of this course the student is supposed to - have deep knowledge about structure and behaviour of high performance materials used in aerospace industry - be able to evaluate properties of composites, ceramic materials and alloys to perform optimal material selection for use in harsh environments and service conditions - will know and understand the most important degradation mechanisms that initiate and evolve due to thermal and mechanical loads and lead to material fatigue and reduced durability - be able to do produce long fiber composites, to measure their mechanical properties, to observe and to quantify damage modes and to analyse their effect on properties - be able to apply composite material degradation models, to perform fracture mechanics analysis in alloys and to predict time dependent material behaviour - be able to perform numerical simulations of structures using commercial software to design optimized structures - have good skills in analysing research papers and writing research reports.}\\
    &\textit{Syllabus}&\multicolumn{4}{p{0.7\textwidth}}{The material classes analyzed in this course are high performance materials like light weight alloys, superalloys, ceramics and different types of composites including materials modified on nanoscale. Methodology will be given to determine properties of these multiscale materials on all considered length scales. The properties most important for design in the aerospace applications are performance at high mechanical loads, extreme temperatures and material aging and fatigue due to extreme environmental effects. Processing methods will be considered in relation to desired material performance. Durability and damage tolerance will be accessed by analyzing degradation, creep and damage mechanisms. Methodology for structural analysis will be given and training performed.}\\
    \fi
    &\textit{Requirements}&\multicolumn{4}{p{0.7\textwidth}}{It satisfies the DocMASE for scientific training, EMMA requirements for scientific courses and Lule\aa\ University of Technology requirements.}\\
    &\textit{Needs}&\multicolumn{4}{p{0.7\textwidth}}{The focus of the course is strongly related to the project theme, as it reviews the methods for performance assessment and damage prediction for materials used in aerospace applications.}\\
    &\textit{Status}&\multicolumn{4}{p{0.7\textwidth}}{Agreed upon with supervisors.}\\
    \midrule
    \bottomrule
\label{tab:proposal_tab1} 
\end{tabularx}
\end{table}
