\begin{table}
\centering
\scriptsize
\begin{tabularx}{\textwidth}{cccccc}
\toprule
\midrule
  \multicolumn{3}{l}{\textbf{Title}}&\textbf{Code} &\textbf{ECTS}&\textbf{Hours} \\
   &&&& \textbf{credits}&\\
    \midrule
    &&&&&\\
    \multicolumn{3}{p{0.5\textwidth}}{\textbf{Modeling of crystal behavior and textures}}&EMMA 05&5&24\\
    \midrule
    &\textit{Institution}&\multicolumn{4}{p{0.7\textwidth}}{Universit\'e de Lorraine.}\\
    &\textit{Organisation}&\multicolumn{4}{p{0.7\textwidth}}{Distance learning format.}\\
    \iffalse
    &\textit{Objective}&\multicolumn{4}{p{0.7\textwidth}}{Nowadays, the basic problems of crystal plasticity are well solved and their applications in the various fields of research of mechanics and physics of materials became standard. The goal of this course is to familiarize with crystal plasticity in order to understand and set up various modeling in the broad field of mechanics of materials. The course is supplemented by simulations to carry out on PC.}\\
    &\textit{Syllabus}&\multicolumn{4}{p{0.7\textwidth}}{Introduction (geometrical considerations, mechanisms of plastic deformation of crystals). Equations of deformation (small and large strain formulation). Crystal plasticity criteria (Schmid, Bishop and Hill, viscoplastic slip). Work hardening of crystals (matrix of work hardening, techniques of simulations). The mechanical problem of crystal plasticity (relation between strain and stress). Polycrystal deformation (static, Sachs, Taylor, relaxed constraints, self consistent models, finite elements). Discrete modelings (molecular, atomic). Application of polycrystalline models to materials (prediction of crystallographic texture, parameters of anisotropy, work hardening and formability for cubic, hexagonal, multiphase, intermetallic, superplastic materials and nano materials). Computer modeling in crystal plasticity. Effects of temperature on crystal plasticity (continuous or discontinuous recrystallization, possibilities of modeling). Heterogeneities of the deformation (instability and localization of deformation in single and polycrystals).}\\
    \fi
    &\textit{Requirements}&\multicolumn{4}{p{0.7\textwidth}}{It satisfies the DocMASE for scientific training and EMMA requirements for scientific courses. It could probably be transferred for credits to satisfy Lule\aa\ requirements.}\\
    &\textit{Needs}&\multicolumn{4}{p{0.7\textwidth}}{The course is related to the doctoral school theme, but not directly to the research project. I think a higher-level course on crystal behaviour fits well and could help me acquire a more complete background in the field.}\\
    &\textit{Status}&\multicolumn{4}{p{0.7\textwidth}}{Under discussion.}\\
    \midrule
    \bottomrule
\label{tab:proposal_tab3} 
\end{tabularx}
\end{table}