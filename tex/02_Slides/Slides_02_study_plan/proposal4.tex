\begin{table}
\centering
\scriptsize
\begin{tabularx}{\textwidth}{cccccc}
\toprule
\midrule
  \multicolumn{3}{l}{\textbf{Title}}&\textbf{Code} &\textbf{ECTS}&\textbf{Hours} \\
   &&&& \textbf{credits}&\\
    \midrule
    &&&&&\\
    \multicolumn{3}{p{0.5\textwidth}}{\textbf{Physique quantique à l'usage exclusif des non physiciens}}&EMMA 11&3&15\\
    \multicolumn{3}{p{0.5\textwidth}}{\textbf{(Quantum physics for non-physicists)}}&&&\\
    \midrule
    &\textit{Institution}&\multicolumn{4}{p{0.7\textwidth}}{Universit\'e de Lorraine.}\\
    &\textit{Organisation}&\multicolumn{4}{p{0.7\textwidth}}{The course will take place from 14:00 to 17:00 on February 24, March 02, 09, 16, 23, 2016 (a total of 5 lectures).}\\
    \iffalse
    &\textit{Objective}&\multicolumn{4}{p{0.7\textwidth}}{The course presents the basics of quantum mechanics for non-specialists with mathematical background.}\\
    &\textit{Syllabus}&\multicolumn{4}{p{0.7\textwidth}}{Axioms and formulations of quantum mechancis. Interpretations of quantum physics. From classical to quantum mechanics. From quantum to classical mechanics. Quantum information and informatics.}\\
    \fi
    &\textit{Requirements}&\multicolumn{4}{p{0.7\textwidth}}{It satisfies the DocMASE for scientific training and EMMA requirements for scientific courses. It could probably be transferred for credits to satisfy Lule\aa\ requirements.}\\
    &\textit{Needs}&\multicolumn{4}{p{0.7\textwidth}}{The course is related to the doctoral school theme, but not directly to the research project. It will provide the basic understandings to work within the (sub-)atomic materials science field.}\\
    &\textit{Status}&\multicolumn{4}{p{0.7\textwidth}}{Under discussion.}\\
    \midrule
    \bottomrule
\label{tab:proposal_tab4} 
\end{tabularx}
\end{table}