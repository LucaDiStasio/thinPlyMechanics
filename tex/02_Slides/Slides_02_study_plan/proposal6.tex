\begin{table}
\centering
\scriptsize
\begin{tabularx}{\textwidth}{cccccc}
\toprule
\midrule
  \multicolumn{3}{l}{\textbf{Title}}&\textbf{Code} &\textbf{ECTS}&\textbf{Hours} \\
   &&&& \textbf{credits}&\\
    \midrule
    &&&&&\\
    \multicolumn{3}{p{0.5\textwidth}}{\textbf{Modelisation des milieux heterogenes}}&RP2E MS 23&$\approx4$&20\\
    \multicolumn{3}{p{0.5\textwidth}}{\textbf{(Heterogeneous materials modeling)}}&&&\\
    \midrule
    &\textit{Institution}&\multicolumn{4}{p{0.7\textwidth}}{Universit\'e de Lorraine.}\\
    &\textit{Organization}&\multicolumn{4}{p{0.7\textwidth}}{The course will take place on March 21, 22, 23, 24 and 25, 2016 (a total of 6 lectures).}\\
    \iffalse
    &\textit{Objective}&\multicolumn{4}{p{0.7\textwidth}}{Provide the scientific foundations for the numerical modeling of heterogeneous materials at multiple scales.}\\
    &\textit{Syllabus}&\multicolumn{4}{p{0.7\textwidth}}{Homogenisation techniques: introduction to the micro-mechanics of materials; homogenisation methods; estimation of effective material properties. Variational and \acrfull{fem}: principles of variational methods in heterogeneous media elasticity; homogenisation and its application to \acrshort{fem} in linear thermo-elasticity.}\\
    \fi
    &\textit{Requirements}&\multicolumn{4}{p{0.7\textwidth}}{It satisfies the DocMASE for scientific training and EMMA requirements for scientific courses. It could probably be transferred for credits to satisfy Lule\aa\ requirements.}\\
    &\textit{Needs}&\multicolumn{4}{p{0.7\textwidth}}{The subject of the course is related to the project theme, as it reviews the methods for the micro-mechanical and multi-scale analysis of heterogeneous materials, such as fiber reinforced polymer composites. It could potentially provide valid tools that can be put to fruitful use in the research work.}\\
    &\textit{Status}&\multicolumn{4}{p{0.7\textwidth}}{Under discussion.}\\
    \midrule
    \bottomrule
\label{tab:proposal_tab6} 
\end{tabularx}
\end{table}