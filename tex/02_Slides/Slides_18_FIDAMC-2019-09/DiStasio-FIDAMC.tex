%\documentclass[first,firstsupp,handout,compress,notes,navigation,hyperref,table]{ETHclass}
%\documentclass[first,firstsupp,handout,lastsupp]{ETHclass}
\documentclass[first,firstsupp,lastsupp,last,hyperref,table]{ETHclass}
%\documentclass[first,firstsupp]{ETHclass}
\usepackage{etex}


\usepackage{adjustbox}
\usepackage{amsmath}
\usepackage{amssymb}
\usepackage{animate}
\usepackage{booktabs}
\usepackage{charter}
\usepackage{enumitem}
\usepackage{etoolbox}
\usepackage{ifthen}
\usepackage{longtable}
\usepackage{mathrsfs}
\usepackage{multicol}
\usepackage{nicefrac}
\usepackage{pgf}
\usepackage{pgfpages}
\usepackage{pgfplots}
\usepackage{pifont}
\usepackage{ragged2e}
\usepackage{standalone}
\usepackage[caption=false]{subfig}
\usepackage{tabularx}
\usepackage{tikz}
\usepackage{verbatim}
\usepackage{wasysym}
\usepackage{xcolor}
\usepackage{hyperref}

\pgfplotsset{compat=1.7}

\setbeamertemplate{navigation symbols}{}
\usetikzlibrary{arrows,decorations.pathreplacing,positioning,shapes,shadows}

%\usepackage[style=numeric-comp]{biblatex}

%\usepackage{lipsum}

%\usetikzlibrary{fit}
\usetikzlibrary{arrows}
\usetikzlibrary{trees}

% Options for beamer:
%
% 9,10,11,12,13,14,17pt  Fontsizes
%
% compress: navigation bar becomes smaller
% t       : place contents of frames on top (alternative: b,c)
% handout : handoutversion
% notes   : show notes
% notes=onlyslideswithnotes
%
%hyperref={bookmarksopen,bookmarksnumbered} : Needed for menues in
%                                             acrobat. Also need
%                                             pdftex as option or
%                                             compile with
% pdflatex '\PassOptionsToPackage{pdftex,bookmarksopen,bookmarksnumbered}{hyperref} \input{file}'

%\usepackage{beamerseminar}
%\usepackage[accumulated]{beamerseminar}
                                % remove ``accumulated'' option
                                % for original behaviour
%\usepackage{beamerbasenotes}
%\setbeamertemplate{note page}[plain]
%\setbeameroption{notes on second screen}

%\setbeamertemplate{note page}[plain]
%\setbeamertemplate{note page}{\ \\[.3cm]
%\textbf{\color{blue}Notes:}\\%[0.1cm]
%{\footnotesize %\tiny
%\insertnote}}
%\setbeameroption{notes on second screen}


\setbeamertemplate{navigation symbols}{} % suppresses all navigation symbols:
% \setbeamertemplate{navigation symbols}[horizontal] % Organizes the navigation symbols horizontally.
 %\setbeamertemplate{navigation symbols}[vertical] % Organizes the navigation symbols vertically.
% \setbeamertemplate{navigation symbols}[only frame symbol] % Shows only the navigational symbol for navigating frames.

\setlayoutscale{0.5}
\setparametertextfont{\scriptsize}
\setlabelfont{\scriptsize}

% \useoutertheme[subsection=false]{miniframes}
% \usepackage{etoolbox}
% \makeatletter
% \patchcmd{\slideentry}{\advance\beamer@xpos by1\relax}{}{}{}
% \def\beamer@subsectionentry#1#2#3#4#5{\advance\beamer@xpos by1\relax}%
% \makeatother

% \makeatletter
%     \newenvironment{withoutheadline}{
%        \setbeamertemplate{headline}{%
% \vspace{15pt}
% }
%     }{}
% \makeatother

\makeatletter
    \newenvironment{withoutheadline}{
         \setbeamertemplate{headline}{%
\vspace{35pt}
}
        %\def\beamer@entrycode{\vspace*{-1.5\headheight}}
    }{}
\makeatother

\makeatletter
    \newenvironment{tocframe}{
         \setbeamertemplate{headline}{%
\vspace{35pt}
}
         \setbeamertemplate{footline}{%
\leavevmode%
  \hbox{
    \begin{beamercolorbox}[wd=.3\paperwidth,ht=2.5ex,dp=0.75ex,left]{author in head/foot}%
      \qquad\color{white}{\tiny\underline{L. Di Stasio}}%
    \end{beamercolorbox}%

    \begin{beamercolorbox}[wd=.45\paperwidth,ht=2.5ex,dp=0.75ex,center]{title in head/foot}%
      \color{white}{Getafe, Madrid (ES) - September 16, 2019}%
    \end{beamercolorbox}%

    \begin{beamercolorbox}[wd=.2\paperwidth,ht=2.5ex,dp=0.75ex,right]{title in head/foot}%
      \color{white}{\insertframenumber\totalframes}%
    \end{beamercolorbox}%
  }
}
        %\def\beamer@entrycode{\vspace*{-1.5\headheight}}
    }{}
\makeatother

\newcommand{\Cross}{$\mathbin{\tikz [x=1.4ex,y=1.4ex,line width=.2ex, red] \draw (0,0) -- (1,1) (0,1) -- (1,0);}$}%

\newcommand{\Checkmark}{$\color{green}\mathbf{\checkmark}$}

\setbeamerfont{subsection in toc}{size=\tiny}

\makeatletter
\patchcmd{\beamer@sectionintoc}
  {\vfill}
  {\vskip1.5\itemsep}
  {}
  {}
\makeatother

\setbeamertemplate{frametitle continuation}{}

\setbeamertemplate{bibliography entry title}{}
\setbeamertemplate{bibliography entry author}{}
\setbeamertemplate{bibliography entry location}{}
\setbeamertemplate{bibliography entry note}{}

\setbeamercolor*{bibliography entry title}{fg=black}
\setbeamercolor*{bibliography entry author}{fg=black}
\setbeamercolor*{bibliography entry location}{fg=black}
\setbeamercolor*{bibliography entry note}{fg=black}
% and kill the abominable icon
%\setbeamertemplate{bibliography item}{\color{forestgreen}$\blacktriangleright$}
\setbeamertemplate{bibliography item}{\insertbiblabel}
%\setbeamertemplate{bibliography item}{\theenumiv}

\newcommand{\highlightred}[1]{%
  \colorbox{red!50}{$\displaystyle#1$}}

\newcommand{\highlightyellow}[1]{%
  \colorbox{yellow!50}{$\displaystyle#1$}}

\newcommand{\highlightgreen}[1]{%
  \colorbox{green!50}{$\displaystyle#1$}}

\AtBeginSection[]{
  \begin{frame}
  \vfill
  \centering
  \begin{beamercolorbox}[sep=8pt,center,shadow=true,rounded=true]{title}
    \usebeamerfont{frametitle}\includegraphics[width=2ex]{freccia_trasparente_verde_foresta.png}\hspace{.5ex}~{\LARGE \textsc{\bfseries \insertsectionhead}}\par%
  \end{beamercolorbox}
  \vfill
  \end{frame}
}

\hyphenpenalty=5000
\tolerance=1000

\graphicspath{{figures/}}

\newenvironment{system}{\left\lbrace\begin{array}{@{}l@{}}}{\end{array}\right.}

\newenvironment{subsystem}{\left\lgroup\begin{array}{@{}l@{}}}{\end{array}\right.}

\defbeamertemplate*{title page}{customized}[1][]
{
\usebeamerfont{subtitle}
\usebeamercolor[fg]{subtitle}

\vspace{-2cm}

{\flushleft
 \usebeamerfont{title}{\inserttitle}\par
}

%\vspace{-.5cm}
%\vspace{-.25cm}
{\flushright
\setbeamercolor{author}{bg=white,fg=Red}
\usebeamerfont{author}{\footnotesize \insertauthor} \par}

\vspace{-.25cm}

{\flushright
\usebeamerfont{institute}{\tiny \insertinstitute}\par }

\vspace{0.5cm}

{\centering
\usebeamerfont{date}{\scriptsize \insertdate} \par }


}


\begin{document}
\setbeamertemplate{caption}{\raggedright\insertcaption\par}

\title{\textsc{\small \textbf{Towards tough self-healing thin-ply laminates}}\\\textsc{\scriptsize Insights from computational micromechanical modeling and high-temperature experimental investigation of onset and propagation of transverse cracking}}

\author{\vspace*{-0.5cm}\scriptsize \underline{Luca Di Stasio}$^{1,2}$}
%\institute{ Science et Ing\'enierie des Mat\'eriaux et M\'etallurgie (SI2M), Institut Jean Lamour, Nancy, France\\Department of Engineering Sciences and Mathematics, Division of Materials Science, Lule\aa\ University of Technology, Lule\aa, Sweden}
\institute{\tiny$^{1}$Division of Materials Science, Lule\aa\ University of Technology, Lule\aa, Sweden\\$^{2}$EEIGM \& IJL, Universit\'e de Lorraine, Nancy, France}
\date{\vspace*{-0.2cm}\tiny Fundaci\'on para la Investigaci\'on, Desarrollo y Aplicaci\'on de Materiales Compuestos (FIDAMC)\\[-3.5pt]Getafe, Madrid (ES) - September 16, 2019}

\begin{frame}[plain]
    \titlepage

\end{frame}

\begin{tocframe}
\begin{frame}
\frametitle{Outline}
\justifying
\vspace*{-0.5cm}
% \tableofcontents[hidesubsections]
% \begin{multicols}{2}
% \tableofcontents[hidesubsections]
% \end{multicols}
% \begin{columns}[t]
%         \begin{column}{.5\textwidth}
%             \tableofcontents[sections={1-2}]
%         \end{column}
%         \begin{column}{.5\textwidth}
%             \tableofcontents[sections={3-6}]
%         \end{column}
%     \end{columns}
% \end{frame}
\tableofcontents[hidesubsections]
\end{frame}
\end{tocframe}

%\note{}

%\begin{frame}
%\pagediagram
%\end{frame}
%% \note{}

%\section[Transverse Cracking in FRPCs]{Initiation of Transverse Cracking in Fiber Reinforced Polymer Composites (FRPCs): Microscopic Observations \& Modeling}

\section{Transverse Cracking in Thin-plies}

\subsection{The Thin-ply ``Advantage''}

\begin{frame}
\frametitle{\vspace{0.3cm}\small The Thin-ply ``Advantage'': new material}
\vspace{-0.8cm}
\centering
\begin{columns}[t]
\begin{column}{0.5\textwidth}
\centering
\tiny
\textbf{2018}, $\mathbf{\left[45^{\circ}, 90^{\circ},-45^{\circ},0^{\circ}\right]_{nS}}$
\end{column}
\begin{column}{0.5\textwidth}
\centering

\end{column}
\end{columns}
\vspace{-0.4cm}
\begin{columns}[c]
\begin{column}{0.5\textwidth}
\begin{figure}
\centering
\includegraphics[width=0.9\columnwidth]{thinply-plythicknesseffect.jpg}
\end{figure}
\vspace{-0.6cm}
\centering
\textcolor{white}{\tiny$t_{90^{\circ}}<10\diameter_{fiber}$}
\end{column}
\begin{column}{0.5\textwidth}
\centering
\end{column}
\end{columns}
\vspace{0.15cm}
\begin{columns}[b]
\begin{column}{0.5\textwidth}
\centering
\pgfmathsetmacro\fontsizeref{4.5}
\pgfmathsetmacro\stretchref{1}
{\fontsize{\fontsizeref}{\stretchref} \selectfont \href{https://doi.org/10.1016/j.compscitech.2018.08.037}{Cugnoni et al., Compos. Sci. Technol. \textbf{168}, 2018.}}
\end{column}
\begin{column}{0.5\textwidth}
\centering
\end{column}
\end{columns}
\end{frame}

%\addtocounter{framenumber}{-1}

\begin{frame}
\frametitle{\vspace{0.3cm}\small The Thin-ply ``Advantage''? New material, new problem...}
\vspace{-0.8cm}
\centering
\begin{columns}[t]
\begin{column}{0.5\textwidth}
\centering
\tiny
\textbf{2018}, $\mathbf{\left[45^{\circ}, 90^{\circ},-45^{\circ},0^{\circ}\right]_{nS}}$
\end{column}
\begin{column}{0.5\textwidth}
\centering

\end{column}
\end{columns}
\vspace{-0.4cm}
\begin{columns}[c]
\begin{column}{0.5\textwidth}
\begin{figure}
\centering
\includegraphics[width=0.9\columnwidth]{thinply-plythicknesseffect-brittle.png}
\end{figure}

\end{column}
\begin{column}{0.5\textwidth}
\centering
\scriptsize
\begin{itemize}[label=\ding{212}]
\item material from NTPT (CH) (on the left)
\item recent tests (master theses $@$ LTU) have shown the same behavior with material from Oxeon (SE)
\end{itemize}
\begin{itemize}[label=\color{green}\ding{52}]
\item extended range of elastic behavior with no damage-induced stiffness reduction
\end{itemize}
\begin{itemize}[label=\color{red}\ding{56}]
\item brittle failure of the whole laminate!
\item very risky for application in safety-critical primary structures!
\end{itemize}
\end{column}
\end{columns}
\vspace{0.15cm}
\begin{columns}[b]
\begin{column}{0.5\textwidth}
\centering
\pgfmathsetmacro\fontsizeref{4.5}
\pgfmathsetmacro\stretchref{1}
{\fontsize{\fontsizeref}{\stretchref} \selectfont \href{https://doi.org/10.1016/j.compscitech.2018.08.037}{Cugnoni et al., Compos. Sci. Technol. \textbf{168}, 2018.}}
\end{column}
\begin{column}{0.5\textwidth}
\
\end{column}
\end{columns}
\end{frame}

%\addtocounter{framenumber}{-1}

\begin{frame}
\frametitle{\vspace{0.3cm}\small The Thin-ply ``Advantage'': new material, old result}
\vspace{-0.8cm}
\centering
\begin{columns}[t]
\begin{column}{0.5\textwidth}
\centering
\tiny
\textbf{2018}, $\mathbf{\left[45^{\circ}, 90^{\circ},-45^{\circ},0^{\circ}\right]_{nS}}$
\end{column}
\begin{column}{0.5\textwidth}
\centering
\tiny
\textbf{1979}, $\mathbf{\left[0^{\circ}, 90^{\circ}_{n}\right]_{S}}$
\end{column}
\end{columns}
\vspace{-0.4cm}
\begin{columns}[c]
\begin{column}{0.5\textwidth}
\begin{figure}
\centering
\includegraphics[width=0.9\columnwidth]{thinply-plythicknesseffect.jpg}
\end{figure}
\vspace{-0.6cm}
\centering
\textcolor{white}{\tiny$t_{90^{\circ}}<10\diameter_{fiber}$}
\end{column}
\begin{column}{0.5\textwidth}
\centering
\begin{figure}
\centering
\includegraphics[width=\columnwidth]{bailey.png}
\end{figure}
\textcolor{white}{\tiny$t_{90^{\circ}}>100\diameter_{fiber}$}
\end{column}
\end{columns}
\vspace{0.15cm}
\begin{columns}[b]
\begin{column}{0.5\textwidth}
\centering
\pgfmathsetmacro\fontsizeref{4.5}
\pgfmathsetmacro\stretchref{1}
{\fontsize{\fontsizeref}{\stretchref} \selectfont \href{https://doi.org/10.1016/j.compscitech.2018.08.037}{Cugnoni et al., Compos. Sci. Technol. \textbf{168}, 2018.}}
\end{column}
\begin{column}{0.5\textwidth}
\centering
\pgfmathsetmacro\fontsizeref{4.5}
\pgfmathsetmacro\stretchref{1}
{\fontsize{\fontsizeref}{\stretchref} \selectfont \href{https://doi.org/10.1098/rspa.1979.0071}{Bailey et al., P. Roy. Soc. A-Math. Phy. \textbf{366} (1727), 1979.}}
\end{column}
\end{columns}
\end{frame}

%\addtocounter{framenumber}{-1}

\begin{frame}
\frametitle{\vspace{0.3cm}\small The Thin-ply ``Advantage'': new material, old result?}
\vspace{-0.8cm}
\centering
\begin{columns}[t]
\begin{column}{0.5\textwidth}
\centering
\tiny
\textbf{2018}, $\mathbf{\left[45^{\circ}, 90^{\circ},-45^{\circ},0^{\circ}\right]_{nS}}$
\end{column}
\begin{column}{0.5\textwidth}
\centering
\tiny
\textbf{1979}, $\mathbf{\left[0^{\circ}, 90^{\circ}_{n}\right]_{S}}$
\end{column}
\end{columns}
\vspace{-0.4cm}
\begin{columns}[c]
\begin{column}{0.5\textwidth}
\begin{figure}
\centering
\includegraphics[width=0.9\columnwidth]{thinply-plythicknesseffect-highlighted.png}
\end{figure}
\vspace{-0.6cm}
\centering
\textcolor{red}{\tiny$t_{90^{\circ}}<10\diameter_{fiber}$}
\end{column}
\begin{column}{0.5\textwidth}
\centering
\begin{figure}
\centering
\includegraphics[width=\columnwidth]{bailey-highlighted.png}
\end{figure}
\textcolor{red}{\tiny$t_{90^{\circ}}>100\diameter_{fiber}$}
\end{column}
\end{columns}
\vspace{0.15cm}
\begin{columns}[b]
\begin{column}{0.5\textwidth}
\centering
\pgfmathsetmacro\fontsizeref{4.5}
\pgfmathsetmacro\stretchref{1}
{\fontsize{\fontsizeref}{\stretchref} \selectfont \href{https://doi.org/10.1016/j.compscitech.2018.08.037}{Cugnoni et al., Compos. Sci. Technol. \textbf{168}, 2018.}}
\end{column}
\begin{column}{0.5\textwidth}
\centering
\pgfmathsetmacro\fontsizeref{4.5}
\pgfmathsetmacro\stretchref{1}
{\fontsize{\fontsizeref}{\stretchref} \selectfont \href{https://doi.org/10.1098/rspa.1979.0071}{Bailey et al., P. Roy. Soc. A-Math. Phy. \textbf{366} (1727), 1979.}}
\end{column}
\end{columns}
\end{frame}

\subsection{Micromechanics of Initiation}

\begin{frame}
\frametitle{\vspace{0.3cm}\small Micromechanics of Initiation}
\vspace{-0.5cm}
\centering
\captionsetup[subfigure]{labelfont=footnotesize}
\begin{tikzpicture}

%Tikz axis starts%

\draw[->,black, line width = 5pt] (-4,0) -- (-4,3);
\draw[->,black, line width = 5pt] (-4,0) -- (-4,-3);
\node  at (-4,0) {\includegraphics[height=0.6\textheight]{0-90_2_symm}};
\draw[<->,red, line width = 1.25pt] (-4.35,-0.5) -- (-3.65,-0.625);
\node[red, below]  at (-4,-0.65) {\tiny$\mathbf{\sim 15\ mm}$};
\node[above, right]  at (-3.8,3) {\Huge$\mathbf{\varepsilon}$};
\node[below, left]  at (-4.2,-3) {\Huge$\mathbf{\varepsilon}$};

\draw[->,black, line width = 5pt] (-2,0) -- (-2,3);
\draw[->,black, line width = 5pt] (-2,0) -- (-2,-3);
\node  at (-2,0) {\includegraphics[height=0.6\textheight]{0-90_symm_edge}};
\draw[<->,red, line width = 1.25pt] (-2.46,-0.6) -- (-1.55,-0.6);
\node[red, below]  at (-2,-0.6) {\tiny$\mathbf{\sim 2\ mm}$};
\node[above, right]  at (-1.8,3) {\Huge$\mathbf{\varepsilon}$};
\node[below, left]  at (-2.2,-3) {\Huge$\mathbf{\varepsilon}$};

\draw[->,black, line width = 5pt] (1,0) -- (1,3);
\draw[->,black, line width = 5pt] (1,0) -- (1,-3);
\node  at (1,0) {\includegraphics[height=0.6\textheight]{0-90_symm_debondetail}};
\draw[<->,red, line width = 1.25pt] (0.075,-0.15) -- (1.1,-0.15);
\node[red, below]  at (0.55,-0.16) {\tiny$\mathbf{\sim 8\ \mu m}$};
\node[above, right]  at (1.2,3) {\Huge$\mathbf{\varepsilon}$};
\node[below, left]  at (0.8,-3) {\Huge$\mathbf{\varepsilon}$};

\node[anchor=south west]  at (3.25,1.55) {\scriptsize\textbf{Left:}};
\node[anchor=west]  at (3.25,1.5) {\scriptsize front view of $\left[0,90_{2}\right]_{S}$,};
\node[anchor=north west]  at (3.25,1.5) {\scriptsize visual inspection.};

\node[anchor=south west]  at (3.25,0.05) {\scriptsize\textbf{Center:}};
\node[anchor=west]  at (3.25,0) {\scriptsize  edge view of $\left[0,90\right]_{S}$,};
\node[anchor=north west]  at (3.25,0) {\scriptsize optical microscope.};

\node[anchor=south west]  at (3.25,-1.45) {\scriptsize\textbf{Right:}};
\node[anchor=west]  at (3.25,-1.5) {\scriptsize edge view of $\left[0,90\right]_{S}$,};
\node[anchor=north west]  at (3.25,-1.5) {\scriptsize optical microscope.};

%Tikz axis ends%

\end{tikzpicture}
\end{frame}

%\addtocounter{framenumber}{-1}

\begin{frame}
\frametitle{\vspace{0.2cm}\small Micromechanics of Initiation}
\vspace{-0.5cm}
\centering
\begin{alertblock}{\centering\scriptsize\bf Stage 1: isolated debonds}
\vspace{-0.25cm}
\begin{figure}
\centering
\includegraphics[width=0.6\textwidth]{stage1-isolateddebonds.pdf}
\end{figure}
\end{alertblock}
\vspace{-0.5cm}
\pgfmathsetmacro\fontsizeref{4.5}
\pgfmathsetmacro\stretchref{1}
{\fontsize{\fontsizeref}{\stretchref} \selectfont \href{https://doi.org/10.1098/rspa.1979.0071}{Bailey et al., P. Roy. Soc. A-Math. Phy. \textbf{366} (1727), 1979.}}\\\vspace{-5pt}
{\fontsize{\fontsizeref}{\stretchref} \selectfont \href{https://doi.org/10.1007/BF00552203}{Bailey et al., J. Mater. Sci. \textbf{16} (3), 1981.}}\\\vspace{-5pt}
{\fontsize{\fontsizeref}{\stretchref} \selectfont \href{https://doi.org/10.1016/S1359-835X(96)00123-6}{Zhang et al., Compos. Part A-Appl. S. \textbf{28} (4), 1997.}}
\end{frame}

\addtocounter{framenumber}{-1}

\begin{frame}
\frametitle{\vspace{0.2cm}\small Micromechanics of Initiation}
\vspace{-0.5cm}
\centering
\begin{alertblock}{\centering\scriptsize\bf Stage 2: consecutive debonds}
\vspace{-0.25cm}
\begin{figure}
\centering
\includegraphics[width=0.6\textwidth]{stage2-critdebonds.pdf}
\end{figure}
\end{alertblock}
\vspace{-0.5cm}
\pgfmathsetmacro\fontsizeref{4.5}
\pgfmathsetmacro\stretchref{1}
{\fontsize{\fontsizeref}{\stretchref} \selectfont \href{https://doi.org/10.1098/rspa.1979.0071}{Bailey et al., P. Roy. Soc. A-Math. Phy. \textbf{366} (1727), 1979.}}\\\vspace{-5pt}
{\fontsize{\fontsizeref}{\stretchref} \selectfont \href{https://doi.org/10.1007/BF00552203}{Bailey et al., J. Mater. Sci. \textbf{16} (3), 1981.}}\\\vspace{-5pt}
{\fontsize{\fontsizeref}{\stretchref} \selectfont \href{https://doi.org/10.1016/S1359-835X(96)00123-6}{Zhang et al., Compos. Part A-Appl. S. \textbf{28} (4), 1997.}}
\end{frame}

\addtocounter{framenumber}{-1}

\begin{frame}
\frametitle{\vspace{0.2cm}\small Micromechanics of Initiation}
\vspace{-0.5cm}
\centering
\begin{alertblock}{\centering\scriptsize\bf Stage 3: kinking}
\vspace{-0.25cm}
\begin{figure}
\centering
\includegraphics[width=0.6\textwidth]{stage3-kinking.pdf}
\end{figure}
\end{alertblock}
\vspace{-0.5cm}
\pgfmathsetmacro\fontsizeref{4.5}
\pgfmathsetmacro\stretchref{1}
{\fontsize{\fontsizeref}{\stretchref} \selectfont \href{https://doi.org/10.1098/rspa.1979.0071}{Bailey et al., P. Roy. Soc. A-Math. Phy. \textbf{366} (1727), 1979.}}\\\vspace{-5pt}
{\fontsize{\fontsizeref}{\stretchref} \selectfont \href{https://doi.org/10.1007/BF00552203}{Bailey et al., J. Mater. Sci. \textbf{16} (3), 1981.}}\\\vspace{-5pt}
{\fontsize{\fontsizeref}{\stretchref} \selectfont \href{https://doi.org/10.1016/S1359-835X(96)00123-6}{Zhang et al., Compos. Part A-Appl. S. \textbf{28} (4), 1997.}}
\end{frame}

\addtocounter{framenumber}{-1}

\begin{frame}
\frametitle{\vspace{0.2cm}\small Micromechanics of Initiation}
\vspace{-0.5cm}
\centering
\begin{alertblock}{\centering\scriptsize\bf Stage 4: coalescence}
\vspace{-0.25cm}
\begin{figure}
\centering
\includegraphics[width=0.6\textwidth]{stage4-coalescence.pdf}
\end{figure}
\end{alertblock}
\vspace{-0.5cm}
\pgfmathsetmacro\fontsizeref{4.5}
\pgfmathsetmacro\stretchref{1}
{\fontsize{\fontsizeref}{\stretchref} \selectfont \href{https://doi.org/10.1098/rspa.1979.0071}{Bailey et al., P. Roy. Soc. A-Math. Phy. \textbf{366} (1727), 1979.}}\\\vspace{-5pt}
{\fontsize{\fontsizeref}{\stretchref} \selectfont \href{https://doi.org/10.1007/BF00552203}{Bailey et al., J. Mater. Sci. \textbf{16} (3), 1981.}}\\\vspace{-5pt}
{\fontsize{\fontsizeref}{\stretchref} \selectfont \href{https://doi.org/10.1016/S1359-835X(96)00123-6}{Zhang et al., Compos. Part A-Appl. S. \textbf{28} (4), 1997.}}
\end{frame}

\subsection{A Counter-intuitive Observation}

\begin{frame}
\frametitle{\vspace{0.3cm}\small A Counter-intuitive Observation}
\vspace{-0.5cm}
\centering
\tiny
$\mathbf{\left[0^{\circ}, 90^{\circ}_{n}\right]_{S}}$
\vspace{-0.25cm}
\begin{columns}[c]
\begin{column}{0.5\textwidth}
\centering
\begin{figure}
\centering
\includegraphics[width=\columnwidth]{saito-thick.png}
\end{figure}
\vspace{-0.4cm}
$n=4$, $t_{90^{\circ}}=160\mu m$
\end{column}
\begin{column}{0.5\textwidth}
\centering

\end{column}
\end{columns}
\vspace{-0.1cm}
\centering
\begin{figure}
\centering
\includegraphics[width=0.5\textwidth]{saito-fill.png}
\end{figure}
\vspace{-0.25cm}
\textcolor{white}{$n=1$, $t_{90^{\circ}}=40\mu m$}\\\vspace{5pt}
\pgfmathsetmacro\fontsizeref{4.5}
\pgfmathsetmacro\stretchref{1}
{\fontsize{\fontsizeref}{\stretchref} \selectfont \href{https://doi.org/10.1163/156855112X629522}{Saito et al., Adv. Compos. Mater. \textbf{21} (1), 2012.}}
\end{frame}

\addtocounter{framenumber}{-1}

\begin{frame}
\frametitle{\vspace{0.3cm}\small A Counter-intuitive Observation}
\vspace{-0.5cm}
\centering
\tiny
$\mathbf{\left[0^{\circ}, 90^{\circ}_{n}\right]_{S}}$
\vspace{-0.25cm}
\begin{columns}[c]
\begin{column}{0.5\textwidth}
\centering
\begin{figure}
\centering
\includegraphics[width=\columnwidth]{saito-thick.png}
\end{figure}
\vspace{-0.4cm}
$n=4$, $t_{90^{\circ}}=160\mu m$
\end{column}
\begin{column}{0.5\textwidth}
\centering
\begin{figure}
\centering
\includegraphics[width=\columnwidth]{saito-mid.png}
\end{figure}
\vspace{-0.4cm}
$n=2$, $t_{90^{\circ}}=80\mu m$
\end{column}
\end{columns}
\vspace{-0.1cm}
\centering
\begin{figure}
\centering
\includegraphics[width=0.5\textwidth]{saito-fill.png}
\end{figure}
\vspace{-0.25cm}
\textcolor{white}{$n=1$, $t_{90^{\circ}}=40\mu m$}\\\vspace{5pt}
\pgfmathsetmacro\fontsizeref{4.5}
\pgfmathsetmacro\stretchref{1}
{\fontsize{\fontsizeref}{\stretchref} \selectfont \href{https://doi.org/10.1163/156855112X629522}{Saito et al., Adv. Compos. Mater. \textbf{21} (1), 2012.}}
\end{frame}

\addtocounter{framenumber}{-1}

\begin{frame}
\frametitle{\vspace{0.3cm}\small A Counter-intuitive Observation}
\vspace{-0.5cm}
\centering
\tiny
$\mathbf{\left[0^{\circ}, 90^{\circ}_{n}\right]_{S}}$
\vspace{-0.25cm}
\begin{columns}[c]
\begin{column}{0.5\textwidth}
\centering
\begin{figure}
\centering
\includegraphics[width=\columnwidth]{saito-thick.png}
\end{figure}
\vspace{-0.4cm}
$n=4$, $t_{90^{\circ}}=160\mu m$
\end{column}
\begin{column}{0.5\textwidth}
\centering
\begin{figure}
\centering
\includegraphics[width=\columnwidth]{saito-mid.png}
\end{figure}
\vspace{-0.4cm}
$n=2$, $t_{90^{\circ}}=80\mu m$
\end{column}
\end{columns}
\vspace{-0.1cm}
\centering
\begin{figure}
\centering
\includegraphics[width=0.5\textwidth]{saito-thin.png}
\end{figure}
\vspace{-0.25cm}
$n=1$, $t_{90^{\circ}}=40\mu m$\\\vspace{5pt}
\pgfmathsetmacro\fontsizeref{4.5}
\pgfmathsetmacro\stretchref{1}
{\fontsize{\fontsizeref}{\stretchref} \selectfont \href{https://doi.org/10.1163/156855112X629522}{Saito et al., Adv. Compos. Mater. \textbf{21} (1), 2012.}}
\end{frame}

\section{Modeling Fiber/Matrix Debonding}

\subsection{Prelude}

\begin{frame}
\frametitle{\vspace{0.3cm}\small Micromechanics of Initiation}
\vspace{-0.5cm}
\centering
\captionsetup[subfigure]{labelfont=footnotesize}
\begin{tikzpicture}

%Tikz axis starts%

\draw[->,black, line width = 5pt] (-4,0) -- (-4,3);
\draw[->,black, line width = 5pt] (-4,0) -- (-4,-3);
\node  at (-4,0) {\includegraphics[height=0.6\textheight]{0-90_2_symm}};
\draw[<->,red, line width = 1.25pt] (-4.35,-0.5) -- (-3.65,-0.625);
\node[red, below]  at (-4,-0.65) {\tiny$\mathbf{\sim 15\ mm}$};
\node[above, right]  at (-3.8,3) {\Huge$\mathbf{\varepsilon}$};
\node[below, left]  at (-4.2,-3) {\Huge$\mathbf{\varepsilon}$};

\draw[->,black, line width = 5pt] (-2,0) -- (-2,3);
\draw[->,black, line width = 5pt] (-2,0) -- (-2,-3);
\node  at (-2,0) {\includegraphics[height=0.6\textheight]{0-90_symm_edge}};
\draw[<->,red, line width = 1.25pt] (-2.46,-0.6) -- (-1.55,-0.6);
\node[red, below]  at (-2,-0.6) {\tiny$\mathbf{\sim 2\ mm}$};
\node[above, right]  at (-1.8,3) {\Huge$\mathbf{\varepsilon}$};
\node[below, left]  at (-2.2,-3) {\Huge$\mathbf{\varepsilon}$};

\draw[->,black, line width = 5pt] (1,0) -- (1,3);
\draw[->,black, line width = 5pt] (1,0) -- (1,-3);
\node  at (1,0) {\includegraphics[height=0.6\textheight]{0-90_symm_debondetail}};
\draw[<->,red, line width = 1.25pt] (0.075,-0.15) -- (1.1,-0.15);
\draw[red, line width = 1.5pt] (3.5,0.0) arc(0:360:2.5);
\node[red, below]  at (0.55,-0.16) {\tiny$\mathbf{\sim 8\ \mu m}$};
\node[above, right]  at (1.2,3) {\Huge$\mathbf{\varepsilon}$};
\node[below, left]  at (0.8,-3) {\Huge$\mathbf{\varepsilon}$};

\node[anchor=south west]  at (3.25,1.55) {\scriptsize\textbf{Left:}};
\node[anchor=west]  at (3.25,1.5) {\scriptsize front view of $\left[0,90_{2}\right]_{S}$,};
\node[anchor=north west]  at (3.25,1.5) {\scriptsize visual inspection.};

\node[anchor=south west]  at (3.25,0.05) {\scriptsize\textbf{Center:}};
\node[anchor=west]  at (3.25,0) {\scriptsize  edge view of $\left[0,90\right]_{S}$,};
\node[anchor=north west]  at (3.25,0) {\scriptsize optical microscope.};

\node[anchor=south west]  at (3.25,-1.45) {\scriptsize\textbf{Right:}};
\node[anchor=west]  at (3.25,-1.5) {\scriptsize edge view of $\left[0,90\right]_{S}$,};
\node[anchor=north west]  at (3.25,-1.5) {\scriptsize optical microscope.};

%Tikz axis ends%

\end{tikzpicture}
\end{frame}

\subsection{Geometry}

\begin{frame}
\frametitle{\vspace{0.2cm}\small Geometry}
\vspace{-1cm}
\centering
\begin{columns}[c]
\begin{column}{0.85\textwidth}
\begin{figure}
\centering
\includegraphics[width=\columnwidth]{laminate-section.pdf}
\end{figure}
\end{column}
\begin{column}{0.15\textwidth}
\scriptsize
\begin{itemize}[label=\ding{212}]
\item $L, W >> t$
\item $L, W \rightarrow \infty$
\item Square packing
\item$L_{d} >> \Delta\theta_{d}$
\item 2D RVE
\end{itemize}
\end{column}
\end{columns}
\end{frame}

\subsection{RVEs}

\begin{frame}
\frametitle{\vspace{0.2cm}\small Representative Volume Elements (RVEs)}
\vspace{-1cm}
\centering
\begin{columns}[c]
\begin{column}{0.55\textwidth}
\centering
\begin{figure}
\centering
\includegraphics[width=\columnwidth]{thinPly.pdf}
\end{figure}
\vspace{-0.25cm}
\begin{equation*}
n\times 1-m\cdot t_{90^{\circ}}
\end{equation*}
\end{column}
\begin{column}{0.45\textwidth}
\centering
\begin{figure}
\centering
\includegraphics[width=\columnwidth]{ThickPly.pdf}
\end{figure}
\vspace{-0.25cm}
\begin{equation*}
n\times  k-m\cdot t_{90^{\circ}}
\end{equation*}
\end{column}
\end{columns}
\end{frame}

\addtocounter{framenumber}{-1}

\begin{frame}
\frametitle{\vspace{0.2cm}\small Representative Volume Elements}
\vspace{-1.25cm}
\centering
\begin{columns}[c]
\begin{column}{0.55\textwidth}
\centering
\begin{figure}
\centering
\includegraphics[width=\columnwidth]{freeThinPly.pdf}
\end{figure}
%\vspace{-0.25cm}
\begin{equation*}
\begin{aligned}
&-free\\
n\times 1&-coupling\\
&-coupling+H\\
\end{aligned}
\end{equation*}
\end{column}
\begin{column}{0.45\textwidth}
\centering
\begin{figure}
\centering
\includegraphics[width=\columnwidth]{freeThickPly.pdf}
\end{figure}
\vspace{-0.25cm}
\begin{equation*}
\begin{aligned}
&-free\\
n\times k&-coupling\\
&-coupling+H\\
\end{aligned}
\end{equation*}
\end{column}
\end{columns}
\end{frame}

\subsection{Assumptions}

\begin{frame}
\frametitle{\vspace{0.2cm}\small Assumptions}
\vspace{-0.75cm}
\centering
\begin{columns}
\begin{column}{0.5\textwidth}
\begin{figure}
\includegraphics[width=\columnwidth]{RUC.pdf}
\end{figure}
\vspace{-0.1cm}
\tiny
\begin{equation*}
R_{f}=1\ \left[\mu m\right]\quad L=\frac{R_{f}}{2}\sqrt{\frac{\pi}{V_{f}}}
\end{equation*}
\end{column}
\begin{column}{0.5\textwidth}
\scriptsize
\begin{itemize}[label=\ding{212}]
\item Linear elastic, homogeneous materials
\item Concentric Cylinders Assembly with Self-Consistent Shear Model for UD
\item Plane strain
\item Frictionless contact interaction
\item Symmetric w.r.t. x-axis
\item Coupling of x-displacements on left and right side (repeating unit cell)
\item Applied uniaxial tensile strain $\bar{\varepsilon}_{x}=1\%$
\item $V_{f}=60\%$
\end{itemize}
\end{column}
\end{columns}
\vspace{0.05cm}
\begin{table}[htbp]
\tiny
\centering
\begin{tabular}{ccccccc}
\textbf{Material} & \textbf{$V_{f}\left[\%\right]$}\  & \textbf{$E_{L}\left[GPa\right]$}\ & \textbf{$E_{T}\left[GPa\right]$}\  & \textbf{$\mu_{LT}\left[GPa\right]$} &\textbf{$\nu_{LT}\left[-\right]$} &\textbf{$\nu_{TT}\left[-\right]$} \\
\midrule
Glass fiber &-   & 70.0 & 70.0  & 29.2 & 0.2  & 0.2\\
Epoxy    &-& 3.5 & 3.5   & 1.25 &  0.4& 0.4\\
UD&60.0&43.442&13.714& 4.315& 0.273&0.465\\
\end{tabular}
\end{table}
\end{frame}

\subsection{Solution}

\begin{frame}
\frametitle{\vspace{0.4cm}\small Solution}
\vspace{-1.1cm}
\centering
\begin{columns}
\begin{column}{0.6\textwidth}
\begin{figure}
\includegraphics[width=0.825\columnwidth]{RUC.pdf}
\end{figure}
\vspace{-1cm}
\begin{columns}
\begin{column}{0.3\columnwidth}
\tiny
\begin{equation*}
\begin{aligned}
&\text{in }\Omega_{f}, \Omega_{m}, \Omega_{UD}:\\
&\frac{\partial^{2}\varepsilon_{xx}}{\partial z^{2}}+\frac{\partial^{2}\varepsilon_{zz}}{\partial x^{2}}=\frac{\partial^{2}\gamma_{zx}}{\partial x\partial z}\\
&\varepsilon_{y}=\gamma_{xy}=\gamma_{yz}=0\\
&\frac{\partial\sigma_{xx}}{\partial x}+\frac{\partial\tau_{zx}}{\partial z} = 0\\
&\frac{\partial\tau_{zx}}{\partial x}+\frac{\partial\sigma_{zz}}{\partial z} = 0\\
&\sigma_{yy}=\nu\left(\sigma_{xx}+\sigma_{zz}\right)\\
\end{aligned}
\end{equation*}
\end{column}
\begin{column}{0.7\columnwidth}
\tiny
\begin{equation*}
\begin{aligned}
&\text{for } 0^{\circ}\leq\alpha\leq\Delta\theta:\\
&\left(\overrightarrow{u}_{m}\left(R_{f},\alpha\right)-\overrightarrow{u}_{f}\left(R_{f},\alpha\right)\right)\cdot\overrightarrow{n}_{\alpha}\geq 0\\
&\text{for } \Delta\theta\leq\alpha\leq 180^{\circ}:\\
&\overrightarrow{u}_{m}\left(R_{f},\alpha\right)-\overrightarrow{u}_{f}\left(R_{f},\alpha\right)=0\\
&\sigma_{ij}=E_{ijkl}\varepsilon_{kl}\\
&+BC
\end{aligned}
\end{equation*}
\end{column}
\end{columns}
\end{column}
\begin{column}{0.35\textwidth}
\scriptsize
\begin{itemize}[label=\ding{212}]
\item Oscillating singularity
\vspace{-0.25cm}
\begin{equation*}
\sigma\sim r^{-\frac{1}{2}}\sin\left(\varepsilon\log r\right),\quad V_{f}\rightarrow 0
\end{equation*}
%\vspace{-0.25cm}
{\tiny
\begin{equation*}
\varepsilon=\frac{1}{2\pi}\log\left(\frac{1-\beta}{1+\beta}\right)
\end{equation*}
%\vspace{-0.25cm}
\begin{equation*}
\beta=\frac{\mu_{2}\left(\kappa_{1}-1\right)-\mu_{1}\left(\kappa_{2}-1\right)}{\mu_{2}\left(\kappa_{1}+1\right)+\mu_{1}\left(\kappa_{2}+1\right)}
\end{equation*}}
\item receding contact problem
\item Finite Element Method (FEM) in Abaqus\texttrademark + VCCT
\item $2^{nd}$ order shape functions
\item 6-nodes triangles \& 8-nodes quadrilaterals
\item regular mesh of quadrilaterals at the crack tip:
\begin{itemize}[label=-]
\item $AR\sim 1$
\item $\delta=0.05^{\circ}$
\end{itemize}
\end{itemize}
\end{column}
\end{columns}
\end{frame}


%\section{Debond Energy Release Rate}

\subsection{Interaction of Debonds}

\begin{frame}
\frametitle{\vspace{0.2cm}\small Interaction of Debonds: Strain Magnification}
\vspace{-.75cm}
\centering
\begin{columns}[c]
\centering
\begin{column}{0.5\textwidth}
\centering
\begin{figure}
\centering
\includegraphics[width=\columnwidth]{nxk-1-vf60-GI-strainmagni1.pdf}
\end{figure}
\end{column}
\begin{column}{0.5\textwidth}
\centering
\begin{figure}
\centering
\includegraphics[width=\columnwidth]{nxk-1-vf60-GII-strainmagni1.pdf}
\end{figure}
\end{column}
\end{columns}
\vspace{-0.25cm}
\includegraphics[width=0.275\textwidth]{twofibers-sameside-strainmagni1.pdf}
\end{frame}

\addtocounter{framenumber}{-1}

\begin{frame}
\frametitle{\vspace{0.2cm}\small Interaction of Debonds: Strain Magnification}
\vspace{-.75cm}
\centering
\begin{columns}[c]
\centering
\begin{column}{0.5\textwidth}
\centering
\begin{figure}
\centering
\includegraphics[width=\columnwidth]{nxk-1-vf60-GI-strainmagni3.pdf}
\end{figure}
\end{column}
\begin{column}{0.5\textwidth}
\centering
\begin{figure}
\centering
\includegraphics[width=\columnwidth]{nxk-1-vf60-GII-strainmagni3.pdf}
\end{figure}
\end{column}
\end{columns}
\vspace{-0.25cm}
\begin{figure}
\centering
\includegraphics[width=0.325\textwidth]{twofibers-sameside-strainmagni3.pdf}
\end{figure}
\end{frame}

\addtocounter{framenumber}{-1}

\begin{frame}
\frametitle{\vspace{0.2cm}\small Interaction of Debonds: Strain Magnification}
\vspace{-.75cm}
\centering
\begin{columns}[c]
\centering
\begin{column}{0.5\textwidth}
\centering
\begin{figure}
\centering
\includegraphics[width=\columnwidth]{nxk-1-vf60-GI-strainmagni21.pdf}
\end{figure}
\end{column}
\begin{column}{0.5\textwidth}
\centering
\begin{figure}
\centering
\includegraphics[width=\columnwidth]{nxk-1-vf60-GII-strainmagni21.pdf}
\end{figure}
\end{column}
\end{columns}
\vspace{-0.25cm}
\begin{figure}
\centering
\includegraphics[width=\textwidth]{twofibers-sameside-strainmagni21.pdf}
\end{figure}
\end{frame}

\begin{frame}
\frametitle{\vspace{0.2cm}\small Interaction of Debonds: Crack Shielding}
\vspace{-.75cm}
\centering
\begin{columns}[c]
\centering
\begin{column}{0.5\textwidth}
\centering
\begin{figure}
\centering
\includegraphics[width=\columnwidth]{nxk-1-vf60-GI-crackshield21.pdf}
\end{figure}
\end{column}
\begin{column}{0.5\textwidth}
\centering
\begin{figure}
\centering
\includegraphics[width=\columnwidth]{nxk-1-vf60-GII-crackshield21.pdf}
\end{figure}
\end{column}
\end{columns}
\begin{figure}
\centering
\includegraphics[width=\textwidth]{twofibers-sameside-crackshielding21.pdf}
\end{figure}
\end{frame}

\addtocounter{framenumber}{-1}

\begin{frame}
\frametitle{\vspace{0.2cm}\small Interaction of Debonds: Crack Shielding}
\vspace{-.75cm}
\centering
\begin{columns}[c]
\centering
\begin{column}{0.5\textwidth}
\centering
\begin{figure}
\centering
\includegraphics[width=\columnwidth]{nxk-1-vf60-GI-crackshield3.pdf}
\end{figure}
\end{column}
\begin{column}{0.5\textwidth}
\centering
\begin{figure}
\centering
\includegraphics[width=\columnwidth]{nxk-1-vf60-GII-crackshield3.pdf}
\end{figure}
\end{column}
\end{columns}
\begin{figure}
\centering
\includegraphics[width=\textwidth]{twofibers-sameside-crackshielding3.pdf}
\end{figure}
\end{frame}

\addtocounter{framenumber}{-1}

\begin{frame}
\frametitle{\vspace{0.2cm}\small Interaction of Debonds: Crack Shielding}
\vspace{-.75cm}
\centering
\begin{columns}[c]
\centering
\begin{column}{0.5\textwidth}
\centering
\begin{figure}
\centering
\includegraphics[width=\columnwidth]{nxk-1-vf60-GI-crackshield1.pdf}
\end{figure}
\end{column}
\begin{column}{0.5\textwidth}
\centering
\begin{figure}
\centering
\includegraphics[width=\columnwidth]{nxk-1-vf60-GII-crackshield1.pdf}
\end{figure}
\end{column}
\end{columns}
\begin{figure}
\centering
\includegraphics[width=\textwidth]{twofibers-sameside-crackshielding1.pdf}
\end{figure}
\end{frame}

\subsection{$0^{\circ}$ ply thickness}

\begin{frame}
\frametitle{\vspace{0.2cm}\small Effect of $0^{\circ}$ ply thickness}
\vspace{-.75cm}
\centering
\begin{columns}[c]
\centering
\begin{column}{0.5\textwidth}
\centering
\begin{figure}
\centering
\includegraphics[width=\columnwidth]{1x1-i-vf60-GI-zero.pdf}
\end{figure}
\end{column}
\begin{column}{0.5\textwidth}
\centering
\begin{figure}
\centering
\includegraphics[width=\columnwidth]{1x1-i-vf60-GII-zero.pdf}
\end{figure}
\end{column}
\end{columns}
\begin{figure}
\centering
\includegraphics[width=\textwidth]{onefiber-sameside-crackshielding1.pdf}
\end{figure}
\end{frame}

\addtocounter{framenumber}{-1}

\begin{frame}
\frametitle{\vspace{0.2cm}\small Effect of $0^{\circ}$ ply thickness}
\vspace{-.75cm}
\centering
\begin{columns}[c]
\centering
\begin{column}{0.5\textwidth}
\centering
\begin{figure}
\centering
\includegraphics[width=\columnwidth]{1x1-i-vf60-GI-free.pdf}
\end{figure}
\end{column}
\begin{column}{0.5\textwidth}
\centering
\begin{figure}
\centering
\includegraphics[width=\columnwidth]{1x1-i-vf60-GII-free.pdf}
\end{figure}
\end{column}
\end{columns}
\begin{figure}
\centering
\includegraphics[width=\textwidth]{onefiber-sameside-crackshielding1.pdf}
\end{figure}
\end{frame}

\addtocounter{framenumber}{-1}

\begin{frame}
\frametitle{\vspace{0.2cm}\small Effect of $0^{\circ}$ ply thickness}
\vspace{-.75cm}
\centering
\begin{columns}[c]
\centering
\begin{column}{0.5\textwidth}
\centering
\begin{figure}
\centering
\includegraphics[width=\columnwidth]{1x1-i-vf60-GI-coupling.pdf}
\end{figure}
\end{column}
\begin{column}{0.5\textwidth}
\centering
\begin{figure}
\centering
\includegraphics[width=\columnwidth]{1x1-i-vf60-GII-coupling.pdf}
\end{figure}
\end{column}
\end{columns}
\begin{figure}
\centering
\includegraphics[width=\textwidth]{onefiber-sameside-crackshielding1.pdf}
\end{figure}
\end{frame}

\addtocounter{framenumber}{-1}

\begin{frame}
\frametitle{\vspace{0.2cm}\small Effect of $0^{\circ}$ ply thickness}
\vspace{-.75cm}
\centering
\begin{columns}[c]
\centering
\begin{column}{0.5\textwidth}
\centering
\begin{figure}
\centering
\includegraphics[width=\columnwidth]{1x1-i-vf60-GI.pdf}
\end{figure}
\end{column}
\begin{column}{0.5\textwidth}
\centering
\begin{figure}
\centering
\includegraphics[width=\columnwidth]{1x1-i-vf60-GII.pdf}
\end{figure}
\end{column}
\end{columns}
\begin{figure}
\centering
\includegraphics[width=\textwidth]{onefiber-sameside-crackshielding1.pdf}
\end{figure}
\end{frame}

\subsection{$90^{\circ}$ ply thickness}

\begin{frame}
\frametitle{\vspace{0.2cm}\small Effect of $90^{\circ}$ ply thickness}
\vspace{-.75cm}
\centering
\begin{columns}[c]
\centering
\begin{column}{0.5\textwidth}
\centering
\begin{figure}
\centering
\includegraphics[width=\columnwidth]{nxk-1-vf60-GI-thick.pdf}
\end{figure}
\end{column}
\begin{column}{0.5\textwidth}
\centering
\begin{figure}
\centering
\includegraphics[width=\columnwidth]{nxk-1-vf60-GII-thick.pdf}
\end{figure}
\end{column}
\end{columns}
\begin{figure}
\centering
\includegraphics[width=\textwidth]{twofibers-sameside-crackshielding11.pdf}
\end{figure}
\end{frame}

\addtocounter{framenumber}{-1}

\begin{frame}
\frametitle{\vspace{0.2cm}\small Effect of $90^{\circ}$ ply thickness}
\vspace{-.75cm}
\centering
\begin{columns}[c]
\centering
\begin{column}{0.5\textwidth}
\centering
\begin{figure}
\centering
\includegraphics[width=\columnwidth]{nxk-1-vf60-GI-thin.pdf}
\end{figure}
\end{column}
\begin{column}{0.5\textwidth}
\centering
\begin{figure}
\centering
\includegraphics[width=\columnwidth]{nxk-1-vf60-GII-thin.pdf}
\end{figure}
\end{column}
\end{columns}
\begin{figure}
\centering
\includegraphics[width=\textwidth]{twofibers-sameside-crackshielding11.pdf}
\end{figure}
\end{frame}

\addtocounter{framenumber}{-1}

\begin{frame}
\frametitle{\vspace{0.2cm}\small Effect of $90^{\circ}$ ply thickness}
\vspace{-.75cm}
\centering
\begin{columns}[c]
\centering
\begin{column}{0.5\textwidth}
\centering
\begin{figure}
\centering
\includegraphics[width=\columnwidth]{nxk-1-vf60-GI-extremethin.pdf}
\end{figure}
\end{column}
\begin{column}{0.5\textwidth}
\centering
\begin{figure}
\centering
\includegraphics[width=\columnwidth]{nxk-1-vf60-GII-extremethin.pdf}
\end{figure}
\end{column}
\end{columns}
\begin{figure}
\centering
\includegraphics[width=\textwidth]{twofibers-sameside-crackshielding21-thin.pdf}
\end{figure}
\end{frame}

\subsection{Summary}

\begin{frame}
\frametitle{Summary}
\vspace{-0.5cm}
\centering
\begin{itemize}[label=\ding{212}]
\item No effect of  $90^{\circ}$ ply thickness can be observed when $t_{90^{\circ}}$ is at least $\sim3\diameter_{fiber}$
\item Only if $t_{90^{\circ}}$ is reduced to $1\diameter_{fiber}$, ERR is reduced for a given level of applied strain, i.e. debond growth is delayed to higher levels of applied strain ($G\sim\varepsilon_{applied}^{2}$)
\item No effect of  $0^{\circ}$ ply thickness can be observed when $\nicefrac{t_{0^{\circ}}}{t_{90^{\circ}}}>1$
\item A small difference can be observed when  $t_{0^{\circ}}=t_{90^{\circ}}$, due to the smaller bending stiffness of a thinner $0^{\circ}$ layer
\end{itemize}
\end{frame}

\section{Measuring Transverse Cracks Propagation}

\subsection{Prelude}

\begin{frame}
\frametitle{\vspace{0.3cm}\small Micromechanics of Initiation}
\vspace{-0.5cm}
\centering
\captionsetup[subfigure]{labelfont=footnotesize}
\begin{tikzpicture}

%Tikz axis starts%

\draw[->,black, line width = 5pt] (-4,0) -- (-4,3);
\draw[->,black, line width = 5pt] (-4,0) -- (-4,-3);
\node  at (-4,0) {\includegraphics[height=0.6\textheight]{0-90_2_symm}};
\draw[<->,red, line width = 1.25pt] (-4.35,-0.5) -- (-3.65,-0.625);
\node[red, below]  at (-4,-0.65) {\tiny$\mathbf{\sim 15\ mm}$};
\node[above, right]  at (-3.8,3) {\Huge$\mathbf{\varepsilon}$};
\node[below, left]  at (-4.2,-3) {\Huge$\mathbf{\varepsilon}$};

\draw[->,black, line width = 5pt] (-2,0) -- (-2,3);
\draw[->,black, line width = 5pt] (-2,0) -- (-2,-3);
\node  at (-2,0) {\includegraphics[height=0.6\textheight]{0-90_symm_edge}};
\draw[<->,red, line width = 1.25pt] (-2.46,-0.6) -- (-1.55,-0.6);
\node[red, below]  at (-2,-0.6) {\tiny$\mathbf{\sim 2\ mm}$};
\node[above, right]  at (-1.8,3) {\Huge$\mathbf{\varepsilon}$};
\node[below, left]  at (-2.2,-3) {\Huge$\mathbf{\varepsilon}$};

\draw[->,black, line width = 5pt] (1,0) -- (1,3);
\draw[->,black, line width = 5pt] (1,0) -- (1,-3);
\node  at (1,0) {\includegraphics[height=0.6\textheight]{0-90_symm_debondetail}};
\draw[<->,red, line width = 1.25pt] (0.075,-0.15) -- (1.1,-0.15);
\draw[red, line width = 1.5pt] (-1.25,0.0) arc(0:360:2);
\node[red, below]  at (0.55,-0.16) {\tiny$\mathbf{\sim 8\ \mu m}$};
\node[above, right]  at (1.2,3) {\Huge$\mathbf{\varepsilon}$};
\node[below, left]  at (0.8,-3) {\Huge$\mathbf{\varepsilon}$};

\node[anchor=south west]  at (3.25,1.55) {\scriptsize\textbf{Left:}};
\node[anchor=west]  at (3.25,1.5) {\scriptsize front view of $\left[0,90_{2}\right]_{S}$,};
\node[anchor=north west]  at (3.25,1.5) {\scriptsize visual inspection.};

\node[anchor=south west]  at (3.25,0.05) {\scriptsize\textbf{Center:}};
\node[anchor=west]  at (3.25,0) {\scriptsize  edge view of $\left[0,90\right]_{S}$,};
\node[anchor=north west]  at (3.25,0) {\scriptsize optical microscope.};

\node[anchor=south west]  at (3.25,-1.45) {\scriptsize\textbf{Right:}};
\node[anchor=west]  at (3.25,-1.5) {\scriptsize edge view of $\left[0,90\right]_{S}$,};
\node[anchor=north west]  at (3.25,-1.5) {\scriptsize optical microscope.};

%Tikz axis ends%

\end{tikzpicture}
\end{frame}

\subsection{Materials}

\begin{frame}
\frametitle{\vspace{0.3cm}\small Materials}
\vspace{-0.75cm}
\centering
\begin{columns}[c]
\begin{column}{0.5\textwidth}
\begin{tikzpicture}
\node at (2,0) {};
\node  at (4.25,0) {\includegraphics[height=0.75\textheight]{0-90_2_symm}};
\draw[->,black, line width = 2.5pt,loosely dashed] (4.25,-3.25) -- (4.25,3.25);
\draw[->,black, line width = 2.5pt,loosely dashed] (2,0) -- (6.5,0);
\node[anchor=south west] at (4.25,3.1) {$\mathbf{L}$};
\node[anchor=north west] at (6.35,0) {$\mathbf{T}$};
\end{tikzpicture}
\end{column}
\begin{column}{0.5\textwidth}
\centering
\scriptsize
\begin{itemize}[label=\ding{212}]
\item Glass fiber/epoxy prepreg
\item $V_{f}\sim50-55\%$
\item Lay-up $\left[0^{\circ},90^{\circ}_{2}\right]$
\item Manufacturing: manual lay-up + vacuum bag + hot press
\item Cutting and polishing of specimens
\item Specimen size (nominal): $200\ mm\times15\ mm\times1.85\ mm$
\end{itemize}
\vspace{-0.25cm}
\begin{table}
\centering
\scriptsize
\begin{tabular}{cc}
\multicolumn{2}{c}{UD properties (measured)}\\
\midrule
$E_{L}\ \left[GPa\right]$&38.5\\
$E_{T}\ \left[GPa\right]$&15.3\\
$\nu_{LT}\ \left[-\right]$&0.3\\
$G_{LT}\ \left[GPa\right]$&3.5\\
$\varepsilon^{lim}_{T}\ \left[\%\right]$&0.43\\
$\sigma^{lim}_{T}\ \left[MPa\right]$&60\\
$T_{stress\ free}\ \left[^{\circ}C\right]$&116.3\\
\end{tabular}
\end{table}
\end{column}
\end{columns}
\end{frame}

\subsection{Equipment}

\begin{frame}
\frametitle{\vspace{0.3cm}\small Equipment}
\vspace{-1cm}
\centering
\begin{columns}[c]
\begin{column}{0.55\textwidth}
\begin{figure}
\centering
\includegraphics[width=0.95\columnwidth]{instron-and-chamber}
\end{figure}
\end{column}
\begin{column}{0.45\textwidth}
\centering
\scriptsize
\vspace{-2.5cm}
\begin{itemize}[label=\ding{212}]
\item Universal electro-mechanical testing machine (Instron 3366)
\item Environmental chamber (Instron), temperature range: $-100^{\circ}$ to $+350^{\circ}C$
\item Extensometer (Instron), gauge length: $50\ mm$
\end{itemize}
\end{column}
\end{columns}
\end{frame}

\subsection{Testing procedure}

\begin{frame}
\frametitle{\vspace{0.3cm}\small Testing procedure}
\vspace{-1.1cm}
\centering
\begin{columns}[c]
\begin{column}{0.5\textwidth}
\begin{figure}
\centering
\includegraphics[width=\columnwidth]{testing-procedure}
\end{figure}
\end{column}
\begin{column}{0.5\textwidth}
\centering
\begin{figure}
\centering
\includegraphics[width=\columnwidth]{testing-procedure-temp}
\end{figure}
\end{column}
\end{columns}
\begin{table}
\centering
\small
\begin{tabular}{cccc}
&\multicolumn{3}{c}{Cross-head speed}\\
\cline{2-4}\\[-7.5pt]
T&1 $\left[\nicefrac{mm}{min}\right]$&50 $\left[\nicefrac{mm}{min}\right]$&500 $\left[\nicefrac{mm}{min}\right]$\\
\midrule
room&3 sp.&4 sp.&4 sp.\\
110$\left[^{\circ}\right]$&3 sp.&4 sp.&4 sp.\\
\end{tabular}
\end{table}
\end{frame}

\subsection{Crack density}

\begin{frame}
\frametitle{\vspace{0.2cm}\small Crack density}
\vspace{-.75cm}
\centering
\begin{columns}[c]
\centering
\begin{column}{0.5\textwidth}
\centering
\begin{figure}
\centering
\includegraphics[width=\columnwidth]{ncracks.pdf}
\end{figure}
\end{column}
\begin{column}{0.5\textwidth}
\centering
\begin{figure}
\centering
\includegraphics[width=\columnwidth]{crackdensity.pdf}
\end{figure}
\end{column}
\end{columns}
\scriptsize
\begin{equation*}
\rho_{kn}=\frac{n\ cracks}{L_{gauge}}t_{90^{\circ}}
\end{equation*}
\end{frame}

\addtocounter{framenumber}{-1}

\begin{frame}
\frametitle{\vspace{0.2cm}\small Crack density}
\vspace{-.75cm}
\centering
\begin{columns}[c]
\centering
\begin{column}{0.5\textwidth}
\centering
\begin{figure}
\centering
\includegraphics[width=\columnwidth]{ncracks-v1.pdf}
\end{figure}
\end{column}
\begin{column}{0.5\textwidth}
\centering
\begin{figure}
\centering
\includegraphics[width=\columnwidth]{crackdensity-v1.pdf}
\end{figure}
\end{column}
\end{columns}
\scriptsize
\begin{equation*}
\rho_{kn}=\frac{n\ cracks}{L_{gauge}}t_{90^{\circ}}
\end{equation*}
\end{frame}

\addtocounter{framenumber}{-1}

\begin{frame}
\frametitle{\vspace{0.2cm}\small Crack density}
\vspace{-.75cm}
\centering
\begin{columns}[c]
\centering
\begin{column}{0.5\textwidth}
\centering
\begin{figure}
\centering
\includegraphics[width=\columnwidth]{ncracks-v50.pdf}
\end{figure}
\end{column}
\begin{column}{0.5\textwidth}
\centering
\begin{figure}
\centering
\includegraphics[width=\columnwidth]{crackdensity-v50.pdf}
\end{figure}
\end{column}
\end{columns}
\scriptsize
\begin{equation*}
\rho_{kn}=\frac{n\ cracks}{L_{gauge}}t_{90^{\circ}}
\end{equation*}
\end{frame}

\addtocounter{framenumber}{-1}

\begin{frame}
\frametitle{\vspace{0.2cm}\small Crack density}
\vspace{-.75cm}
\centering
\begin{columns}[c]
\centering
\begin{column}{0.5\textwidth}
\centering
\begin{figure}
\centering
\includegraphics[width=\columnwidth]{ncracks-v500.pdf}
\end{figure}
\end{column}
\begin{column}{0.5\textwidth}
\centering
\begin{figure}
\centering
\includegraphics[width=\columnwidth]{crackdensity-v500.pdf}
\end{figure}
\end{column}
\end{columns}
\scriptsize
\begin{equation*}
\rho_{kn}=\frac{n\ cracks}{L_{gauge}}t_{90^{\circ}}
\end{equation*}
\end{frame}

\begin{frame}[plain]
\frametitle{}
\end{frame}

\end{document}
