%%%%%%%%%%%%%%%%%%%%%%% file template.tex %%%%%%%%%%%%%%%%%%%%%%%%%
%
% This is a general template file for the LaTeX package SVJour3
% for Springer journals.          Springer Heidelberg 2010/09/16
%
% Copy it to a new file with a new name and use it as the basis
% for your article. Delete % signs as needed.
%
% This template includes a few options for different layouts and
% content for various journals. Please consult a previous issue of
% your journal as needed.
%
%%%%%%%%%%%%%%%%%%%%%%%%%%%%%%%%%%%%%%%%%%%%%%%%%%%%%%%%%%%%%%%%%%%
%
% First comes an example EPS file -- just ignore it and
% proceed on the \documentclass line
% your LaTeX will extract the file if required
%
\RequirePackage{fix-cm}
%
%\documentclass{svjour3}                     % onecolumn (standard format)
%\documentclass[smallcondensed]{svjour3}     % onecolumn (ditto)
%\documentclass[smallextended]{svjour3}       % onecolumn (second format)
\documentclass[twocolumn]{svjour3}          % twocolumn
%
\smartqed  % flush right qed marks, e.g. at end of proof
%
\usepackage{graphicx}
%
% \usepackage{mathptmx}      % use Times fonts if available on your TeX system
%
% insert here the call for the packages your document requires
%\usepackage{latexsym}
% etc.
%
% please place your own definitions here and don't use \def but
% \newcommand{}{}
%
% Insert the name of "your journal" with
\journalname{International Journal of Fracture}
%
\begin{document}

\title{Finite Element solution of the linear elastic fiber/matrix interface crack problem%\thanks{Grants or other notes
%about the article that should go on the front page should be
%placed here. General acknowledgments should be placed at the end of the article.}
}
\subtitle{Convergence properties and mode mixity of the Virtual Crack Closure Technique}

%\titlerunning{Short form of title}        % if too long for running head

\author{Luca Di Stasio         \and
             Janis Varna  \and
             Zoubir Ayadi
}

%\authorrunning{Luca Di Stasio         \and
%             Janis Varna  \and
%             Zoubir Ayadi} % if too long for running head

\institute{Luca Di Stasio \at
              Lule\aa\ University of Technology, University Campus, SE-97187 Lule\aa, Sweden\\
              Universit\'e de Lorraine, EEIGM, IJL, 6 Rue Bastien Lepage, F-54010 Nancy, France \\
              \email{luca.di.stasio@ltu.se}           %  \\
%             \emph{Present address:} of F. Author  %  if needed
           \and
           Janis Varna \at
              Lule\aa\ University of Technology, University Campus, SE-97187 Lule\aa, Sweden\\
              \email{janis.varna@ltu.se}
           \and
         Zoubir Ayadi \at
              Universit\'e de Lorraine, EEIGM, IJL, 6 Rue Bastien Lepage, F-54010 Nancy, France \\
              \email{zoubir.ayadi@univ-lorraine.fr}
}

\date{Received: date / Accepted: date}
% The correct dates will be entered by the editor


\maketitle

\begin{abstract}
The bi-material interface arc crack has been the focus of interest in the composite community, where it is usually referred to as the fiber-matrix interface crack. In this work, we investigate the convergence properties of the Virtual Crack Closure Technique (VCCT) when applied to the evaluation of the Mode I, Mode II and total Energy Release Rate of the fiber-matrix interface crack in the context of the Finite Element Method (FEM). We first propose a synthetic vectorial formulation of the VCCT. Thanks to this formulation, we then study the convergence properties of the method, both analytically and numerically. It is found that Mode I and Mode II ERR possess a logarithmic dependency with respect to the size of the elements in the crack tip neighborhood, while the total ERR is independent of element size.
\keywords{First keyword \and Second keyword \and More}
% \PACS{PACS code1 \and PACS code2 \and more}
% \subclass{MSC code1 \and MSC code2 \and more}
\end{abstract}

\section{Introduction}\label{sec:intro}
Bi-material interfaces represent the basic load transfer mechanism at the heart of Fiber Reinforced Polymer Composite (FRPC) materials. They are present at the macroscale, in the form of adhesive joints; at the mesoscale, as interfaces between layers with different orientations; at the microscale, as fiber-matrix interfaces. Bi-material interfaces have for long attracted the attention of researchers in Fracture Mechanics~\cite{Comninou1990,Hills1993}, due to their hidden complexity.\\
The problem was first addressed in the 1950's by Williams~\cite{Williams1959}, who derived through a linear elastic asymptotic analysis the stress distribution around an \emph{open} crack (i.e. with crack faces nowhere in contact for any size of the crack) between two infinite half-planes of dissimilar materials and found the existence of a strong oscillatory behavior in the stress singularity at the crack tip of the form 

%\begin{equation}\label{eq:singularitywilliams}
%r^{-\frac{1}{2}}\sin\left(\varepsilon\log r\right)\quad\text{with}\quad\varepsilon=\frac{1}{2\pi}\log\left(\frac{1-\beta}{1+\beta}\right);
%\end{equation}
%
%in which $\beta$ is one of the two parameters introduced by Dundurs~\cite{Dundurs1969} to characterize bi-material interfaces:
%
%\begin{equation}\label{eq:dundursbeta}
%\beta=\frac{\mu_{2}\left(\kappa_{1}-1\right)-\mu_{1}\left(\kappa_{2}-1\right)}{\mu_{2}\left(\kappa_{1}+1\right)+\mu_{1}\left(\kappa_{2}+1\right)}
%\end{equation}

where $\kappa=3-4\nu$ in plane strain and $\kappa=\frac{3-4\nu}{1+\nu}$ in plane stress, $\mu$ is the shear modulus, $\nu$ Poisson's coefficient, and indexes $1,2$ refer to the two bulk materials joined at the interface. Defining $a$ as the length of the crack, it was found that the size of the oscillatory region is in the order of $10^{-6}a$~\cite{Erdogan1963}. Given the oscillatory behaviour of the crack tip singularity of the stress field of Eq.~\ref{eq:singularitywilliams}, the definition of Stress Intensity Factor (SIF) $\lim_{r\rightarrow 0}\sqrt{2\pi r}\sigma$ ceases to be valid as it returns logarithmically infinite terms~\cite{Comninou1990}. Furthermore, it implies that the Mode mixity problem at the crack tip is ill-posed.\\
It was furthermore observed, always in the context of Linear Elastic Fracture Mechanics (LEFM), that an interpenetration zone exists close to the crack tip~\cite{England1965,Malyshev1965} with a length in the order of $10^{-4}$~\cite{England1965}. Following conclusions firstly proposed in\cite{Malyshev1965}, the presence of a \emph{contact zone} in the crack tip neighborhood, of a length to be determined from the solution of the elastic problem, was introduced in~\cite{Comninou1977} and shown to provide a physically consistent solution to the straight bi-material interface crack problem.\\
The curved bi-material interface crack, more often refered to as the fiber-matrix interface crack (or debond) due to its relevance in FRPCs, was first treated by England~\cite{England1966} and by Perlman and Sih~\cite{Perlman1967}, who provided the analytical solution of stress and displacement fields for a circular inclusion with respectively a single debond and an arbitrary number of debonds. Building on their work, Toya~\cite{Toya1974} particularized the solution and provided the expression of the Energy Release Rate (ERR) at the crack  tip. The same problems exposed previously for the \emph{open} straight bi-material were shown to exist also for the \emph{open} fiber-matrix interface crack: the presence of strong oscillations in the crack tip singularity and crack face interpenetration after a critical initial flaw size.\footnote{For the fiber-matrix interface crack, flaw size is measured in terms of the angle $\Delta\theta$ subtended by half of the arc-crack, i.e. $a=2\Delta\theta$.}\\
In order to treat cases more complex than the single partially debonded fiber in an infinite matrix of~\cite{England1966,Perlman1967,Toya1974}, numerical studies followed. In the 1990's, Par{\'{\i}}s and collaborators~\cite{Paris1996} developed a Boundary Element Method (BEM) with the use of discontinuous singular elements at the crack tip and the Virtual Crack Closure Integral (VCCI)~\cite{Irwin1958} for the evaluation of the Energy Release Rate (ERR). They validated their results~\cite{Paris1996} with respect to Toya's analytical solution~\cite{Toya1974} and analyzed the effect of BEM interface discretization on the stress field in the neighborhood of the crack tip~\cite{DelCano1997}. Following Comninou's work on the straight crack~\cite{Comninou1977}, they furthermore recognized the importance of contact to retrieve a physical solution avoiding interpenetration~\cite{Paris1996} and studied the effect of the contact zone on debond ERR~\cite{Varna1997a}. Their algorithm was then applied to investigate the fiber-matrix interface crack under different geometrical configurations and mechanical loadings ~\cite{Paris2007,Correa2007,Correa2011,Correa2013,Correa2014,Sandino2016,Sandino2018}.\\
Recently the Finite Element Method (FEM) was also applied to the solution of the fiber-matrix interface crack problem~\cite{Zhuang2018,Varna2017,Zhuang2018a}, in conjunction with the Virtual Crack Closure Technique (VCCT)~\cite{Rybicki1977,Krueger2004} for the evaluation of the ERR at the crack tip. In~\cite{Zhuang2018}, the authors validated their model with respect to the BEM results of~\cite{Paris1996}, but no analysis of the effect of the discretization in the crack tip neighborhood comparable to~\cite{DelCano1997} was proposed. Thanks to the interest in evaluating the ERR of interlaminar delamination, different studies exist in the literature on the effect of mesh discretization on Mode I and Mode II ERR of the bi-material interface crack when evaluated with the VCCT in the context of the FEM~\cite{Sun1987,Manoharan1990,Sun1997}. However, no comparable analysis can be found in the literature on the application of the VCCT to the fiber-matrix interface crack (circular bi-material interface crack) problem in the context of a linear elastic FEM solution. It is this gap that the present work aims to address. We first present the FEM formulation of the problem, together with the main geometrical characteristics, material properties, boundary conditions and loading. We then propose a vectorial formulation of the VCCT and express the Mode I and Mode II ERR in terms of the FEM natural variables. With this tool, we derive an analytical estimate of the ERR convergence and compare it with numerical results.

\section{FEM formulation of the fiber-matrix interface crack problem}\label{sec:femmodel}


\begin{acknowledgements}
Luca Di Stasio gratefully acknowledges the support of the European School of Materials (EUSMAT) through the DocMASE Doctoral Programme and the European Commission through the Erasmus Mundus Programme.
\end{acknowledgements}

% BibTeX users please use one of
\bibliographystyle{spbasic}      % basic style, author-year citations
%\bibliographystyle{spmpsci}      % mathematics and physical sciences
%\bibliographystyle{spphys}       % APS-like style for physics
\bibliography{refs}   % name your BibTeX data base

%% Non-BibTeX users please use
%\begin{thebibliography}{}
%%
%% and use \bibitem to create references. Consult the Instructions
%% for authors for reference list style.
%%
%\bibitem{RefJ}
%% Format for Journal Reference
%Author, Article title, Journal, Volume, page numbers (year)
%% Format for books
%\bibitem{RefB}
%Author, Book title, page numbers. Publisher, place (year)
%% etc
%\end{thebibliography}

\end{document}
% end of file template.tex

