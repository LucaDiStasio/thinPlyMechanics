\documentclass[review]{elsarticle}

\usepackage{amsmath}
\usepackage{subcaption}
\usepackage[usenames]{xcolor}
\usepackage{lineno,hyperref}
\modulolinenumbers[5]

\journal{Engineering Fracture Mechanics}

%%%%%%%%%%%%%%%%%%%%%%%
%% Elsevier bibliography styles
%%%%%%%%%%%%%%%%%%%%%%%
%% To change the style, put a % in front of the second line of the current style and
%% remove the % from the second line of the style you would like to use.
%%%%%%%%%%%%%%%%%%%%%%%

%% Numbered
%\bibliographystyle{model1-num-names}

%% Numbered without titles
%\bibliographystyle{model1a-num-names}

%% Harvard
%\bibliographystyle{model2-names.bst}\biboptions{authoryear}

%% Vancouver numbered
%\usepackage{numcompress}\bibliographystyle{model3-num-names}

%% Vancouver name/year
%\usepackage{numcompress}\bibliographystyle{model4-names}\biboptions{authoryear}

%% APA style
%\bibliographystyle{model5-names}\biboptions{authoryear}

%% AMA style
%\usepackage{numcompress}\bibliographystyle{model6-num-names}

%% `Elsevier LaTeX' style
\bibliographystyle{elsarticle-num}
%%%%%%%%%%%%%%%%%%%%%%%

\begin{document}

\begin{frontmatter}

\title{Finite Element solution of the fiber/matrix interface crack problem: convergence properties and mode mixity of the Virtual Crack Closure Technique}
%\tnotetext[mytitlenote]{Fully documented templates are available in the elsarticle package on \href{http://www.ctan.org/tex-archive/macros/latex/contrib/elsarticle}{CTAN}.}

%% Group authors per affiliation:
%\author{Luca Di Stasio\fnref{myfootnote}}
%\address{Radarweg 29, Amsterdam}
%\fntext[myfootnote]{Since 1880.}

%% or include affiliations in footnotes:
\author[nancy,lulea]{Luca Di Stasio}
\author[lulea]{Janis Varna}
\author[nancy]{Zoubir Ayadi}
%\ead[url]{www.elsevier.com}

%\author[mysecondaryaddress]{Global Customer Service\corref{mycorrespondingauthor}}
%\cortext[mycorrespondingauthor]{Corresponding author}
%\ead{support@elsevier.com}

\address[nancy]{Universit\'e de Lorraine, EEIGM, IJL, 6 Rue Bastien Lepage, F-54010 Nancy, France}
\address[lulea]{Lule\aa\ University of Technology, University Campus, SE-97187 Lule\aa, Sweden}

\begin{abstract}
\noindent
\textcolor{purple}{{\em Priority}: 3}\\
\textcolor{purple}{{\em Target journal(s)}: Engineering Fracture Mechanics, Theoretical and Applied Fracture Mechanics, International Journal of Fracture}\\
\end{abstract}

%\begin{keyword}
%\texttt{elsarticle.cls}\sep \LaTeX\sep Elsevier \sep template
%\MSC[2010] 00-01\sep  99-00
%\end{keyword}

\end{frontmatter}

\linenumbers

\section{Introduction}

Bi-material interfaces represent the basic load transfer mechanism at the heart of Fiber Reinforced Polymer Composite (FRPC) materials. They are present at the macroscale, in the form of adhesive joints; at the mesoscale, as interfaces between layers with different orientations; at the microscale, as fiber-matrix interfaces. Bi-material interfaces have for long attracted the attention of researchers in Fracture Mechanics~\cite{Comninou1990,Hills1993}, due to their hidden complexity.\\
The problem was first addressed in the 1950's by Williams~\cite{Williams1959}, who derived through a linear elastic asymptotic analysis the stress distribution around an \emph{open} crack (with crack faces nowhere in contact for any size of the crack) between two infinite half-planes of dissimilar materials and found the existence of a strong oscillatory behavior in the stress singularity at the crack tip of the form 

\begin{equation}\label{eq:singularitywilliams}
r^{-\frac{1}{2}}\sin\left(\varepsilon\log r\right)\quad\text{with}\quad\varepsilon=\frac{1}{2\pi}\log\left(\frac{1-\beta}{1+\beta}\right);
\end{equation}

in which $\beta$ is one of the two parameters introduced by Dundurs~\cite{Dundurs1969} to characterize bi-material interfaces:

\begin{equation}\label{eq:dundursbeta}
\beta=\frac{\mu_{2}\left(\kappa_{1}-1\right)-\mu_{1}\left(\kappa_{2}-1\right)}{\mu_{2}\left(\kappa_{1}+1\right)+\mu_{1}\left(\kappa_{2}+1\right)}
\end{equation}

where $\kappa=3-4\nu$ in plane strain and $\kappa=\frac{3-4\nu}{1+\nu}$ in plane stress, $\mu$ is the shear modulus, $\nu$ Poisson's coefficient, and indexes $1,2$ refer to the two bulk materials joined at the interface. Defining $a$ as the length of the crack, it was found that the size of the oscillatory region is in the order of $10^{-6}a$~\cite{Erdogan1963}. Given the oscillatory behaviour of the crack tip singularity of the stress field of Eq.~\ref{eq:singularitywilliams}, the definition of Stress Intensity Factor (SIF) $\lim_{r\rightarrow 0}\sqrt{2\pi r}\sigma$ ceases to be valid as it returns logarithmically infinite terms~\cite{Comninou1990}. Furthermore, it implies that the Mode mixity problem at the crack tip is ill-posed.\\
It was furthermore observed, always in the context of Linear Elastic Fracture Mechanics (LEFM), that an interpenetration zone exists close to the crack tip~\cite{England1965,Malyshev1965} with a length in the order of $10^{-4}$~\cite{England1965}. Following conclusions firstly proposed in\cite{Malyshev1965}, the presence of a \emph{contact zone} in the crack tip neighborhood, of a length to be determined from the solution of the elastic problem, was introduced in~\cite{Comninou1977} and shown to provide a physically consistent solution to the straight bi-material interface crack problem.\\
The curved bi-material interface crack, more often refered to as the fiber-matrix interface due to its relevance in FRPCs, was first treated by Perlman and Sih~\cite{Perlman1967}, who provided the analytical solution of stress and displacement fields for a circular inclusion with an arbitrary number of cracks. Building on their work, Toya~\cite{Toya1974} particularized the solution for an inclusion with a single interface crack (or debond) and provided the expression of the Energy Release Rate (ERR) at the crack  tip.

\section{Vectorial formulation of the Virtual Crack Closure Technique (VCCT)}

\section{Formulation of the ERR with respect to the FEM solution's variables}

\section{Convergence analysis}

\subsection{Analytical considerations}

\subsection{Numerical results}

\section{Conclusions \& Outlook}

\section*{Acknowledgements}

Luca Di Stasio gratefully acknowledges the support of the European School of Materials (EUSMAT) through the DocMASE Doctoral Programme and the European Commission through the Erasmus Mundus Programme.

\bibliography{refs}

\end{document}
