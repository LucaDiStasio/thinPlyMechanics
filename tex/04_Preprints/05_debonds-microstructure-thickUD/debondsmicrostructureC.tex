\documentclass[review]{elsarticle}

\usepackage{amsmath}
\usepackage{dirtytalk}
\usepackage{subcaption}
\usepackage[usenames]{xcolor}
\usepackage{lineno,hyperref}
\modulolinenumbers[5]

\journal{Composites Part A}

%%%%%%%%%%%%%%%%%%%%%%%
%% Elsevier bibliography styles
%%%%%%%%%%%%%%%%%%%%%%%
%% To change the style, put a % in front of the second line of the current style and
%% remove the % from the second line of the style you would like to use.
%%%%%%%%%%%%%%%%%%%%%%%

%% Numbered
%\bibliographystyle{model1-num-names}

%% Numbered without titles
%\bibliographystyle{model1a-num-names}

%% Harvard
%\bibliographystyle{model2-names.bst}\biboptions{authoryear}

%% Vancouver numbered
%\usepackage{numcompress}\bibliographystyle{model3-num-names}

%% Vancouver name/year
%\usepackage{numcompress}\bibliographystyle{model4-names}\biboptions{authoryear}

%% APA style
%\bibliographystyle{model5-names}\biboptions{authoryear}

%% AMA style
%\usepackage{numcompress}\bibliographystyle{model6-num-names}

%% `Elsevier LaTeX' style
\bibliographystyle{elsarticle-num}
%%%%%%%%%%%%%%%%%%%%%%%

\begin{document}

\begin{frontmatter}

\title{Growth of transverse cracks from multiple adjacent debonds: debond-debond interaction between rows of partially debonded fibers in UD composites}
%\tnotetext[mytitlenote]{Fully documented templates are available in the elsarticle package on \href{http://www.ctan.org/tex-archive/macros/latex/contrib/elsarticle}{CTAN}.}

%% Group authors per affiliation:
%\author{Luca Di Stasio\fnref{myfootnote}}
%\address{Radarweg 29, Amsterdam}
%\fntext[myfootnote]{Since 1880.}

%% or include affiliations in footnotes:
\author[nancy,lulea]{Luca Di Stasio}
\author[lulea]{Janis Varna}
\author[nancy]{Zoubir Ayadi}
%\ead[url]{www.elsevier.com}

%\author[mysecondaryaddress]{Global Customer Service\corref{mycorrespondingauthor}}
%\cortext[mycorrespondingauthor]{Corresponding author}
%\ead{support@elsevier.com}

\address[nancy]{Universit\'e de Lorraine, EEIGM, IJL, 6 Rue Bastien Lepage, F-54010 Nancy, France}
\address[lulea]{Lule\aa\ University of Technology, University Campus, SE-97187 Lule\aa, Sweden}

\begin{abstract}
\noindent
%\textcolor{purple}{{\em Priority}: 1}\\
%\textcolor{purple}{{\em Target journal(s)}: Composites Part B: Engineering, Composites Part A: Applied Science and Manufacturing, Composite Structures, Journal of Composite Materials, Composite Communications}\\
The effects of crack shielding, fiber content and ratio of $0^{\circ}$ to $90^{\circ}$ ply thickness on fiber/matrix debond growth in thin cross-ply laminates are investigated with Representative Volume Elements (RVEs) of different ordered microstructures. Debond growth is characterized by the estimation of the Energy Release Rates (ERRs) using the Virtual Crack Closure Technique (VCCT) and the J-integral. It is found that 
\end{abstract}

\begin{keyword}
Polymer-matrix Composites (PMCs)\sep Thin-ply\sep Transverse Failure \sep Debonding \sep Finite Element Analysis (FEA)
\end{keyword}


\end{frontmatter}

\linenumbers

\section{Introduction}

Since the development of the \emph{spred tow} technology or \say{FUKUI method}~\cite{Kawabe2008,Kawabe2008en}, significant efforts have been directed toward the characterization of \emph{thin-ply} laminates~\cite{Sasayama2003,Yamaguchi2005,Tsai2005,Sihn2007,Yokozeki2008,Yokozeki2010,Saito2012,Arteiro2013,Arteiro2014,Amacher2014,Guillamet2014,Huang2018,Cugnoni2018} and their application to mission-critical structures in the aerospace sector~\cite{Moon2011,Kim2017,Kopp2017,McCarville2018}.\\


\section{RVE models \& FE discretization}

\subsection{Models of Representative Volume Element(RVE)}


\subsection{Finite Element (FE) discretization}


\section{Results \& Discussion}



\section{Conclusions \& Outlook}

\section*{Acknowledgements}

Luca Di Stasio gratefully acknowledges the support of the European School of Materials (EUSMAT) through the DocMASE Doctoral Programme and the European Commission through the Erasmus Mundus Programme.

\bibliography{refs}

\end{document}
