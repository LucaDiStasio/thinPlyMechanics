\documentclass[review]{elsarticle}

\usepackage{amsmath}
\usepackage{booktabs}
\usepackage{nicefrac}
\usepackage{subcaption}
\usepackage[usenames]{xcolor}
\usepackage{lineno,hyperref}
\modulolinenumbers[5]

\journal{Theoretical and Applied Fracture Mechanics}

%%%%%%%%%%%%%%%%%%%%%%%
%% Elsevier bibliography styles
%%%%%%%%%%%%%%%%%%%%%%%
%% To change the style, put a % in front of the second line of the current style and
%% remove the % from the second line of the style you would like to use.
%%%%%%%%%%%%%%%%%%%%%%%

%% Numbered
%\bibliographystyle{model1-num-names}

%% Numbered without titles
%\bibliographystyle{model1a-num-names}

%% Harvard
%\bibliographystyle{model2-names.bst}\biboptions{authoryear}

%% Vancouver numbered
%\usepackage{numcompress}\bibliographystyle{model3-num-names}

%% Vancouver name/year
%\usepackage{numcompress}\bibliographystyle{model4-names}\biboptions{authoryear}

%% APA style
%\bibliographystyle{model5-names}\biboptions{authoryear}

%% AMA style
%\usepackage{numcompress}\bibliographystyle{model6-num-names}

%% `Elsevier LaTeX' style
\bibliographystyle{elsarticle-num}
%%%%%%%%%%%%%%%%%%%%%%%

\begin{document}

\begin{frontmatter}

\title{Similarity laws of the fiber-matrix interface crack in Fiber-Reinforced Polymer Composites}
%\tnotetext[mytitlenote]{Fully documented templates are available in the elsarticle package on \href{http://www.ctan.org/tex-archive/macros/latex/contrib/elsarticle}{CTAN}.}

%% Group authors per affiliation:
%\author{Luca Di Stasio\fnref{myfootnote}}
%\address{Radarweg 29, Amsterdam}
%\fntext[myfootnote]{Since 1880.}

%% or include affiliations in footnotes:
\author[lulea,nancy]{Luca Di Stasio}
\author[lulea]{Janis Varna}
\author[nancy]{Zoubir Ayadi}
%\ead[url]{www.elsevier.com}

%\author[mysecondaryaddress]{Global Customer Service\corref{mycorrespondingauthor}}
%\cortext[mycorrespondingauthor]{Corresponding author}
%\ead{support@elsevier.com}

\address[lulea]{Lule\aa\ University of Technology, University Campus, SE-97187 Lule\aa, Sweden}
\address[nancy]{Universit\'e de Lorraine, EEIGM, IJL, 6 Rue Bastien Lepage, F-54010 Nancy, France}

\begin{abstract}
\noindent
\\
\end{abstract}

\begin{keyword}
Fiber Reinforced Polymer Composite (FRPC) \sep Debonding \sep Similarity \sep Dimensional analysis
\end{keyword}

\end{frontmatter}

\linenumbers

\section{Introduction}

One of the most promising developments in Fiber Reinforced Polymer Composites (FRPCs) for advanced structural applications is currently represented by \emph{thin-ply} laminates~\cite{Kopp2017}. Constituted by extremely thin plies, with $t_{90^{\circ}}$ as small as just $\sim4-5$ fiber diameters, this family of laminates is characterized by its damage tolerance, in particular the capability of delaying to higher strains and even suppressing the onset and propagation of transverse cracks~\cite{Cugnoni2018}. The recent experimental assessment of transverse cracks suppression in \emph{thin-ply} laminates~\cite{Sasayama2003,Saito2012,Amacher2014} validates the existence of a \emph{ply-thickness} effect~\cite{Amacher2014} at scales $10x$ smaller than those at which it was originally observed at the end of the 1970's~\cite{Bailey1979}.\\
Onset of transverse cracks coincides at the microscopic level with the formation of fiber/matrix interface cracks~\cite{Bailey1981}, or debonds. After the inter-fiber stress~\cite{Asp1996} and strain concentration~\cite{Kies1962} causes the matrix to fail at or close to the fiber interface, debonds grow along the fiber arc direction until a maximum or critical size is reached. If the applied load is increased, debonds move into the matrix or ``kink'' out of the fiber/matrix interface~\cite{Zhang1997,Paris2007}. Coalescence of debonds then occurs, which corresponds macroscopically to through-the-thickness transverse crack propagation~\cite{Zhang1997,Zhuang2018b}. Finally, propagation through the specimen width occurs~\cite{Zhang1997}.\\
Given that \emph{thin-plies}, as previously noted, can reach nowadays thicknesses of just $\sim4-5$ fiber diameters, the characteristic size of the ply, i.e. the thickness $t_{90^{\circ}}$, is now comparable in magnitude to the characteristic size of debonds, i.e. the fiber diameter $2R_{f}$, such that $\nicefrac{t_{90^{\circ}}}{\left(2R_{f}\right)}\sim\mathcal{O}\left(1\right)$. This has motivated in recent years a renewed interest in debond growth modeling~\cite{Zhuang2018,Sandino2016,Varna2017,Sandino2018}. Since the elastic solution to the interface crack problem implies an oscillating solution at the crack tip~\cite{Comninou1977} in the \emph{open} case (crack faces not in contact), Stress Intensify Factors (SIFs) are not defined and debond growth characterization has focused on the determination of Mode I, Mode II and total Energy Release Rate (ERR). Many authors have reported their results in normalized form~\cite{Paris2007,Toya1974,Paris1996}, by defining a reference ERR $G_{0}$. The definition of such reference ERR would be useful to establish similarity laws and thus to allow comparisons between different material systems, scales, loads and microstructural arrangement. However, no agreement can be found in the literature on the very definition of $G_{0}$ and expressions vary between authors. Furthermore, no clear derivation of $G_{0}$ has been proposed. In this brief contribution, we provide a derivation of $G_{0}$ based on arguments of dimensional analysis, material homogenization and fracture mechanics; we then apply the derived expression of reference ERR to the analysis of debond growth in Representative Volume Elements (RVEs) of UD composites and cross-ply laminates. 

\section{Representative Volume Elements (RVEs) and Finite Element (FE) discretization}

In this contribution, we analyze debond initiation and propagation in Representative Volume Elements (RVEs) of Uni-Directional (UD) composites and $\left[0_{m\cdot k\cdot2L}^{\circ},90_{k\cdot2L}^{\circ},0_{m\cdot k\cdot2L}^{\circ}\right]$ laminates. Given a global reference frame with axis $x$, $y$ and $z$, both types of composites are modeled as plates lying in the $x-y$ plane, with the through-the-thickness direction thus aligned with the $z$ axis. The UD composite $0^{\circ}$ direction is parallel to the $y$ axis, while the cross-ply $0^{\circ}$ direction is parallel to the $x$ axis. Both composites are loaded in tension along the $x$ axis, which thus corresponds to: transverse loading of the UD specimen; axial loading of the cross-ply specimen. In both composites, damage is present only in the form of fiber/matrix interface cracks, or debonds. In cross-plies, debonds are assumed to be present only in the central ${90^{\circ}}$ ply. Given that: first, in the presence of a load in the $x$-direction, in both lay-ups the $y$-strain due to Poisson's effect is very small; second, debond size is assumed to be considerably larger in the fiber than in the arc direction~\cite{Zhang1997}; third, we are interested in studying debond growth along the arc direction; we can consider $2D$ models under plane strain conditions defined in the $x-z$ plane. Although generalized plane strain conditions would represent a more appropriate modeling choice, it would limit the ability to compare with previous results. Simple plane strain conditions are thus preferred. UD composites and $90^{\circ}$ plies are characterized by a regular microstructure following a square-packing configuration of fibers, built through the repetition of a one-fiber unit cell along the horizontal and the vertical direction. This unit cell is a square with the center occupied by one fiber of radius $R_{f}=1\ \mu m$ and the rest of the element constituted by matrix. The size of the one-fiber unit cell is $2L\times 2L$, such that:

\begin{equation}\label{eq:LVf}
L=\frac{R_{f}}{2}\sqrt{\frac{\pi}{V_{f}}},
\end{equation}

where $V_{f}$ is the fiber volume fraction, here assumed to be $60\%$. It is worth to specify at this point that the choice $R_{f}=1\ \mu m$ is arbitrary and stems from the fact that ERR, in the contest of a linear elastic solution as the one considered in this article, is proportional to the geometrical dimensions of the model. Simplicity is thus the main reason for this choice. Also, $V_{f}$ is always the same in the one-fiber unit cell and the entire RVE, i.e. no fiber clustering is analyzed in this work. In the case of cross-ply laminates, the $0^{\circ}$ layer is homogenized with properties evaluated according to the Concentric Cylinders Assembly with Self-Consistent Shear (CCA-SCS) model~\cite{Hashin1983,Christensen1979}. A glass fiber-epoxy system is considered for both UDs and cross-ply laminates. Material properties are reported in Table~\ref{tab:phaseprop}.

\begin{table}[!htbp]
 \centering
 \caption{Summary of mechanical properties of fiber, matrix and UD layer.}%$E$ stands for Young's modulus, $\mu$ for shear modulus and $\nu$ for Poisson's ratio. Indexes $L$ and $T$ stand respectively for \emph{longitudinal} and \emph{transverse}.}
 \begin{tabular}{ccccccc}
\textbf{Material} & \textbf{$V_{f}\left[\%\right]$}\  & \textbf{$E_{L}\left[GPa\right]$}\ & \textbf{$E_{T}\left[GPa\right]$}\  & \textbf{$G_{LT}\left[GPa\right]$} &\textbf{$\nu_{LT}\left[-\right]$} & \textbf{$\nu_{TT}\left[-\right]$} \\
\midrule
Glass fiber &-   & 70.0 & 70.0  & 29.2 & 0.2  & 0.2\\
Epoxy    &-& 3.5 & 3.5   & 1.25 &  0.4& 0.4\\
UD&60.0&43.442&13.714& 4.315& 0.273&0.465\\
\end{tabular}
\label{tab:phaseprop}
\end{table}

The use of coupling conditions allows the study of a Repeating Unit Cells (RUC) of reduced size with respect to the corresponding RVE, which translates in a gain in terms of computational time and memory usage during the evaluation of Finite Element (FE) solution. The RVEs studied in this article are reported in Figure~\ref{fig:rves} with the corresponding RUC highlighted by dashed line (in blue in the online color version) and with symmetry and coupling conditions represented by rollers (\includegraphics[scale=0.5]{roller.pdf}). Details about the central one-fiber unit cell are shown in Figure~\ref{fig:ruc}. Notice that the analysis, in terms of stresses and Energy Release Rate (ERR), is conducted on this central one-fiber unit cell, both in the case of an undamaged and of a partially debonded fiber.\\
Nomenclature and main features of the RVEs studied are described in the following.

\begin{description}
\item [$\mathbf{n\times k-free}$, Figure~\ref{fig:rves-a}: ]UD composite with thickness $t_{UD}=k\cdot2L$, where $k$ is the number of fiber ``rows'' in the vertical (through-the-thickness) direction and $2L$ the side length of the one-fiber unit cell as defined in Equation~\ref{eq:LVf}. Debonds appear only in the central fiber ``row'' with $n-1$ fully bonded fibers in between, where $n$ is the number of fibers present in the RUC along the horizontal direction.
\item [$\mathbf{n\times k-coupling}$, Figure~\ref{fig:rves-b}: ]UD composite with multiple fiber ``rows'' containing debonds with $n-1$ fully bonded fibers in between, where $n$ is the number of fibers present in the RUC along the horizontal direction. Fiber ``rows'' containing debonds are separated along the vertical (through-the-thickness) direction by $k-1$ ``rows'' of fully bonded fibers, where $k$ is the number of fiber ``rows'' present in the RUC along the vertical direction. Debonds that are vertically aligned are placed on the same side of their respective fiber. To model this configuration, conditions of coupling of the vertical displacement are applied on the top side.
\item [$\mathbf{n\times k-asymm}$, Figure~\ref{fig:rves-c}: ]same as in $n\times k-coupling$, but debonds that are vertically aligned are placed on opposite sides of their respective fiber. To model this configuration, the following set of conditions is applied to the upper boundary:

\begin{equation}\label{eq:asymm}
\begin{aligned}
u_{z}\left(x,kL\right)-u_{z}\left(0,kL\right)&=-\left(u_{z}\left(-x,kL\right)-u_{z}\left(0,kL\right)\right)\\
u_{x}\left(x,kL\right)&=-u_{x}\left(-x,kL\right)
\end{aligned}
\end{equation}

which represent conditions of anti-symmetric coupling~\cite{DiStasio2019b}.

\item [$\mathbf{n\times k-m\times t_{90^{\circ}}}$, Figure~\ref{fig:rves-d}: ]Cross-ply laminate with $90^{\circ}$ layer thickness $t_{90^{\circ}}=k\cdot2L$ and $0^{\circ}$ layer thickness $t_{0^{\circ}}=m\cdot t_{90^{\circ}}$. $k$ is the number of fiber ``rows'' in the vertical (through-the-thickness) direction of the $90^{\circ}$ layer and $2L$ the side length of the one-fiber unit cell as defined in Equation~\ref{eq:LVf}. Debonds are present only in the central fiber ``row'' of the $90^{\circ}$ layer with $n-1$ fully bonded fibers in between, where $n$ is the number of fibers present in the RUC along the horizontal direction.
\end{description}

%\begin{figure}[!h]
%\centering
%    \subfigure[$n\times k-free$]{\label{fig:rves-a}\raisebox{0.07\textheight}{\includegraphics[width=0.4\textwidth]{thickPlyUD.pdf}}}\quad 
%    \subfigure[$n\times k-coupling$]{\label{fig:rves-b}\includegraphics[width=0.4\textwidth]{coupling.pdf}}\\
%    \subfigure[$n\times k-asymm$]{\label{fig:rves-c}\raisebox{0.0275\textheight}{\includegraphics[width=0.4\textwidth]{asymm.pdf}}}\quad
%    \subfigure[$n\times k-m\times t_{90^{\circ}}$]{\label{fig:rves-d}\includegraphics[width=0.4\textwidth]{ThickPlyCP.pdf}}
%\caption{Composite RVEs and corresponding RUCs analyzed.}\label{fig:rves}
%\end{figure}

Every RUC is symmetric with respect to the horizontal ($x$) direction, thus only half of it is modeled in the FE solution through the use of symmetry boundary conditions on the bottom side. Conditions of coupling of the horizontal displacement are applied on the left and right side, to model the repetition of the RUC along the horizontal direction. A tensile load is applied on the right and left side in the form of displacement $\bar{u}_{x}=\pm\bar{\varepsilon}_{x}nL$ with $\bar{\varepsilon}_{x}=1\%$. The debond has a size of $2\Delta\theta$ (see Figure~\ref{fig:ruc}), with $\Delta\theta\geq0$ ($\Delta\theta=0$ is the case of no damage at all). For large debonds ($\Delta\theta\geq 60^{\circ}-80^{\circ}$), a region called \emph{contact zone}, of size $\Delta\Phi$ to be determined by the solution itself, appears at the crack tip. Correct resolution of this behavior requires the imposition of conditions of non-interpenetration of the crack faces. Crack face contact is assumed to be frictionless.

\begin{figure}[!h]
\centering
        \includegraphics[height=0.15\textheight]{RUC.pdf}
\caption{One-fiber unit cell and main parameters characterizing the debonding process.}\label{fig:ruc}
\end{figure}

The FE solution is obtained using Abaqus~\cite{abq12} using second order, 2D, plane strain triangular (CPE6) and rectangular (CPE8) elements. To accurately resolve the singularity at the crack tip, a regular mesh of only rectangular elements is used with almost unitary aspect ratio and angular size $\delta=0.05^{\circ}$. The crack faces are represented as element-based surfaces with frictionless small-sliding contact pair interaction.

\section{Dimensional analysis}

We first recall that the Energy Release Rate $G$ has units of energy $E$ per unit area:

\begin{equation}\label{eq:errunits}
\left[G\right]=\frac{E}{L^{2}},
\end{equation}

where $L$ stands for unit of length. By algebraic manipulation of Equation~\ref{eq:errunits} we can write the units of ERR as

\begin{equation}\label{eq:errunitsreworked}
\frac{E}{L^{2}}=\frac{F\cdot L}{L^{2}}=\frac{F}{L^{2}}\frac{L}{L}L,
\end{equation}

where $F$ stands for unit of force. We recognize that, in Equation~\ref{eq:errunitsreworked}

\begin{equation}\label{eq:sigmaepsunits}
\frac{F}{L^{2}}=\left[\sigma\right]\qquad\frac{L}{L}=\left[\varepsilon\right],
\end{equation}

where $\sigma$ and $\varepsilon$ are respectively stress and strain. The reference Energy Release Rate is thus dimensionally equivalent to a reference stress $\sigma_{ref}$ times a reference strain $\varepsilon_{ref}$ times a refence length $l_{ref}$ and we can write

\begin{equation}\label{eq:G}
G_{0}\sim\sigma_{ref}\varepsilon_{ref}l_{ref}.
\end{equation}

\section{Linear Elastic Fracture Mechanics (LEFM) considerations}

In the case of uniaxial loading, we can assume that: in a stress-controlled experiment, $\sigma_{ref}$ is equal to the applied stress $\sigma_{0}$ and $\varepsilon_{ref}$ to the average strain $\varepsilon_{0}$ in the Representative Volume Element (RVE); in a strain-controlled experiment, $\varepsilon_{ref}$ is equal to the applied strain $\varepsilon_{0}$ and $\sigma_{ref}$ to the average stress $\sigma_{av}$ in the Representative Volume Element (RVE).\\
Under the assumption of linear elastic material constituents, we have, respectively for a stress- and strain-controlled experiment:

\begin{equation}\label{eq:elasticresponse}
\varepsilon_{av}=E_{homo}\sigma_{0}\qquad\sigma_{av}=E_{homo}\varepsilon_{0},
\end{equation}

where $E_{homo}$ is a homogenized RVE Young's modulus which measures the RVE elastic response in the presence of different material phases and damage. It is worth to point out here that, as we are interested in studying debond growth in the context of transverse crack onset, RVEs are loaded in the direction transverse to the fibers in the layer where debonds are present. Furthermore, we consider RVEs that are 2-dimensional and under the assumption of plane strain or plane stress conditions. This implies, considering the elastic response of a transversely isotropic material in its plane of transverse isotropy (indeces $2-3$, index $1$ corresponds to the axis of rotational symmetry) with no damage, that~\cite{Timoshenko1987,Mantic2009}

\begin{equation}\label{eq:Ehomo}
E_{homo}=\frac{E_{2}}{1-\nu_{12}\nu_{21}}\qquad E_{homo}=E_{2},
\end{equation}

respectively for plane strain and plane stress, with $E_{2}$ the homogenized transverse Young's modulus of the ply and $\nu_{12}$, $\nu_{21}$ the major and minor Poisson's ratios. Notice that homogenized elastic properties depend on consittuents' elastic properties and theIn the presence of damage, we can assume the homogenized Young's modulus of the damaged RVE to be a fraction of the undamaged modulus $E_{homo}^{0}$ (expressed in Eq.~\ref{eq:Ehomo}):

\begin{equation}\label{eq:Edamage}
E_{homo}=f\left(\Delta\theta\right)E_{homo}^{0},
\end{equation}

where $0<f\left(\Delta\theta\right)<1$ is a function of the damage state in the material, in this case represented by the debond half-size $\Delta\theta$ (debond size is $2\Delta\theta$). By substituting Eq.~\ref{eq:elasticresponse}, Eq.~\ref{eq:Ehomo} and Eq.~\ref{eq:Edamage} in Eq.~\ref{eq:G}, we have

\begin{equation}\label{eq:Gexpandedstraincontrol}
G_{0}\sim f\left(\Delta\theta\right)\frac{E_{2}}{1-\nu_{12}\nu_{21}}\varepsilon_{0}^{2}l_{ref}\quad G_{0}\sim f\left(\Delta\theta\right)E_{2}\varepsilon_{0}^{2}l_{ref},
\end{equation}

respectively for plane strain and plane stress conditions under applied strain $\varepsilon_{0}$, and

\begin{equation}\label{eq:Gexpandedstresscontrol}
G_{0}\sim f\left(\Delta\theta\right)\frac{1-\nu_{12}\nu_{21}}{E_{2}}\sigma_{0}^{2}l_{ref}\quad G_{0}\sim f\left(\Delta\theta\right)\frac{\sigma_{0}^{2}}{E_{2}}l_{ref},
\end{equation}

respectively for plane strain and plane stress conditions under applied strain $\sigma_{0}$. Notice that, incidentally: the plane strain expression in Eq.~\ref{eq:Gexpandedstresscontrol} is the same as the ERR expression used for \textit{in-situ} strenght modeling in~\cite{Camanho2006} and derived in~\cite{Laws1983} by considering the fiber-reinforced polymer as a 3-phase composite with one phase constituted by sharp voids (cracks)\footnote{Often expressed as $\Lambda_{22}^{0}=2\left(\frac{1}{E_{2}}-\frac{\nu_{12}^{2}}{E_{1}}\right)$, which can be shown to be equivalent to $\Lambda_{22}^{0}=2\frac{1-\nu_{12}\nu_{21}}{E_{2}}$ by recalling that $\nu_{21}=\frac{E_{2}}{E_{1}}\nu_{12}$.}; the plane stress expression in Eq.~\ref{eq:Gexpandedstresscontrol} is the same as the Mode I ERR in~\cite{Varna2018}, derived from the definition of ERR and problem geometry.\\
In accord with the classic Linear Elastic Fracture Mechanics (LEFM), the Energy Release Rate is directly proportional to the crack size $a$~\cite{Tada2000}. Given that $a=R_{f}2\Delta\theta$ for debonds, where $R_{f}$ is the fiber radius, it is reasonable to assume $R_{f}$ as the reference length:

\begin{equation}\label{eq:lref}
l_{ref}=R_{f}.
\end{equation}

The reference Energy Release Rate thus becomes

\begin{equation}\label{eq:Gfinalstraincontrol}
G_{0}\sim f\left(\Delta\theta\right)\frac{E_{2}}{1-\nu_{12}\nu_{21}}\varepsilon_{0}^{2}R_{f}\quad G_{0}\sim f\left(\Delta\theta\right)E_{2}\varepsilon_{0}^{2}R_{f},
\end{equation}

respectively for plane strain and plane stress conditions under applied strain $\varepsilon_{0}$, and

\begin{equation}\label{eq:Gfinalstresscontrol}
G_{0}\sim f\left(\Delta\theta\right)\frac{1-\nu_{12}\nu_{21}}{E_{2}}\sigma_{0}^{2}R_{f}\quad G_{0}\sim f\left(\Delta\theta\right)\frac{\sigma_{0}^{2}}{E_{2}}R_{f},
\end{equation}

respectively for plane strain and plane stress conditions under applied strain $\sigma_{0}$.

\section{Similarity and geometry correction factor}

In agreement with the classic Fracture Mechanics (FM) treatment~\cite{Tada2000}, we can recognize in the function $f\left(\Delta\theta\right)$ of Eq.~\ref{eq:Gfinalstraincontrol} and Eq.~\ref{eq:Gfinalstresscontrol} the geometry correction factor ($f\left(a\right)$ or $Y$) that establishes the relation of similarity~\cite{Barenblatt2006}

\begin{equation}\label{eq:Gsim}
K=f\left(a\right)\sigma\sqrt{a}\quad\text{or}\quad G=f^{2}\left(a\right)\frac{\sigma^{2}}{E}a
\end{equation}

between the Stress Intensity Factor (SIF) $K$ and Energy Release Rate (ERR) $G$ of a generic configuration of structural and crack geometry and the solution for a Center Crack (CC) in an infinite plate

\begin{equation}\label{eq:Gsim}
K_{CC}=\sigma\sqrt{a}\quad\text{or}\quad G_{CC}=\frac{\sigma^{2}}{E}a,
\end{equation}

where the crack size is $2a$. It thus seems reasonable to look for a functional form of $f\left(\Delta\theta\right)$ in Eq.~\ref{eq:Gfinalstraincontrol} and Eq.~\ref{eq:Gfinalstresscontrol} among known analytical solutions of SIFs and ERRs and such that a physically-meaningful similarity between the two configurations could be established.

\begin{itemize}
\item \textbf{Straight central crack in an infinite isotropic plate under far-field transverse tension~\cite{Tada2000}.}

\begin{equation}\label{eq:straightcrack}
f_{I}\left(\Delta\theta\right)=\sin\left(\Delta\theta\right)\quad f_{II}\left(\Delta\theta\right)=0
\end{equation}

It is the simplest choice, based on considering the debond chord $2R_{f}\sin{\Delta\theta}$ as its representative size. However, as apparent in Eq.~\ref{eq:straightcrack}, there is no Mode II geometry correction factor available (a straight crack in transverse tension propagates only in Mode I) and it is thus not suited to establish a relation of similarity with debond ERR, which is Mode II dominated for large $\Delta\theta$.

\item \textbf{Inclined central crack in an infinite isotropic plate under far-field tension~\cite{Tada2000}.}

\begin{equation}\label{eq:inclinedcrack}
\begin{aligned}
f_{I}\left(\Delta\theta\right)=&\sin\left(\Delta\theta\right)\sin^{4}\left(\frac{\pi}{2}-\Delta\theta\right)\\ f_{II}\left(\Delta\theta\right)=&\sin\left(\Delta\theta\right)\sin^{2}\left(\frac{\pi}{2}-\Delta\theta\right)\cos^{2}\left(\frac{\pi}{2}-\Delta\theta\right)
\end{aligned}
\end{equation}

A first attempt to amend the shortcomings of Eq.~\ref{eq:straightcrack} is to consider the geometry correction factor of the inclined crack subjected to transverse load. However, $f_{II}\left(90^{\circ}\right)=0$ in Eq.~\ref{eq:inclinedcrack}, which makes also this choice not a good choice to establish a similarity relation with debond ERR (Mode II ERR is well-defined and different from $0$ at $\Delta\theta=90^{\circ}$ for debonds).

\item \textbf{Circular crack in an infinite isotropic plate under far-field tension transverse to crack's chord~\cite{Ioakmidis1977}.}

\begin{equation}\label{eq:circularcrack}
\begin{aligned}
f_{I}&\left(\Delta\theta\right)=\frac{G_{I}}{\sigma_{ref}\varepsilon_{ref}R}=\frac{1}{2}\sin\left(\Delta\theta\right)\times\\\times&\left(\frac{1-\sin^{2}\left(\frac{\Delta\theta}{2}\right)\cos^{2}\left(\frac{\Delta\theta}{2}\right)}{1+\sin^{2}\left(\frac{\Delta\theta}{2}\right)}\cos\left(\frac{\Delta\theta}{2}\right)+\cos\left(\frac{3}{2}\Delta\theta\right)\right)^{2}\\
 f_{II}&\left(\Delta\theta\right)=\frac{G_{II}}{\sigma_{ref}\varepsilon_{ref}R}=\frac{1}{2}\sin\left(\Delta\theta\right)\times\\\times&\left(\frac{1-\sin^{2}\left(\frac{\Delta\theta}{2}\right)\cos^{2}\left(\frac{\Delta\theta}{2}\right)}{1+\sin^{2}\left(\frac{\Delta\theta}{2}\right)}\sin\left(\frac{\Delta\theta}{2}\right)+\sin\left(\frac{3}{2}\Delta\theta\right)\right)^{2}
\end{aligned}
\end{equation}

The geometry correction factors of Eq.~\ref{eq:circularcrack} (shown in Fig.~\ref{fig:curvedcrackgeomcorr}) present a solution to the issues characterising Eq.~\ref{eq:straightcrack} and Eq.~\ref{eq:inclinedcrack}: Mode II is defined and both modes are defined and continuous for $\Delta\theta=0^{\circ}-180^{\circ}$. By evaluating the elastic properties $E_{2}$, $\nu_{12}$ and $\nu_{21}$ at the value of $V_{f}$ of the composite under consideration and substituting Eq.~\ref{eq:circularcrack} in Eq.~\ref{eq:Gfinalstresscontrol} and Eq.~\ref{eq:Gfinalstraincontrol}, we obtain the Mode I $G_{I0}$ and Mode II $G_{II0}$ ERR of a circular crack of size $a=2\Delta\theta$ in an infinite isotropic medium, which has properties equivalent to the elastic properties of a UD composite of fiber volume fraction $V_{f}$ in its plane of transverse isotropy, obtained by application of the Concentric Cylinders Assembly~\cite{Hashin1983} with Self-Consistent Shear~\cite{Christensen1979} (CCA-SCS) model. The following expressions are derived,

\begin{enumerate}

\item under plane strain conditions and applied far-field transverse strain:

\begin{equation}\label{eq:Gcircularcrack1}
\begin{aligned}
G_{I0}&=f_{I}\left(\Delta\theta\right)\frac{E_{2}\left(V_{f}\right)}{1-\nu_{12}\left(V_{f}\right)\nu_{21}\left(V_{f}\right)}\varepsilon_{0}^{2}R_{f}\\G_{II0}&=f_{II}\left(\Delta\theta\right)\frac{E_{2}}{1-\nu_{12}\nu_{21}}\varepsilon_{0}^{2}R_{f};
\end{aligned}
\end{equation}

\item under plane strain conditions and applied far-field transverse stress:

\begin{equation}\label{eq:Gcircularcrack2}
\begin{aligned}
G_{I0}&=f_{I}\left(\Delta\theta\right)\frac{1-\nu_{12}\left(V_{f}\right)\nu_{21}\left(V_{f}\right)}{E_{2}\left(V_{f}\right)}\sigma_{0}^{2}R_{f}\\G_{II0}&=f_{II}\left(\Delta\theta\right)\frac{1-\nu_{12}\left(V_{f}\right)\nu_{21}\left(V_{f}\right)}{E_{2}\left(V_{f}\right)}\sigma_{0}^{2}R_{f}
\end{aligned}
\end{equation}

\item under plane stress conditions and applied far-field transverse strain:

\begin{equation}\label{eq:Gcircularcrack3}
\begin{aligned}
G_{I0}&=f_{I}\left(\Delta\theta\right)E_{2}\left(V_{f}\right)\varepsilon_{0}^{2}R_{f}\\G_{II0}&=f_{II}\left(\Delta\theta\right)E_{2}\left(V_{f}\right)\varepsilon_{0}^{2}R_{f}
\end{aligned}
\end{equation}

\item under plane stress conditions and applied far-field transverse stress:

\begin{equation}\label{eq:Gcircularcrack4}
\begin{aligned}
G_{I0}&=f_{I}\left(\Delta\theta\right)\frac{\sigma_{0}^{2}}{E_{2}\left(V_{f}\right)}R_{f}\\G_{II0}&=f_{II}\left(\Delta\theta\right)\frac{\sigma_{0}^{2}}{E_{2}\left(V_{f}\right)}R_{f}
\end{aligned}
\end{equation}
\end{enumerate}

This configuration thus establishes a physically-meaningful relation of similarity, as the ratios $\frac{G_{I}}{G_{I0}}=g_{I}\left(\Delta\theta,V_{f}\right)\ \left[-\right]$ and $\frac{G_{II}}{G_{II0}}=g_{II}\left(\Delta\theta,V_{f}\right)\ \left[-\right]$ measure the effect of: the mismatch in elastic properties between phases (in Eq.~\ref{eq:circularcrack} the medium is isotropic); the finite size of the geometry (in Eq.~\ref{eq:circularcrack} the medium is infinite); the interaction with neighboring undamaged and partially debonded fibers, a free surface (in UD composites) or the $0^{\circ}/90^{\circ}$ interface (in cross-ply laminates).
\end{itemize}

\begin{figure}
\includegraphics[width=\textwidth]{curvedcracks.pdf}
\caption{Mode I ($f_{I}$) and Mode II ($f_{II}$) geometry correction functions for a circular crack in infinite isotropic medium. The chord of the crack is normal to the loading direction and the crack size is $a=2\Delta\theta$.}\label{fig:curvedcrackgeomcorr}
\end{figure}

Given that debond ERR for RVEs presented in Section~\ref{} is evaluated under conditions of plane strain and applied transverse strain, the expressions of $G_{I0}$ and $G_{II0}$ in Eq.~\ref{eq:Gcircularcrack1} are adopted in the following.

\section{Effect of elastic properties mismatch}

\begin{figure}[!h]
\centering
    \begin{subfigure}[b]{0.475\textwidth}
        \includegraphics[height=0.225\textheight]{comparescaling-Vf01.pdf}
        \caption{Dimensional ERR.}\label{subfig:comparescalingVf01}
    \end{subfigure} \quad
    \begin{subfigure}[b]{0.475\textwidth}
        \includegraphics[height=0.225\textheight]{comparescaling-Vf01-normalized.pdf}
        \caption{Normalized ERR.}\label{subfig:comparescalingVf01normalized}
    \end{subfigure}

\caption{\textit{Left}: Mode I and Mode II ERR for the circular crack in an infinite isotropic medium ($G_{I0}$ and $G_{II0}$) and for a single partially debonded fiber in an infinite matrix ($1\times 1-free$, $V_{f}=0.1\%$). In $G_{I0}$ and $G_{II0}$, $E_{2}$, $\nu_{12}$ and $\nu_{21}$ are evaluated using the Concentric Cylinders Assembly~\cite{Hashin1983} with Self-Consistent Shear~\cite{Christensen1979} (CCA-SCS) model for $V_{f}=0.1\%$. In both cases, a transverse load is applied in the form of transverse strain $\varepsilon_{x}$ of $1\%$. \textit{Right}: Mode I and Mode II ERR of single partially debonded fiber in an infinite matrix ($1\times 1-free$, $V_{f}=0.1\%$) normalized by Mode I and Mode II ERR of the circular crack in an infinite isotropic medium ($G_{I0}$ and $G_{II0}$).}\label{fig:comparescalingVf01}
\end{figure}

\section{Effect of fiber volume fraction}

\begin{figure}[!h]
\centering
    \begin{subfigure}[b]{0.475\textwidth}
        \includegraphics[height=0.225\textheight]{comparescaling-Vf30.pdf}
        \caption{Dimensional ERR.}\label{subfig:comparescalingVf30}
    \end{subfigure} \quad
    \begin{subfigure}[b]{0.45\textwidth}
        \includegraphics[height=0.225\textheight]{comparescaling-Vf30-normalized.pdf}
        \caption{Normalized ERR.}\label{subfig:comparescalingVf30normalized}
    \end{subfigure}

\caption{\textit{Left}: Mode I and Mode II ERR for the circular crack in an infinite isotropic medium ($G_{I0}$ and $G_{II0}$) and for a single partially debonded fiber in an infinite matrix ($1\times 1-free$, $V_{f}=30\%$). In $G_{I0}$ and $G_{II0}$, $E_{2}$, $\nu_{12}$ and $\nu_{21}$ are evaluated using the Concentric Cylinders Assembly~\cite{Hashin1983} with Self-Consistent Shear~\cite{Christensen1979} (CCA-SCS) model for $V_{f}=30\%$. In both cases, a transverse load is applied in the form of transverse strain $\varepsilon_{x}$ of $1\%$. \textit{Right}: Mode I and Mode II ERR of single partially debonded fiber in an infinite matrix ($1\times 1-free$, $V_{f}=30\%$) normalized by Mode I and Mode II ERR of the circular crack in an infinite isotropic medium ($G_{I0}$ and $G_{II0}$).}\label{fig:comparescalingVf30}
\end{figure}

\begin{figure}[!h]
\centering
    \begin{subfigure}[b]{0.475\textwidth}
        \includegraphics[height=0.225\textheight]{comparescaling-Vf60.pdf}
        \caption{Dimensional ERR.}\label{subfig:comparescalingVf60}
    \end{subfigure} \quad
    \begin{subfigure}[b]{0.475\textwidth}
        \includegraphics[height=0.225\textheight]{comparescaling-Vf60-normalized.pdf}
        \caption{Normalized ERR.}\label{subfig:comparescalingVf60normalized}
    \end{subfigure}

\caption{\textit{Left}: Mode I and Mode II ERR for the circular crack in an infinite isotropic medium ($G_{I0}$ and $G_{II0}$) and for a single partially debonded fiber in an infinite matrix ($1\times 1-free$, $V_{f}=60\%$). In $G_{I0}$ and $G_{II0}$, $E_{2}$, $\nu_{12}$ and $\nu_{21}$ are evaluated using the Concentric Cylinders Assembly~\cite{Hashin1983} with Self-Consistent Shear~\cite{Christensen1979} (CCA-SCS) model for $V_{f}=60\%$. In both cases, a transverse load is applied in the form of transverse strain $\varepsilon_{x}$ of $1\%$. \textit{Right}: Mode I and Mode II ERR of single partially debonded fiber in an infinite matrix ($1\times 1-free$, $V_{f}=60\%$) normalized by Mode I and Mode II ERR of the circular crack in an infinite isotropic medium ($G_{I0}$ and $G_{II0}$).}\label{fig:comparescalingVf60}
\end{figure}
\section{Effect of neighboring fibers}


\bibliography{refs}



\section{Conclusions}

\end{document}
