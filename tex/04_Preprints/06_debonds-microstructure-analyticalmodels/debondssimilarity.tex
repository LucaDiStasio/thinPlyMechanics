%% 
%% Copyright 2019 Elsevier Ltd
%% 
%% This file is part of the 'CAS Bundle'.
%% --------------------------------------
%% 
%% It may be distributed under the conditions of the LaTeX Project Public
%% License, either version 1.2 of this license or (at your option) any
%% later version.  The latest version of this license is in
%%    http://www.latex-project.org/lppl.txt
%% and version 1.2 or later is part of all distributions of LaTeX
%% version 1999/12/01 or later.
%% 
%% The list of all files belonging to the 'CAS Bundle' is
%% given in the file `manifest.txt'.
%% 
%% Template article for cas-dc documentclass for 
%% double column output.

%\documentclass[a4paper,fleqn,longmktitle]{cas-dc}
\documentclass[a4paper,fleqn]{cas-dc}

%\usepackage[authoryear,longnamesfirst]{natbib}
%\usepackage[authoryear]{natbib}
\usepackage[numbers]{natbib}
\usepackage{nicefrac}

%%%Author definitions
\def\tsc#1{\csdef{#1}{\textsc{\lowercase{#1}}\xspace}}
\tsc{WGM}
\tsc{QE}
\tsc{EP}
\tsc{PMS}
\tsc{BEC}
\tsc{DE}
%%%

\begin{document}
\let\WriteBookmarks\relax
\def\floatpagepagefraction{1}
\def\textpagefraction{.001}
\shorttitle{Similarity laws of the fiber-matrix interface crack in polymer composites}
\shortauthors{Luca Di Stasio et~al.}

\title [mode = title]{Similarity laws of the fiber-matrix interface crack in polymer composites}                      
%\tnotemark[1,2]

%\tnotetext[1]{This document is the results of the research
   %project funded by the National Science Foundation.}

%\tnotetext[2]{The second title footnote which is a longer text matter
   %to fill through the whole text width and overflow into
   %another line in the footnotes area of the first page.}



\author[1,2]{Luca {Di Stasio}}%[type=editor,
                        %auid=000,bioid=1,
                        %prefix=Sir,
                        %role=Researcher,
                        %orcid=0000-0001-7511-2910]
%\cormark[1]
%\fnmark[1]
%\ead{cvr_1@tug.org.in}
%\ead[url]{www.cvr.cc, cvr@sayahna.org}

%\credit{Conceptualization of this study, Methodology, Software}

\author[1]{Janis Varna}

\author[2]{Zoubir Ayadi}

%\fnmark[2]
%\ead{cvr3@sayahna.org}
%\ead[URL]{www.sayahna.org}

%\credit{Data curation, Writing - Original draft preparation}

\address[1]{Lule\aa\ University of Technology, University Campus, SE-97187 Lule\aa, Sweden}

\address[2]{Universit\'e de Lorraine, EEIGM, IJL, 6 Rue Bastien Lepage, F-54010 Nancy, France}

%\author%
%[1,3]
%{Rishi T.}
%\cormark[2]
%\fnmark[1,3]
%\ead{rishi@stmdocs.in}
%\ead[URL]{www.stmdocs.in}
%
%\address[3]{STM Document Engineering Pvt Ltd., Mepukada,
%    Malayinkil, Trivandrum 695571, India}
%
%\cortext[cor1]{Corresponding author}
%\cortext[cor2]{Principal corresponding author}
%\fntext[fn1]{This is the first author footnote. but is common to third
%  author as well.}
%\fntext[fn2]{Another author footnote, this is a very long footnote and
%  it should be a really long footnote. But this footnote is not yet
%  sufficiently long enough to make two lines of footnote text.}
%
%\nonumnote{This note has no numbers. In this work we demonstrate $a_b$
%  the formation Y\_1 of a new type of polariton on the interface
%  between a cuprous oxide slab and a polystyrene micro-sphere placed
%  on the slab.
%  }

\begin{abstract}
This template helps you to create a properly formatted \LaTeX\ manuscript.

\noindent\texttt{\textbackslash begin{abstract}} \dots 
\texttt{\textbackslash end{abstract}} and
\verb+\begin{keyword}+ \verb+...+ \verb+\end{keyword}+ 
which
contain the abstract and keywords respectively. 

\noindent Each keyword shall be separated by a \verb+\sep+ command.
\end{abstract}

%\begin{graphicalabstract}
%\includegraphics{figs/grabs.pdf}
%\end{graphicalabstract}
%
%\begin{highlights}
%\item Research highlights item 1
%\item Research highlights item 2
%\item Research highlights item 3
%\end{highlights}

\begin{keywords}
Fiber Reinforced Polymer Composite (FRPC) \sep Debonding \sep Similarity \sep Dimensional analysis
\end{keywords}


\maketitle

\section{Introduction}

One of the most promising developments in Fiber Reinforced Polymer Composites (FRPCs) for advanced structural applications is currently represented by \emph{thin-ply} laminates~\cite{Kopp2017}. Constituted by extremely thin plies, with $t_{90^{\circ}}$ as small as just $\sim4-5$ fiber diameters, this family of laminates is characterized by its damage tolerance, in particular the capability of delaying to higher strains and even suppressing the onset and propagation of transverse cracks~\cite{Cugnoni2018}. The recent experimental assessment of transverse cracks suppression in \emph{thin-ply} laminates~\cite{Sasayama2003,Saito2012,Amacher2014} validates the existence of a \emph{ply-thickness} effect~\cite{Amacher2014} at scales $10x$ smaller than those at which it was originally observed at the end of the 1970's~\cite{Bailey1979}.\\
Onset of transverse cracks coincides at the microscopic level with the formation of fiber/matrix interface cracks~\cite{Bailey1981}, or debonds. After the inter-fiber stress~\cite{Asp1996} and strain concentration~\cite{Kies1962} causes the matrix to fail at or close the fiber interface, debonds grow along the fiber arc direction until a maximum or critical size is reached. If the applied load is increased, debonds move into the matrix or ``kink'' out of the fiber/matrix interface~\cite{Zhang1997,Paris2007}. Coalescence of debonds then occurs, which corresponds macroscopically to through-the-thickness transverse crack propagation~\cite{Zhang1997,Zhuang2018}. Finally, propagation through the specimen width occurs~\cite{Zhang1997}.\\
Given that \emph{thin-plies}, as previously noted, can reach nowadays thicknesses of just $\sim4-5$ fiber diameters, the characteristic size of the ply, i.e. the thickness $t_{90^{\circ}}$, is now comparable in magnitude to the characteristic size of debonds, i.e. the fiber diameter $2R_{f}$, such that $\nicefrac{t_{90^{\circ}}}{\left(2R_{f}\right)}\sim\mathcal{O}\left(1\right)$. This has motivated in recent years a renewed interest in debond growth modeling~\cite{Zhuang2018,Sandino2016,Varna2017,Sandino2018}. Since the elastic solution to the interface crack implies an oscillating solution at the crack tip~\cite{Comninou1977} in the \emph{open} case (crack faces not in contact), Stress Intensify Factors (SIFs) are not defined and debond growth characterization has focused on the determination of Mode I, Mode II and total Energy Release Rate (ERR).

\section{Dimensional analysis}

\section{Representative Volume Elements (RVEs)}






\section{Similarity laws}



\section{Conclusions}



%% Loading bibliography style file
\bibliographystyle{model1-num-names}
%\bibliographystyle{cas-model2-names}

% Loading bibliography database
\bibliography{refs}


%\vskip3pt

%\bio{}
%Author biography without author photo.
%Author biography. Author biography. Author biography.
%Author biography. Author biography. Author biography.
%Author biography. Author biography. Author biography.
%Author biography. Author biography. Author biography.
%Author biography. Author biography. Author biography.
%Author biography. Author biography. Author biography.
%Author biography. Author biography. Author biography.
%Author biography. Author biography. Author biography.
%Author biography. Author biography. Author biography.
%\endbio
%
%\bio{figs/pic1}
%Author biography with author photo.
%Author biography. Author biography. Author biography.
%Author biography. Author biography. Author biography.
%Author biography. Author biography. Author biography.
%Author biography. Author biography. Author biography.
%Author biography. Author biography. Author biography.
%Author biography. Author biography. Author biography.
%Author biography. Author biography. Author biography.
%Author biography. Author biography. Author biography.
%Author biography. Author biography. Author biography.
%\endbio
%
%\bio{figs/pic1}
%Author biography with author photo.
%Author biography. Author biography. Author biography.
%Author biography. Author biography. Author biography.
%Author biography. Author biography. Author biography.
%Author biography. Author biography. Author biography.
%\endbio

\end{document}

