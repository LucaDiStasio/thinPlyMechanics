\documentclass[12pt,a4paper]{article}
\usepackage{composites2019}
\usepackage{booktabs}

\begin{document}
\thispagestyle{empty}

\vspace*{-3.4cm}
\begin{table}[!h]
\begin{tabular}{r}
\hspace*{5.5cm} \scriptsize \textsf{7th ECCOMAS Thematic Conference on the Mechanical Response of Composites} \\
\hspace*{5.5cm} \scriptsize \textsf{ COMPOSITES 2019} \\
\hspace*{5.5cm} \tiny \textsf{A. Turon, P. Maimí \& M. Fagerström (Editors)}
\end{tabular}
\end{table}

\begin{center}
\title{ESTIMATING THE AVERAGE SIZE OF FIBER/MATRIX INTERFACE CRACKS IN UD AND CROSS-PLY LAMINATES}
\end{center}
\begin{center}
\textbf{\underline{Luca Di Stasio}$^{1,2,*}$, Janis Varna$^{1}$, Zoubir Ayadi$^{2}$} \\ [7pt]
\small{$^1$~Lule\aa\ University of Technology, University Campus, SE-97187 Lule\aa, Sweden}  \\  [2pt]  
\small{$^2$~Universit\'e de Lorraine, EEIGM, IJL, 6 Rue Bastien Lepage, F-54010 Nancy, France}  \\  [2pt]
\small{$^*$~\texttt{luca.di.stasio@ltu.se}} \\
\end{center}


\paragraph{Keywords:} Fiber Reinforced Polymer (FRP), Debonding, Linear Elastic Fracture Mechanics (LEFM).

\paragraph{Summary:} \textit{This document provides information and instructions for preparing the (optional) full-length paper for the COMPOSITES 2019 Conference (September 18-20, 2019 in Girona, Spain).}

\section{INTRODUCTION}

The Conference publication will consist of a pen drive containing papers of the contributions received and a printed Book of Abstracts containing a one page version of the accepted abstracts. The authors must submit a full-length paper (max. 12 pages) using the same format of this template. Submission of a full-length paper is not mandatory but authors are strongly encouraged to send it before June 27, 2019.

The deadline date for early registration date is April 30, 2019. Presenting authors must register by June 13, 2019. Papers with authors not registered by this date will be removed from the final program.
Registration closes on September 5, 2019. Further information can be found at the conference website: \texttt{www.composites2019.udg.edu}

\section{RVE MODELS AND FE DISCRETIZATION}

In this contribution, we analyze debond initiation and propagation in Representative Volume Elements (RVEs) of Uni-Directional (UD) composites and $\left[0_{m\cdot k\cdot2L}^{\circ},90_{k\cdot2L}^{\circ},0_{m\cdot k\cdot2L}^{\circ}\right]$ laminates. Given a global reference frame with axis $x$, $y$ and $z$, both types of composites are modeled as plates lying in the $x-y$ plane, with the through-the-thickness direction thus aligned with the $z$ axis. The UD composite $0^{\circ}$ direction is parallel to the $y$ axis, while the cross-ply $0^{\circ}$ direction is parallel to the $x$ axis. Both composites are loaded in tension along the $x$ axis, which thus corresponds to: transverse loading of the UD specimen; axial loading of the cross-ply specimen. In both composites, damage is present only in the form of fiber/matrix interface cracks, or debonds. In cross-plies, debonds are assumed to be present only in the central ${90^{\circ}}$. UD composites and $90^{\circ}$ plies are characterized by a regular microstructure following a square-packing configuration of fibers, built through the repetition of a one-fiber unit cell along the horizontal and the vertical direction. This unit cell is a square with the center occupied by one fiber of radius $R_{f}=1\ \mu m$ and the rest of the element constituted by matrix. The size of the one-fiber unit cell is $2L\times 2L$, such that:

\begin{equation}\label{eq:LVf}
L=\frac{R_{f}}{2}\sqrt{\frac{\pi}{V_{f}}},
\end{equation}

where $V_{f}$ is the fiber volume fraction, here assumed to be $60\%$. It is worth to specify at this point that the choice $R_{f}=1\ \mu m$ is arbitrary and stems from the fact that the linear elastic solution, as the one considered in this article, is proportional to the geometrical dimensions of the model. Simplicity is thus the main reason for this choice. Also, $V_{f}$ is always the same in the one-fiber unit cell and the entire RVE, i.e. no fiber clustering is analyzed in this work. In the case of cross-ply laminates, the $0^{\circ}$ layer is homogenized with properties evaluated according to the Concentric Cylinders Assembly with Self-Consistent Shear (CCA-SCS) model~\cite{Hashin1983,Christensen1979}. A glass fiber-epoxy system is considered for both UDs and cross-ply laminates. Material properties are reported in Table~\ref{tab:phaseprop}.

\begin{table}[!htbp]
 \centering
 \caption{Summary of mechanical properties of fiber, matrix and UD layer.}%$E$ stands for Young's modulus, $\mu$ for shear modulus and $\nu$ for Poisson's ratio. Indexes $L$ and $T$ stand respectively for \emph{longitudinal} and \emph{transverse}.}
 \begin{tabular}{ccccccc}
\textbf{Material} & \textbf{$V_{f}\left[\%\right]$}\  & \textbf{$E_{L}\left[GPa\right]$}\ & \textbf{$E_{T}\left[GPa\right]$}\  & \textbf{$G_{LT}\left[GPa\right]$} &\textbf{$\nu_{LT}\left[-\right]$} & \textbf{$\nu_{TT}\left[-\right]$} \\
\midrule
Glass fiber &-   & 70.0 & 70.0  & 29.2 & 0.2  & 0.2\\
Epoxy    &-& 3.5 & 3.5   & 1.25 &  0.4& 0.4\\
UD&60.0&43.442&13.714& 4.315& 0.273&0.465\\
\end{tabular}
\label{tab:phaseprop}
\end{table}

The use of coupling conditions allows the study of a Repeating Unit Cells (RUC) of reduced size with respect to the corresponding RVE, which translates in a gain in terms of computational time and memory usage during the evaluation of Finite Element (FE) solution. The RVEs studied in this article are reported in Figure~\ref{} with the corresponding RUC highlighted by dashed line (in blue in the online color version) and coupling conditions represented by rollers (\includegraphics[scale=0.5]{roller.pdf}). Details about the central one-fiber unit cell are shown in Figure~\ref{}. Notice that the analysis, in terms of stresses and Energy Release Rate (ERR), is conducted on this central one-fiber unit cell, both in the case of an undamaged and partially debonded fiber.

\section{STRESS-BASED ANALYSIS OF DEBOND INITIATION}

\section{ENERGY-BASED ANALYSIS OF DEBOND PROPAGATION}

\section{CONCLUSIONS}

We are looking forward to receiving your contributions for this conference.

\section*{ACKNOWLEDGEMENTS}

Luca Di Stasio gratefully acknowledges the support of the European School of Materials (EUSMAT) through the DocMASE Doctoral Programme and the European Commission through the Erasmus Mundus Programme.

\bibliographystyle{unsrt}

%\begin{thebibliography}{10}
%
%\bibitem{Barbero} E.J. Barbero, \textit{Finite Element Analysis of Composite Materials}. CRC Press, Boca Raton, 2008.
%\bibitem{Pimenta} S. Pimenta, S.T. Pinho, The effect of recycling on the mechanical response of carbon fibres and their composites. \textit{Composite Structures}, \textbf{94}, 3669-3684, 2012.
%
%\end{thebibliography}

\bibliography{refs}

\end{document}

