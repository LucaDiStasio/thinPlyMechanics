At the end of the second decade of the XXI century, the transportation industry at large faces several challenges that will shape its evolution in the next decade and beyond. The first main challenge is the increasing public awareness and governmental action on climate change, which are increasing the pressure on the industrial sectors responsible for the greatest share of emissions, the transportation industry being one of them, to reduce their environmental footprint. A second challenge lies instead in the renewed push toward price reduction, due to increased competition (as for example the entry of private entities in the market for low-Earth orbit launchers) and innovative business models (like ride-sharing and ride-hailing in the automotive sector or low-cost carriers in civil aviation).\\
A common technical solution strategy to these challenges is the reduction of vehicles' structural mass, while keeping the payload mass constant. By reducing consumption, a reduced weight leads to reduced emissions in fossil-fuels powered vehicles and to increased autonomy in electrical vehicles. By reducing the quantity of materials required in structures, a weight reduction strategy favors a reduction of production costs and thus lower prices. Transportation is however a sector where safety is a paramount concern, and structures must satisfy strict requirements and validation procedures to guarantee their integrity and reliability during service life. This represents a significant constraint which limits the scope of weight reduction strategies.\\
In the last twenty years, the development of a novel type of Fiber-Reinforced Polymer Composite (FRPC) laminates, i.e. \emph{thin-ply} laminates, proposes a solution to these competing requirements (weight with to respect to structural integrity) by providing at the same time weight reduction and increased strength. Several experimental investigations have shown, in fact, that \emph{thin-ply} laminates are capable of delaying, and even suppress, the onset of transverse cracking. Transverse cracks are a kind of sub-critical damage and occur early in the failure process, causing the degradation of elastic properties and favoring other, often more critical, modes of damage (delaminations, fiber breaks). Delay and suppression of transverse cracks were already linked, at the of the 1970's, to the use of thinner plies.
