%%%%%%%%%%%%%%%%%%%%%%%%%%%%%%%%%%%%%%%%%%%%%%%%%%%%%%%%%%%%%%%%%%%%%%%
%      SEC. 1
%%%%%%%%%%%%%%%%%%%%%%%%%%%%%%%%%%%%%%%%%%%%%%%%%%%%%%%%%%%%%%%%%%%%%%%
\section{Introduction and structure of the thesis}

Passion and curiosity should always lie at the heart of the scientific activity, and that ought to be enough to define the value of a research effort~\cite{Weber1917,Shapin2015}. Time is the real arbiter of the significance of a piece of research, as many examples in the history of science show~\cite{Brush1967,Niss2008}\footnote{The Ising-Lenz model is one such example~\cite{Brush1967,Niss2008,Niss2004}. It was suggested by physicist Wilhelm Lenz to his doctoral student Ernst Ising to study phase transitions in ferromagnetic materials. Ising solved it analytically in 1D as part of his Ph.D. defense in 1925, but the solution for a 1D lattice did not show any phase transition. This apparent failure is thought to be the reason of Ising's decision to take a job outside academia. Almost 20 years later, Onsager solved the 2D version of the model and showed the possibility of phase transitions in the Ising-Lenz model. By the time Ising arrived in the USA in 1947, the Ising-Lenz model was already entering the canon of physics and, to his surprise, he was asked if he was ``the Ising" of the ``Ising model".}.However, in these years of increasing mistrust towards scientific research and brewing doubts on the value of universities and research institutes~\cite{Biesta2002,Biesta2004,Santos2012}, it is worthwhile to try to place one's own work into the wider picture of one's own time. It is also a valuable exercise for the researcher who, sensibly, progresses in the work by investigating one detail at a time, to spend a moment away from one's own graphs and equations and see their place in the wider perspective of the world outside the laboratory.\\
It is thus in this spirit that I propose to open this thesis with a reflection on the challenges that the transportation industry faces at the closing of $21^{st}$ century second decade. Against this background, in Chapter~1 \emph{thin-ply} laminates are introduced as a very promising material for innovative structural design and their main characteristics are discussed. I will then focus on the most renown quality of \emph{thin-ply} laminates, i.e. their ability to delay and even suppress onset and propagation of transverse cracking, and discuss the modeling issues that this new material poses. Finally, a link is established with the growth of fiber/matrix interface cracks or, as very often called in the rest of this work, debonds. Chapter~2 opens with an introduction to the main concepts of Fracture Mechanics. The fiber/matrix interface crack is then discussed in more detail, and previous analytical, computational and experimental studies available in the literature are reviewed. The modeling strategy adopted in this thesis is then presented and its implementation described. Finally, Chapter~3 provides a summary of the main results of the this work, organized according to the publications reported in Part~II of the thesis. The first chapter is thus a journey of scales: we start from the challenges of an industrial sector, move to the structural requirements of its products, focus on a promising new material, and focus on understanding the mechanisms of damage initiation in it.

\section{Vision 2030: challenges of the next decade and beyond for the transportation industry}
