L'objectif principal de cette th\`ese est d'\'etudier l'effet de la microstructure sur l'amor\c{c}age et propagation de d\'ecollements entre fibre et matrice. Dans ce but, des mod\`eles de Volume El\'ementaire Repr\'esentatif (VER) des composites unidirectionnels et des stratifi\'es crois\'es sont d\'evelopp\'es, caract\'eris\'es par diff\'erentes configurations des fibres et degr\'e d'endommagement. L'amor\c{c}age du d\'ecollement est analys\'e par rapport \`a la distribution des contraintes \`a l'interface entre fibre et matrice. En revanche, la propagation du d\'ecollement est \'etudi\'ee avec l'approche de la M\'ecanique Lin\'eaire Elastique de la Rupture (MLER), et plus sp\'ecifiquement avec l'\'evaluation du taux de restitution d'\'energie en Mode I et Mode II. Les champs de d\'eplacement et contrainte sont calcul\'es avec la M\'ethode des \'el\'ements finis (MEF) dans le logiciel Abaqus. La d\'etermination des composants du taux de restitution d'\'energie est effectu\'ee avec la technique de fermeture virtuelle de fissure impl\'ement\'ee par l'auteur en langage Python.\\
La solution \'elastique du probl\`eme de d\'ecollement entre fibre et matrice est caract\'eris\'ee par la pr\'esence de deux r\'egimes : celui de fissure ouverte et celui de fissure ferm\'ee. Dans le deuxi\`eme cas, il existe une zone proche de la pointe de fissure o\`u les l\`evres du d\'ecollement sont en contact. Dans le premier cas, le d\'ecollement est ouvert et il n'existe aucun contact entre les l\`evres du d\'ecollement. Dans le r\'egime de fissure ouverte, les champs des d\'eplacements et contraintes pr\'esentent une singularit\'e oscillatoire. Un '\'etude de convergence de la technique de fermeture virtuelle de fissure est donc requis et constitue le premier \'el\'ement du travail de cette th\`ese. Il est constat\'e que le taux de restitution d'\'energie total ne d\'epend pas de la taille des \'el\'ements proches de la pointe de fissure, alors que le taux en Mode I et Mode II pr\'esent une d\'ependance significative de la taille des \'el\'ements dans le cas de fissure ouverte. Il est montr\'e que le taux de restitution d'\'energie en Mode I et Mode II ne converge pas, ce \`a dire que le comportement asymptotique n'est pas limit\'e. Par cons\'equence, il n'est pas possible d'utiliser l'erreur entre it\'erations successives comme mesure de la convergence de la solution et une comparaison est donc n\'ecessaire avec des r\'esultats obtenus avec une autre m\'ethode. Le taux de restitution d'\'energie calcul\'e avec la m\'ethode d'\'el\'ements de fronti\`ere, disponible dans la litt\'erature, est choisi comme r\'ef\'erence.
Ensuite, la propagation de d\'ecollement entre fibre et matrice est \'etudi\'ee dans Volume El\'ementaire Repr\'esentative de : composites unidirectionnels avec \'epaisseur variable, mesur\'e par le nombre des rang\'ees des fibres, de ceux extr\^emement minces (une rang\'ee des fibres) au plus \'epais ; stratifi\'e crois\'e avec un pli central \`a 90$^{\circ}$ d'\'epaisseur variable, mesur\'e par le nombre des rang\'ees des fibres, de ceux extr\^emement minces (une rang\'ee des fibres) au plus \'epais ; composites unidirectionnels \'epais, mod\'elis\'es comme infinis \`a travers l'\'epaisseur. Configurations multiples de l'endommagement sont aussi examin\'ees, qui correspondent \`a diff\'erentes \'etapes du processus d'amor\c{c}age des fissures transverses : d\'ecollements isol\'es ; d\'ecollements interagissant distribu\'es dans la direction d'application de la charge m\'ecanique ; d\'ecollements localis\'es sur fibres cons\'ecutives \`a travers l'\'epaisseur. Entre les r\'esultats plus importants, il est constat\'e que ni l'\'epaisseur du pli \`a 90$^{\circ}$ ni l'\'epaisseur du pli \`a 0$^{\circ}$ influence le taux de restitution d'\'energie du d\'ecollement, diff\'eremment de ce qu'a \'et\'e observ\'e pour les fissures transverses. En revanche, il est montr\'e que le taux de restitution d'\'energie est affect\'e de mani\`ere significative par l'interaction mutuelle entre d\'ecollements dans la direction d'application de la charge et qu'il existe une distance caract\'eristique (mesur\'e par le nombre des fibres sans endommagement) d\'eterminant la r\'egion d'influence entre d\'ecollements.\\
Enfin, la taille du d\'ecollement juste apr\`es l'amor\c{c}age et la taille ultime du d\'ecollement sont estim\'ees \`a partir de l'analyse de la distribution des contraintes \`a l'interface entre fibre et matrice (pour l'amor\c{c}age) et sur la base du crit\`ere de Griffith de la MLER. La taille maximale d'un d\'ecollement dans un stratifi\'e crois\'e est estim\'e dans l'intervalle 40$^{\circ}$ - 60$^{\circ}$, r\'esultat qui est en tr\`es bon accord avec pr\'ec\'edentes observations microscopiques disponibles dans la litt\'erature.
