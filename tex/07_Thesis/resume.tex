A la fin de la deuxi\`eme d\'ecennie du XXI si\`ecle, l'industrie du transport fait face de nombreux d\'efis qui d\'etermineront son \'evolution dans la prochaine d\'ecennie et au-del\`a. Le premier d\'efi est la sensibilisation croissante du grand public aux probl\`emes environnementaux et, en cons\'equence, l'intensification de l'action gouvernementale \`a regard du changement climatique, fait qui d\'etermine une mont\'ee en pression sur tous les secteurs industriels qui sont grands \'emetteurs et dont lesquels le transport fait partie. Le deuxi\`eme d\'efi est repr\'esent\'e en revanche par la course \`a la r\'eduction des prix, d\^u \`a une majeure concurrence (comme, par exemple, dans les secteurs des vecteurs spatiales avec l'entr\'ee des acteurs priv\'es dans le march\'e) et \`a nouveaux mod\`eles commerciaux (comme le co-voiturage dans l'industrie automobile ou les compagnies \`a bas prix dans le transport a\'erien).\\
Une strat\'egie simple mais efficace pour r\'epondre \`a ces d\'efis est la r\'eduction du poids des structures du v\'ehicule, en maintenant constantes la capacit\'e payante. Le premier effet de cette strat\'egie est de r\'eduire la consommation de carburant, fait qu'en revanche conduit \`a une r\'eduction des \'emissions dans les v\'ehicules \`a carburants fossiles et \`a l'augmentation de l'autonomie des v\'ehicules \'electriques. En outre, la r\'eduction de la quantit\'e des mat\'eriaux utilis\'ee dans les structures se traduit souvent en une r\'eduction des co\^uts de fabrications et donc du prix pour l'utilisateur. D'autre c\^ot\'e le transport est un secteur dont l'attention \`a la s\'ecurit\'e est prioritaire, avec des processus de certifications extr\^emement rigoureux. Cette exigence pose des contraintes consid\'erables sur l'ampleur des interventions de r\'eduction du poids des structures.\\
Le d\'eveloppement dans les derni\`eres vingt ans d'un nouvel type de stratifi\'e en polym\`ere avec renfort en fibre, les stratifi\'es thin-ply, propose une solution \`a ce probl\`eme en offrant des stratifies consid\'erablement plus l\'eg\`eres avec, au m\^eme temps, des meilleures propri\'et\'es m\'ecaniques. Nombreux essais ont en fait montr\'e la capacit\'e de ces stratifi\'es de retarder et aussi emp\^echer l'amor\c{c}age et la propagation des fissures transverses. Les fissures transverses repr\'esentent un m\'ecanisme de rupture \`a l'\'echelle des plis qui a lieu plut\^ot t\^ot dans le processus d'endommagement du stratifi\'e et qui conduit \`a la d\'egradation des propri\'et\'es m\'ecaniques du composite et favorise l'apparition des autres formes d'endommagement (d\'elaminage, rupture des fibres) souvent plus critique pour l'int\'egrit\'e de la structure. Dans les ann\'ees 1970, la capacit\'e des stratifies composites de retarder l'amor\c{c}age des fissures transverses \'etait observ\'ee et li\'ee \`a l'\'epaisseur des plis. N\'eanmoins, l'\'epaisseur des thin-plies aujourd'hui sur le march\'e est au moins 10 fois plus petit que celui des plis des ann\'ees 1970. Ce fait se traduit par un changement d'\'echelle du probl\`eme, de millim\`etres \`a microm\`etres. Au niveau microscopique, les fissures transverses sont form\'ees \`a partir de nombreux d\'ecollements (ou d\'ecoh\'esions) entre fibre et matrice connect\'es entre eux. Une compr\'ehension d\'etaill\'ee de m\'ecanismes qui emp\^echent les fissures transverses requiert la connaissance des ph\'enom\`enes d'amor\c{c}age des fissures transverse \`a l'\'echelle microm\'ecanique et donc des conditions favorables \`a l'amor\c{c}age et propagation des d\'ecollements entre fibre et matrice.\\
L'objectif principal de cette th\`ese est d'\'etudier l'effet de la microstructure sur l'amor\c{c}age et propagation de d\'ecollements entre fibre et matrice. Dans ce but, des mod\`eles de Volume El\'ementaire Repr\'esentatif (VER) des composites unidirectionnels et des stratifi\'es crois\'es sont d\'evelopp\'es, caract\'eris\'es par diff\'erentes configurations des fibres et degr\'e d'endommagement. L'amor\c{c}age du d\'ecollement est analys\'e par rapport \`a la distribution des contraintes \`a l'interface entre fibre et matrice. En revanche, la propagation du d\'ecollement est \'etudi\'ee avec l'approche de la M\'ecanique Lin\'eaire Elastique de la Rupture (MLER), et plus sp\'ecifiquement avec l'\'evaluation du taux de restitution d'\'energie en Mode I et Mode II. Les champs de d\'eplacement et contrainte sont calcul\'es avec la M\'ethode des \'el\'ements finis (MEF) dans le logiciel Abaqus. La d\'etermination des composants du taux de restitution d'\'energie est effectu\'ee avec la technique de fermeture virtuelle de fissure impl\'ement\'ee par l'auteur en langage Python.\\
La solution \'elastique du probl\`eme de d\'ecollement entre fibre et matrice est caract\'eris\'ee par la pr\'esence de deux r\'egimes : celui de fissure ouverte et celui de fissure ferm\'ee. Dans le deuxi\`eme cas, il existe une zone proche de la pointe de fissure o\`u les l\`evres du d\'ecollement sont en contact. Dans le premier cas, le d\'ecollement est ouvert et il n'existe aucun contact entre les l\`evres du d\'ecollement. Dans le r\'egime de fissure ouverte, les champs des d\'eplacements et contraintes pr\'esentent une singularit\'e oscillatoire. Un '\'etude de convergence de la technique de fermeture virtuelle de fissure est donc requis et constitue le premier \'el\'ement du travail de cette th\`ese. Il est constat\'e que le taux de restitution d'\'energie total ne d\'epend pas de la taille des \'el\'ements proches de la pointe de fissure, alors que le taux en Mode I et Mode II pr\'esent une d\'ependance significative de la taille des \'el\'ements dans le cas de fissure ouverte. Il est montr\'e que le taux de restitution d'\'energie en Mode I et Mode II ne converge pas, ce \`a dire que le comportement asymptotique n'est pas limit\'e. Par cons\'equence, il n'est pas possible d'utiliser l'erreur entre it\'erations successives comme mesure de la convergence de la solution et une comparaison est donc n\'ecessaire avec des r\'esultats obtenus avec une autre m\'ethode. Le taux de restitution d'\'energie calcul\'e avec la m\'ethode d'\'el\'ements de fronti\`ere, disponible dans la litt\'erature, est choisi comme r\'ef\'erence.
Ensuite, la propagation de d\'ecollement entre fibre et matrice est \'etudi\'ee dans Volume El\'ementaire Repr\'esentative de : composites unidirectionnels avec \'epaisseur variable, mesur\'e par le nombre des rang\'ees des fibres, de ceux extr\^emement minces (une rang\'ee des fibres) au plus \'epais ; stratifi\'e crois\'e avec un pli central \`a 90$^{\circ}$ d'\'epaisseur variable, mesur\'e par le nombre des rang\'ees des fibres, de ceux extr\^emement minces (une rang\'ee des fibres) au plus \'epais ; composites unidirectionnels \'epais, mod\'elis\'es comme infinis \`a travers l'\'epaisseur. Configurations multiples de l'endommagement sont aussi examin\'ees, qui correspondent \`a diff\'erentes \'etapes du processus d'amor\c{c}age des fissures transverses : d\'ecollements isol\'es ; d\'ecollements interagissant distribu\'es dans la direction d'application de la charge m\'ecanique ; d\'ecollements localis\'es sur fibres cons\'ecutives \`a travers l'\'epaisseur. Entre les r\'esultats plus importants, il est constat\'e que ni l'\'epaisseur du pli \`a 90$^{\circ}$ ni l'\'epaisseur du pli \`a 0$^{\circ}$ influence le taux de restitution d'\'energie du d\'ecollement, diff\'eremment de ce qu'a \'et\'e observ\'e pour les fissures transverses. En revanche, il est montr\'e que le taux de restitution d'\'energie est affect\'e de mani\`ere significative par l'interaction mutuelle entre d\'ecollements dans la direction d'application de la charge et qu'il existe une distance caract\'eristique (mesur\'e par le nombre des fibres sans endommagement) d\'eterminant la r\'egion d'influence entre d\'ecollements.\\
Enfin, la taille du d\'ecollement juste apr\`es l'amor\c{c}age et la taille ultime du d\'ecollement sont estim\'ees \`a partir de l'analyse de la distribution des contraintes \`a l'interface entre fibre et matrice (pour l'amor\c{c}age) et sur la base du crit\`ere de Griffith de la MLER. La taille maximale d'un d\'ecollement dans un stratifi\'e crois\'e est estim\'e dans l'intervalle 40$^{\circ}$ - 60$^{\circ}$, r\'esultat qui est en tr\`es bon accord avec pr\'ec\'edentes observations microscopiques disponibles dans la litt\'erature.
