%-------------------------------------------------------------------
% Document class and package definitions
%-------------------------------------------------------------------
\documentclass[12pt,a4paper,openright,final,twoside]{msethesis}
%Included for Gather Purpose only:
%input "thesisreferences.bib"

\usepackage{epigraph}
\usepackage[final]{changes}

\addbibresource{thesisreferences.bib}
\addbibresource{paperA/refspaperA.bib}
\addbibresource{paperB/refspaperB.bib}
\addbibresource{paperC/refspaperC.bib}
%\addbibresource{paperD/refspaperD.bib}
%\addbibresource{paperE/refspaperE.bib}

\begin{document}

\newrefsection

%\defaultbibliography{thesisreferences}     %% Change this only.
%\defaultbibliographystyle{ieeetr}        %% Could be changed if you like
                                           %% references typeset differently.
%-------------------------------------------------------------------
% Define title, author, etc.
%-------------------------------------------------------------------
\def\thesistitle{Influence of microstructure on debonding at the fiber/matrix interface in fiber-reinforced polymers under tensile loading}
\def\theauthor{Luca Di Stasio}
\def\theaddress{Division of Materials Science\\Department of Engineering Sciences and Mathematics\\
Lule{\aa} University of Technology\\ Lule{\aa}, Sweden}

\def\supervisors{Janis Varna, Zoubir Ayadi}
\def\supervisorstring{Supervisors:} % Edit here if you have only one supervisor
\def\dedication{A mio figlio, Levante Libero Antonio}

% Read abstract and preface from separate files.
% Make sure these exist. See example files.
\def\theabstract{At the end of the second decade of the XXI century, the transportation industry at large faces several challenges that will shape its evolution in the next decade and beyond. The first main challenge is the increasing public awareness and governmental action on climate change, which are increasing the pressure on the industrial sectors responsible for the greatest share of emissions, the transportation industry being one of them, to reduce their environmental footprint. A second challenge lies instead in the renewed push toward price reduction, due to increased competition (as for example the entry of private entities in the market for low-Earth orbit launchers) and innovative business models (like ride-sharing and ride-hailing in the automotive sector or low-cost carriers in civil aviation).\\
A common technical solution strategy to these challenges is the reduction of vehicles' structural mass, while keeping the payload mass constant. By reducing consumption, a reduced weight leads to reduced emissions in fossil-fuels powered vehicles and to increased autonomy in electrical vehicles. By reducing the quantity of materials required in structures, a weight reduction strategy favors a reduction of production costs and thus lower prices. Transportation is however a sector where safety is a paramount concern, and structures must satisfy strict requirements and validation procedures to guarantee their integrity and reliability during service life. This represents a significant constraint which limits the scope of weight reduction strategies.\\
In the last twenty years, the development of a novel type of Fiber-Reinforced Polymer Composite (FRPC) laminates, i.e. \emph{thin-ply} laminates, proposes a solution to these competing requirements (weight with to respect to structural integrity) by providing at the same time weight reduction and increased strength. Several experimental investigations have shown, in fact, that \emph{thin-ply} laminates are capable of delaying, and even suppress, the onset of transverse cracking. Transverse cracks are a kind of sub-critical damage and occur early in the failure process, causing the degradation of elastic properties and favoring other, often more critical, modes of damage (delaminations, fiber breaks). Delay and suppression of transverse cracks were already linked, at the of the 1970's, to the use of thinner plies. However, \emph{thin-plies} available today on the market are at least 10 times thinner than those studied in the 1970's. This changes the length scale of the problem, from millimeters to micrometers. At the microscale, transverse cracks are formed by several fiber/matrix interface cracks (or debonds) coalescing together. Understanding the mechanisms of transverse cracking delay and suppression in \emph{thin-ply} laminates requires detailed knowledge regarding onset of transverse cracking at the microscale, and thus the study of the mechanisms that favor or prevent debond growth.\\
The main objective of the present work is to investigate the influence of the microstructure on debond growth along the fiber arc direction. To this end, models of 2-dimensional Representative Volume Elements are developed
}
\def\thepreface{I bought my first and current car, \emph{La Melanza}, in August 2015, just a few weeks before starting my doctoral studies at Lule\aa\ University of Technology and Universit\'e de Lorraine. Today, October 2019, \emph{La Melanza} has traveled $127'712$ kilometers. It has been, indeed, a long journey. One that has brought me to live in two different countries, France and Sweden, and to visit five more, Germany, Greece, Russia, Italy and Spain, for conferences, summer schools and exchanges. A journey in which I have learned a lot, made new friends and built a family. And, apparently, even managed to write a Ph.D. thesis! No such journey could be ventured alone, and here I would like to thank everyone who helped and supported me in these years.\\
It is common use to place supervisors at the top of the acknowledgements list, and I will not be any different. It is however not in adherence to custom, but with sincere gratitude that I place them here in the first place. Thus, many thanks to Prof. Janis Varna for accepting me as his Ph.D. student, sharing his knowledge, correcting my mistakes, pointing my efforts in the right direction, always being curious and passionate about research. Thanks to Prof. Zoubir Ayadi, for welcoming me in France and supporting me all along.\\
I then wish to thank all the members of Polymeric Composite Group at LTU for welcoming me in Lule\aa, for showing me how to survive at $-30^{\circ}$, for the interesting discussions over a coffee and for their help to solve the problems in the lab: Johanna, Roberts, Patrik, Lennart, Zainab, Nawres, Hiba, Liva, Andrejs, Stephanie, Linqi.\\
I wish to thank also all the people that have helped me extricate myself in all the administrative needs that an international project requires, and have always answered with patience and a smile to my (at times many) questions: Birgitta, Fredrik, Marie-Louise, Christine, Martine, Nadine and Flavio.\\
And finally, my thoughts go to my family. To Scarlett, for ``the purest love in the world is between a grumpy dad and the pet he said he never wanted", and I guess I'm just another proof of it. To Levante, for forcing me to work in order to stay awake late at night guarding him, and for bringing already so much joy in my life. To Valentina, for following me in two different countries, for bringing so many beautiful things in my life and, every now and then, reminding me that there are worse things in life than a deadline for a paper (or a thesis!).\\

\noindent Lule\aa, October 2019\\
Luca Di Stasio
}


% Change here if you want to remove the logo printed on the first page

%\def\thelogo{\includegraphics[width=2.5cm]{eu1_f_eng}} % old EU logo
\def\thelogo{} % no logo

% The definitions above could be put directly in the function call below,
% but is here defined explicitly, for the purpose of clarity.

\startpreamble
  {\thesistitle}
  {\theauthor}
  {\theaddress}
  {\supervisors}
  {\dedication}
  {\theabstract}
  {\thepreface}
  {\thelogo}

%%%%%%%%%%%%%%%%%%%%%%%%%%%%%%%%%%%%%%%%%%%%%%%%%%%%%%%%%%%%%%%%%%%%
%% Begin Part I
%%%%%%%%%%%%%%%%%%%%%%%%%%%%%%%%%%%%%%%%%%%%%%%%%%%%%%%%%%%%%%%%%%%%
\makepartpage{Part I}

%% Initialize part containing the thesis introduction chapters
\startchapters

%\begin{bibunit}
%------------------------ Start chapter 1 --------------------------
% The \makechapter command takes three arguments
%  1) An abbreviated version of the chapter name,
%     to be used as page header
%  2) String to be added to the table of contents
%  3) The chapter name, possibly split in to lines,
%     as in Chapter 2 below.
%
%  The different arguments can have different line breaks.
%
% The actual contents of the chapter is included by removing the
% comment from the \input line below. Make sure the file
% chapter1.tex exists.
%-------------------------------------------------------------------

%\def\myquote{``This report, by its very length, defends itself against the risk of being read."\\[.5\baselineskip] Winston Churchill}

%\makechapter{Running header}{Table of contents entry}{Title
%appearing on the\\ chapter start page\label{ch2}}
%\section{The World Wide Failure Exercises}
This is the text of the second chapter. \cite{*}

\appendix
\section{This is an appendix section}
Text of the appendix

\subsection{Subsection 1}
Yet some text, and an equation
\begin{equation}
    \textnormal{abs}\left(e^{j\pi}\right)=\,?
\end{equation}

\subsection{Subsection 2}
And then some...

\section{This is another appendix section}
This section concludes the appendix.


\def\chaponequote{\dots a ``sage", as an anonymous writer has pointed out, ``calls up in the average mind the picture of something grey and pedantic if not green and aromatic"\\[.5\baselineskip]Arthur D. Little}
\def\chaponequotetext{\dots a ``sage", as an anonymous writer has pointed out, ``calls up in the average mind the picture of something grey and pedantic if not green and aromatic".}
\def\chaponequoteauthor{Arthur D. Little~\cite{Little1924}}
\makechapter{A journey of scales}{A journey of scales}{A journey of scales\label{ch1}}
\epigraph{\chaponequotetext}{\chaponequoteauthor}
\section{General information}
Current version of the \texttt{cseethesis} document class is: \classversion.\\
Last modification: \classdate\\

\noindent As of version 3.0, the template is no longer backwards compatible.

\subsection{About the document class}
This document class was originally created in 2002 when I was working on my own PhD thesis. Since then, many people used it, found bugs (and occasionally even corrected them), and suggested improvements.

The style is tailor-made for the typical types of theses that we write at the department, i.e.\ an introductory part followed by a collection of published or submitted research papers. 

The template supports the use of both \LaTeX\ and pdfLaTeX. If you use the command \texttt{$\backslash$includegraphics} to import your figure and you supply the filename with its extension (e.g.\ .eps, .pdf), compilation should be possible with either one. For this to work, both .eps and .pdf versions of all figures must be available.

The template is totally free to use, modify and distribute, as long as reference to the original author is kept and as long as all files remain in the package. Modified versions can only be distributed if it is clearly mentioned in the document class that modifications have been made and by whom.

The whole package comes AS IS. I will correct bugs every now and then, but other than that, don't expect any support whatsoever.


\subsection{About this document}
This document, as well as the actual \LaTeX\ code for it, makes up the documentation on how to use the document class.

Only this chapter contains any readable information. Chapters 2 and 3 are only included as examples of some of the features of the template. The text is nonsense, but the corresponding \LaTeX\ code may be of some use. The same goes for the appended papers, which are only there as examples of a few options of the document class.

Read this chapter carefully. If you have comments on what else should be in here in order to simplify the use of the document class, let me know.

\section{Chapters}
\subsection{Defining chapters}
In order to add flexibility to the template, a new command, called \texttt{$\backslash$makechapter} is provided. This command takes three mandatory and one optional argument, as 
\begin{center}
	\texttt{$\backslash$makechapter[optional quote]\{page header\}\{toc entry\}\{Chapter title\}}
\end{center}

The reason this is solved like this is to allow for shorter page headers if the chapter name is very long. Also, if the actual chapter heading needs to be manually split in several lines (if the automatic splitting does not look so good), the table of contents (toc) entry might have to be defined differently. Note that normally, the last three arguments can be the same. 

The use of an optional quote as an introduction to the chapter is demonstrated in this chapter. It can just as well be left out, which is demonstrated in this document (see the code).

\subsection{Importing chapter contents}
The sub-documents containing the chapters should start directly, i.e.\ they must not contain any $\backslash$begin\{document\} or $\backslash$end\{document\} tags.

See this file, \textit{chapter1.tex} for details.

\section{How to append papers}
Papers are included using the \texttt{$\backslash$input} command, just as with chapters. You have to typeset paper title, authors, and abstract manually. See the example papers accompanying this document for an example.

To make the separator sheet preceding each paper, use one of the following commands:
\begin{itemize}
	\item \texttt{$\backslash$makepaper} -- Published paper.
	\item \texttt{$\backslash$makepaperaccepted} -- Accepted, not yet published paper.
	\item \texttt{$\backslash$makepapersubmitted} -- Submitted, not yet accepted paper.
	\item \texttt{$\backslash$makepapertobesubmitted} -- Not yet submitted paper.
\end{itemize}

See code for this example document for examples on how to use.

\section{Cross-references}
All labels throughout the thesis have to be unique. If the same
equation or figure shows up twice e.g.\ a figure used
both in the introduction and one of the included papers), it has
to be given different labels. Referring to labels is done as
usual.

A simple trick to make sure this is the case and that will also help you keep track of all labels you used is to use the following naming convention:
\begin{itemize}
	\item \texttt{ch1:fig:labelname}, \texttt{ch1:tab:labelname}, \texttt{ch1:eq:labelname}, etc.\ all denote figures, tables and equations in Chapter 1.
	\item \texttt{paperA:fig:labelname}, \texttt{paperA:tab:labelname}, \texttt{paperA:eq:labelname}, etc.\ all denote figures, tables and equations in Paper A.
\end{itemize}

For existing text, e.g.\ papers, this is easily achieved by a simple search-and-replace operation on the string \texttt{$\backslash$label\{}. Any text editor will do that for you!

\section{Appendices\label{sec:app}}
It is possible to have any number of appendices for each chapter.
Simply type \texttt{$\backslash$appendix} before the first appendix,
which is then a normal \texttt{section}, but numbered differently.

To add appendices to papers, use the
\texttt{$\backslash$paperappendix} command

\section{Including bibliography lists\label{sec:bib}}
In a thesis one might want several separate bibliographies. For
example, one for the first part, and then separate bibliographies
for each of the included papers.

This is solved using the \texttt{bibunits} package together with a
slight work-around in this template. For the first part of the
thesis, there is only one bibliography list, typeset like a chapter
(see this example document). In the papers, the bibliography lists
are typeset as un-numbered sections. See this file
\texttt{cseethesis\_example.tex} and \texttt{paper1.tex} for examples
how to place the bibliographies. 

Note that the command
\begin{itemize}
	\item \texttt{$\backslash$makebib} is used in Part I, to typeset the reference list in the thesis introduction.
	\item \texttt{$\backslash$putbib} is used in the papers in Part II.
\end{itemize}

\section{How to compile your project}
Finally, you probably like to know how to build your project to a
final PDF or PostScript file. Start by verifying that you can
compile this document. This is how it goes:
%
\begin{enumerate}
    \item Run \LaTeX\ (or pdfLaTex) once.
    \item Then run BibTeX on all the
    bu<i> files.
    \item Run \LaTeX\ twice more, to build the final DVI document
    (or pdfLaTeX if you want a PDF file).
\end{enumerate}

The above steps are easily collected in a script or batch file. See
the files \texttt{make.bat} and \texttt{compilebibunits.bat} for
examples.

\section{Revision history}
The template has evolved during several years and the exact
revisions are not clear to anyone. Starting from version 1.6,
however, the changes are more well-documented. This example
document will always support only the latest release of the
template. Below is a list of the revisions made to the document
class (and when applicable, the example document):%
\begin{itemize}
	 \item Version 3.1, September 1, 2010
		\begin{itemize}
			\item Fixed bug related to appendix numbering.
			\item Fixed page numbering of ``Part" pages.
			\item Added a comment at top of cseethesis\_example.tex, for improved compatibility with some editors.
		\end{itemize}
	 \item Version 3.0, June 7, 2009
		\begin{itemize}
			\item Fixed bug related to page headers in chapters containing no subsections.
		   \item Removed the EU class option and replaced with a logo argument to the preamble. See the code of this document for an example.
			\item The template is \textbf{no longer compatible with previous versions}.
         \item Removed the definition of boldface Greek letters from the document class, since this is not the proper place for that.
		\end{itemize}
  	 \item Version 2.5, March 5 2009
		\begin{itemize}
         \item Renamed the template \textit{cseethesis}. It is a continuation of the project initially called \textit{eisthesis}, but since its use has spread I decided to change the name.
			\item Various minor bug fixes.
			\item Update of example document, making examples of additions and revisions in recent versions of the template.
		\end{itemize}
	 \item Version 2.36: Added "Part" to the table of contents.
    \item Version 2.35:
        \begin{itemize}
				\item Fixed a bug regarding the page headers in the "appended papers part".
				\item Fixed a bug causing the section numbering to be wrong in a chapter succeeding a chapter containing appendices.
			   \item Added a chapter 3 in this example document, illustrating how to handle page headers for chapters without any sections. See the code at the top of \texttt{chapter3.tex}.
		  \end{itemize}
    \item Version 2.3:
        \begin{itemize}
            \item Added the commands \texttt{$\backslash$appendix}
            and \texttt{$\backslash$paperappendix}, see section
            \ref{sec:app}.
            \item The bibliography list is now typeset similar to a
            chapter in Part I, and as an un-numbered section in Part
            II. This required the use of separate commands for
            including the lists (see Sec.~\ref{sec:bib})
            \item Typesetting fixes for the table of contents page.
            As a consequence, the package \texttt{titletoc} is now
            required.
            \item Minor other code cleanup and bug fixes.
        \end{itemize}
    \item Version 2.25:
    \begin{itemize}
        \item Fixed a minor bug that used to generate a warning
        message regarding font shapes in the page headers.
    \end{itemize}
    \item Version 2.2:
    \begin{itemize}
        \item pdf and eps class options removed. The document class
        compiles with either pdfLaTeX or \LaTeX.
        \item The reference lists in papers are now typeset in the
        same way as in Part I of the thesis.
        \item Some minor adjustments of page header heights.
    \end{itemize}
    \item Version 2.1:
    \begin{itemize}
        \item Cross-references to chapters now work like they
        should. See the main document of this example.
        \item Major bug fixes to BibTeX reference lists, table of
        contents generation etc.
        \item The only change the user has to do is to use the new \texttt{$\backslash$makebib} instead of
        bibunits' \texttt{$\backslash$putbib}.
    \end{itemize}
    \item Version 2.0:
    \begin{itemize}
        \item Support for EU logotype on main page. The files
        \texttt{eu1\_f\_eng.pdf} and\\ \texttt{eu1\_f\_eng.eps} must be
        placed in the same directory as the document class.
        \item Support for pdf class options.
        \item Update of the \texttt{$\backslash$makechapter}
        command. It now requires three arguments. See main
        document for example.
        \item Support for BibTeX, using the \texttt{bibunits.sty}
        package.
    \end{itemize}
    \item Version 1.6: Various bug fixes to figure spacing etc.
\end{itemize}


\makechapter{Modeling damage in FRPC}{Modeling damage in FRPC}{Modeling damage in FRPC}
%\section{The World Wide Failure Exercises}
This is the text of the second chapter. \cite{*}

\appendix
\section{This is an appendix section}
Text of the appendix

\subsection{Subsection 1}
Yet some text, and an equation
\begin{equation}
    \textnormal{abs}\left(e^{j\pi}\right)=\,?
\end{equation}

\subsection{Subsection 2}
And then some...

\section{This is another appendix section}
This section concludes the appendix.


\makechapter{The fiber-matrix interface crack problem}{The fiber-matrix interface crack problem}{The fiber-matrix interface crack problem}
%This is a chapter with no sections, only here in order to test the document template. Please ignore the rest of this.
\vspace*{24pt}
dkhgfdsh fdsjhgiepy kdslkfds ewiuyfe fkjfdsohew dskjhfd fduew dsk di e sd lkjd dslkfd oiew sao cdk�wq acslkg� sdspo dsjdspe dkfo dkhgfdsh fdsjhgiepy kdslkfds ewiuyfe fkjfdsohew dskjhfd fduew dsk di e sd lkjd dslkfd oiew sao cdk�wq acslkg� sdspo dsjdspe dkfo dkhgfdsh fdsjhgiepy kdslkfds ewiuyfe fkjfdsohew dskjhfd fduew dsk di e sd lkjd dslkfd oiew sao cdk�wq acslkg� sdspo dsjdspe dkfo

dkhgfdsh fdsjhgiepy kdslkfds ewiuyfe fkjfdsohew dskjhfd fduew dsk di e sd lkjd dslkfd oiew sao cdk�wq acslkg� sdspo dsjdspe dkfodkhgfdsh fdsjhgiepy kdslkfds ewiuyfe fkjfdsohew dskjhfd fduew dsk di e sd lkjd dslkfd oiew sao cdk�wq acslkg� sdspo dsjdspe dkfo dkhgfdsh fdsjhgiepy kdslkfds ewiuyfe fkjfdsohew dskjhfd fduew dsk di e sd lkjd dslkfd oiew sao cdk�wq acslkg� sdspo dsjdspe dkfo dkhgfdsh fdsjhgiepy kdslkfds ewiuyfe fkjfdsohew dskjhfd fduew dsk di e sd lkjd dslkfd oiew sao cdk�wq acslkg� sdspo dsjdspe dkfo

dkhgfdsh fdsjhgiepy kdslkfds ewiuyfe fkjfdsohew dskjhfd fduew dsk di e sd lkjd dslkfd oiew sao cdk�wq acslkg� sdspo dsjdspe dkfodkhgfdsh fdsjhgiepy kdslkfds ewiuyfe fkjfdsohew dskjhfd fduew dsk di e sd lkjd dslkfd oiew sao cdk�wq acslkg� sdspo dsjdspe dkfo dkhgfdsh fdsjhgiepy kdslkfds ewiuyfe fkjfdsohew dskjhfd fduew dsk di e sd lkjd dslkfd oiew sao cdk�wq acslkg� sdspo dsjdspe dkfo dkhgfdsh fdsjhgiepy kdslkfds ewiuyfe fkjfdsohew dskjhfd fduew dsk di e sd lkjd dslkfd oiew sao cdk�wq acslkg� sdspo dsjdspe dkfo dkhgfdsh fdsjhgiepy kdslkfds ewiuyfe fkjfdsohew dskjhfd fduew dsk di e sd lkjd  dslkfd oiew sao cdk�wq acslkg� sdspo dsjdspe dkfodkhgfdsh fdsjhgiepy kdslkfds ewiuyfe fkjfdsohew dskjhfd fduew dsk di e sd lkjd dslkfd oiew sao cdk�wq acslkg� sdspo dsjdspe dkfo
dkhgfdsh fdsjhgiepy kdslkfds ewiuyfe fkjfdsohew dskjhfd fduew dsk di e sd lkjd dslkfd oiew sao cdk�wq acslkg� sdspo dsjdspe dkfo
dkhgfdsh fdsjhgiepy kdslkfds ewiuyfe fkjfdsohew dskjhfd fduew dsk di e sd lkjd dslkfd oiew sao cdk�wq acslkg� sdspo dsjdspe dkfo
dkhgfdsh fdsjhgiepy kdslkfds ewiuyfe fkjfdsohew dskjhfd fduew dsk di e sd lkjd dslkfd oiew sao cdk�wq acslkg� sdspo dsjdspe dkfo
dkhgfdsh fdsjhgiepy kdslkfds ewiuyfe fkjfdsohew dskjhfd fduew dsk di e sd lkjd dslkfd oiew sao cdk�wq acslkg� sdspo dsjdspe dkfo
dkhgfdsh fdsjhgiepy kdslkfds ewiuyfe fkjfdsohew dskjhfd fduew dsk di e sd lkjd dslkfd oiew sao cdk�wq acslkg� sdspo dsjdspe dkfo

dkhgfdsh fdsjhgiepy kdslkfds ewiuyfe fkjfdsohew dskjhfd fduew dsk di e sd lkjd dslkfd oiew sao cdk�wq acslkg� sdspo dsjdspe dkfo
dkhgfdsh fdsjhgiepy kdslkfds ewiuyfe fkjfdsohew dskjhfd fduew dsk di e sd lkjd dslkfd oiew sao cdk�wq acslkg� sdspo dsjdspe dkfo

dkhgfdsh fdsjhgiepy kdslkfds ewiuyfe fkjfdsohew dskjhfd fduew dsk di e sd lkjd dslkfd oiew sao cdk�wq acslkg� sdspo dsjdspe dkfo
 
dkhgfdsh fdsjhgiepy kdslkfds ewiuyfe fkjfdsohew dskjhfd fduew dsk di e sd lkjd dslkfd oiew sao cdk�wq acslkg� sdspo dsjdspe dkfodkhgfdsh fdsjhgiepy kdslkfds ewiuyfe fkjfdsohew dskjhfd fduew dsk di e sd lkjd dslkfd oiew sao cdk�wq acslkg� sdspo dsjdspe dkfodkhgfdsh fdsjhgiepy kdslkfds ewiuyfe fkjfdsohew dskjhfd fduew dsk di e sd lkjd dslkfd oiew sao cdk�wq acslkg� sdspo dsjdspe dkfodkhgfdsh fdsjhgiepy kdslkfds ewiuyfe fkjfdsohew dskjhfd fduew dsk di e sd lkjd dslkfd oiew sao cdk�wq acslkg� sdspo dsjdspe dkfodkhgfdsh fdsjhgiepy kdslkfds ewiuyfe fkjfdsohew dskjhfd fduew dsk di e sd lkjd dslkfd oiew sao cdk�wq acslkg� sdspo dsjdspe dkfodkhgfdsh fdsjhgiepy kdslkfds ewiuyfe fkjfdsohew dskjhfd fduew dsk di e sd lkjd dslkfd oiew sao cdk�wq acslkg� sdspo dsjdspe dkfodkhgfdsh fdsjhgiepy kdslkfds ewiuyfe fkjfdsohew dskjhfd fduew dsk di e sd lkjd dslkfd oiew sao cdk�wq acslkg� sdspo dsjdspe dkfodkhgfdsh fdsjhgiepy kdslkfds ewiuyfe fkjfdsohew dskjhfd fduew dsk di e sd lkjd dslkfd oiew sao cdk�wq acslkg� sdspo dsjdspe dkfodkhgfdsh fdsjhgiepy kdslkfds ewiuyfe fkjfdsohew dskjhfd fduew dsk di e sd lkjd dslkfd oiew sao cdk�wq acslkg� sdspo dsjdspe dkfodkhgfdsh fdsjhgiepy kdslkfds ewiuyfe fkjfdsohew dskjhfd fduew dsk di e sd lkjd dslkfd oiew sao cdk�wq acslkg� sdspo dsjdspe dkfodkhgfdsh fdsjhgiepy kdslkfds ewiuyfe fkjfdsohew dskjhfd fduew dsk di e sd lkjd dslkfd oiew sao cdk�wq acslkg� sdspo dsjdspe dkfo
dkhgfdsh fdsjhgiepy kdslkfds ewiuyfe fkjfdsohew dskjhfd fduew dsk di e sd lkjd dslkfd oiew sao cdk�wq acslkg� sdspo dsjdspe dkfodkhgfdsh fdsjhgiepy kdslkfds ewiuyfe fkjfdsohew dskjhfd fduew dsk di e sd lkjd dslkfd oiew sao cdk�wq acslkg� sdspo dsjdspe dkfo

dkhgfdsh fdsjhgiepy kdslkfds ewiuyfe fkjfdsohew dskjhfd fduew dsk di e sd lkjd dslkfd oiew sao cdk�wq acslkg� sdspo dsjdspe dkfodkhgfdsh fdsjhgiepy kdslkfds ewiuyfe fkjfdsohew dskjhfd fduew dsk di e sd lkjd dslkfd oiew sao cdk�wq acslkg� sdspo dsjdspe dkfodkhgfdsh fdsjhgiepy kdslkfds ewiuyfe fkjfdsohew dskjhfd fduew dsk di e sd lkjd dslkfd oiew sao cdk�wq acslkg� sdspo dsjdspe dkfodkhgfdsh fdsjhgiepy kdslkfds ewiuyfe fkjfdsohew dskjhfd fduew dsk di e sd lkjd dslkfd oiew sao cdk�wq acslkg� sdspo dsjdspe dkfo
dkhgfdsh fdsjhgiepy kdslkfds ewiuyfe fkjfdsohew dskjhfd fduew dsk di e sd lkjd dslkfd oiew sao cdk�wq acslkg� sdspo dsjdspe dkfo

dkhgfdsh fdsjhgiepy kdslkfds ewiuyfe fkjfdsohew dskjhfd fduew dsk di e sd lkjd dslkfd oiew sao cdk�wq acslkg� sdspo dsjdspe dkfodkhgfdsh fdsjhgiepy kdslkfds ewiuyfe fkjfdsohew dskjhfd fduew dsk di e sd lkjd dslkfd oiew sao cdk�wq acslkg� sdspo dsjdspe dkfodkhgfdsh fdsjhgiepy kdslkfds ewiuyfe fkjfdsohew dskjhfd fduew dsk di e sd lkjd dslkfd oiew sao cdk�wq acslkg� sdspo dsjdspe dkfodkhgfdsh fdsjhgiepy kdslkfds ewiuyfe fkjfdsohew dskjhfd fduew dsk di e sd lkjd dslkfd oiew sao cdk�wq acslkg� sdspo dsjdspe dkfodkhgfdsh fdsjhgiepy kdslkfds ewiuyfe fkjfdsohew dskjhfd fduew dsk di e sd lkjd dslkfd oiew sao cdk�wq acslkg� sdspo dsjdspe dkfo

dkhgfdsh fdsjhgiepy kdslkfds ewiuyfe fkjfdsohew dskjhfd fduew dsk di e sd lkjd dslkfd oiew sao cdk�wq acslkg� sdspo dsjdspe dkfodkhgfdsh fdsjhgiepy kdslkfds ewiuyfe fkjfdsohew dskjhfd fduew dsk di e sd lkjd dslkfd oiew sao cdk�wq acslkg� sdspo dsjdspe dkfodkhgfdsh fdsjhgiepy kdslkfds ewiuyfe fkjfdsohew dskjhfd fduew dsk di e sd lkjd dslkfd oiew sao cdk�wq acslkg� sdspo dsjdspe dkfodkhgfdsh fdsjhgiepy kdslkfds ewiuyfe fkjfdsohew dskjhfd fduew dsk di e sd lkjd dslkfd oiew sao cdk�wq acslkg� sdspo dsjdspe dkfodkhgfdsh fdsjhgiepy kdslkfds ewiuyfe fkjfdsohew dskjhfd fduew dsk di e sd lkjd dslkfd oiew sao cdk�wq acslkg� sdspo dsjdspe dkfodkhgfdsh fdsjhgiepy kdslkfds ewiuyfe fkjfdsohew dskjhfd fduew dsk di e sd lkjd dslkfd oiew sao cdk�wq acslkg� sdspo dsjdspe dkfodkhgfdsh fdsjhgiepy kdslkfds ewiuyfe fkjfdsohew dskjhfd fduew dsk di e sd lkjd dslkfd oiew sao cdk�wq acslkg� sdspo dsjdspe dkfo
dkhgfdsh fdsjhgiepy kdslkfds ewiuyfe fkjfdsohew dskjhfd fduew dsk di e sd lkjd dslkfd oiew sao cdk�wq acslkg� sdspo dsjdspe dkfodkhgfdsh fdsjhgiepy kdslkfds ewiuyfe fkjfdsohew dskjhfd fduew dsk di e sd lkjd dslkfd oiew sao cdk�wq acslkg� sdspo dsjdspe dkfo

dkhgfdsh fdsjhgiepy kdslkfds ewiuyfe fkjfdsohew dskjhfd fduew dsk di e sd lkjd dslkfd oiew sao cdk�wq acslkg� sdspo dsjdspe dkfodkhgfdsh fdsjhgiepy kdslkfds ewiuyfe fkjfdsohew dskjhfd fduew dsk di e sd lkjd dslkfd oiew sao cdk�wq acslkg� sdspo dsjdspe dkfodkhgfdsh fdsjhgiepy kdslkfds ewiuyfe fkjfdsohew dskjhfd fduew dsk di e sd lkjd dslkfd oiew sao cdk�wq acslkg� sdspo dsjdspe dkfodkhgfdsh fdsjhgiepy kdslkfds ewiuyfe fkjfdsohew dskjhfd fduew dsk di e sd lkjd dslkfd oiew sao cdk�wq acslkg� sdspo dsjdspe dkfodkhgfdsh fdsjhgiepy kdslkfds ewiuyfe fkjfdsohew dskjhfd fduew dsk di e sd lkjd dslkfd oiew sao cdk�wq acslkg� sdspo dsjdspe dkfo
dkhgfdsh fdsjhgiepy kdslkfds ewiuyfe fkjfdsohew dskjhfd fduew dsk di e sd lkjd dslkfd oiew sao cdk�wq acslkg� sdspo dsjdspe dkfodkhgfdsh fdsjhgiepy kdslkfds ewiuyfe fkjfdsohew dskjhfd fduew dsk di e sd lkjd dslkfd oiew sao cdk�wq acslkg� sdspo dsjdspe dkfo

dkhgfdsh fdsjhgiepy kdslkfds ewiuyfe fkjfdsohew dskjhfd fduew dsk di e sd lkjd dslkfd oiew sao cdk�wq acslkg� sdspo dsjdspe dkfodkhgfdsh fdsjhgiepy kdslkfds ewiuyfe fkjfdsohew dskjhfd fduew dsk di e sd lkjd dslkfd oiew sao cdk�wq acslkg� sdspo dsjdspe dkfodkhgfdsh fdsjhgiepy kdslkfds ewiuyfe fkjfdsohew dskjhfd fduew dsk di e sd lkjd dslkfd oiew sao cdk�wq acslkg� sdspo dsjdspe dkfo

dkhgfdsh fdsjhgiepy kdslkfds ewiuyfe fkjfdsohew dskjhfd fduew dsk di e sd lkjd dslkfd oiew sao cdk�wq acslkg� sdspo dsjdspe dkfodkhgfdsh fdsjhgiepy kdslkfds ewiuyfe fkjfdsohew dskjhfd fduew dsk di e sd lkjd dslkfd oiew sao cdk�wq acslkg� sdspo dsjdspe dkfodkhgfdsh fdsjhgiepy kdslkfds ewiuyfe fkjfdsohew dskjhfd fduew dsk di e sd lkjd dslkfd oiew sao cdk�wq acslkg� sdspo dsjdspe dkfodkhgfdsh fdsjhgiepy kdslkfds ewiuyfe fkjfdsohew dskjhfd fduew dsk di e sd lkjd dslkfd oiew sao cdk�wq acslkg� sdspo dsjdspe dkfodkhgfdsh fdsjhgiepy kdslkfds ewiuyfe fkjfdsohew dskjhfd fduew dsk di e sd lkjd dslkfd oiew sao cdk�wq acslkg� sdspo dsjdspe dkfodkhgfdsh fdsjhgiepy kdslkfds ewiuyfe fkjfdsohew dskjhfd fduew dsk di e sd lkjd dslkfd oiew sao cdk�wq acslkg� sdspo dsjdspe dkfodkhgfdsh fdsjhgiepy kdslkfds ewiuyfe fkjfdsohew dskjhfd fduew dsk di e sd lkjd dslkfd oiew sao cdk�wq acslkg� sdspo dsjdspe dkfo
dkhgfdsh fdsjhgiepy kdslkfds ewiuyfe fkjfdsohew dskjhfd fduew dsk di e sd lkjd dslkfd oiew sao cdk�wq acslkg� sdspo dsjdspe dkfodkhgfdsh fdsjhgiepy kdslkfds ewiuyfe fkjfdsohew dskjhfd fduew dsk di e sd lkjd dslkfd oiew sao cdk�wq acslkg� sdspo dsjdspe dkfo

dkhgfdsh fdsjhgiepy kdslkfds ewiuyfe fkjfdsohew dskjhfd fduew dsk di e sd lkjd dslkfd oiew sao cdk�wq acslkg� sdspo dsjdspe dkfodkhgfdsh fdsjhgiepy kdslkfds ewiuyfe fkjfdsohew dskjhfd fduew dsk di e sd lkjd dslkfd oiew sao cdk�wq acslkg� sdspo dsjdspe dkfodkhgfdsh fdsjhgiepy kdslkfds ewiuyfe fkjfdsohew dskjhfd fduew dsk di e sd lkjd dslkfd oiew sao cdk�wq acslkg� sdspo dsjdspe dkfodkhgfdsh fdsjhgiepy kdslkfds ewiuyfe fkjfdsohew dskjhfd fduew dsk di e sd lkjd dslkfd oiew sao cdk�wq acslkg� sdspo dsjdspe dkfo
dkhgfdsh fdsjhgiepy kdslkfds ewiuyfe fkjfdsohew dskjhfd fduew dsk di e sd lkjd dslkfd oiew sao cdk�wq acslkg� sdspo dsjdspe dkfodkhgfdsh fdsjhgiepy kdslkfds ewiuyfe fkjfdsohew dskjhfd fduew dsk di e sd lkjd dslkfd oiew sao cdk�wq acslkg� sdspo dsjdspe dkfodkhgfdsh fdsjhgiepy kdslkfds ewiuyfe fkjfdsohew dskjhfd fduew dsk di e sd lkjd dslkfd oiew sao cdk�wq acslkg� sdspo dsjdspe dkfodkhgfdsh fdsjhgiepy kdslkfds ewiuyfe fkjfdsohew dskjhfd fduew dsk di e sd lkjd dslkfd oiew sao cdk�wq acslkg� sdspo dsjdspe dkfo
dkhgfdsh fdsjhgiepy kdslkfds ewiuyfe fkjfdsohew dskjhfd fduew dsk di e sd lkjd dslkfd oiew sao cdk�wq acslkg� sdspo dsjdspe dkfodkhgfdsh fdsjhgiepy kdslkfds ewiuyfe fkjfdsohew dskjhfd fduew dsk di e sd lkjd dslkfd oiew sao cdk�wq acslkg� sdspo dsjdspe dkfodkhgfdsh fdsjhgiepy kdslkfds ewiuyfe fkjfdsohew dskjhfd fduew dsk di e sd lkjd dslkfd oiew sao cdk�wq acslkg� sdspo dsjdspe dkfodkhgfdsh fdsjhgiepy kdslkfds ewiuyfe fkjfdsohew dskjhfd fduew dsk di e sd lkjd dslkfd oiew sao cdk�wq acslkg� sdspo dsjdspe dkfo
dkhgfdsh fdsjhgiepy kdslkfds ewiuyfe fkjfdsohew dskjhfd fduew dsk di e sd lkjd dslkfd oiew sao cdk�wq acslkg� sdspo dsjdspe dkfodkhgfdsh fdsjhgiepy kdslkfds ewiuyfe fkjfdsohew dskjhfd fduew dsk di e sd lkjd dslkfd oiew sao cdk�wq acslkg� sdspo dsjdspe dkfodkhgfdsh fdsjhgiepy kdslkfds ewiuyfe fkjfdsohew dskjhfd fduew dsk di e sd lkjd dslkfd oiew sao cdk�wq acslkg� sdspo dsjdspe dkfo

dkhgfdsh fdsjhgiepy kdslkfds ewiuyfe fkjfdsohew dskjhfd fduew dsk di e sd lkjd dslkfd oiew sao cdk�wq acslkg� sdspo dsjdspe dkfodkhgfdsh fdsjhgiepy kdslkfds ewiuyfe fkjfdsohew dskjhfd fduew dsk di e sd lkjd dslkfd oiew sao cdk�wq acslkg� sdspo dsjdspe dkfodkhgfdsh fdsjhgiepy kdslkfds ewiuyfe fkjfdsohew dskjhfd fduew dsk di e sd lkjd dslkfd oiew sao cdk�wq acslkg� sdspo dsjdspe dkfodkhgfdsh fdsjhgiepy kdslkfds ewiuyfe fkjfdsohew dskjhfd fduew dsk di e sd lkjd dslkfd oiew sao cdk�wq acslkg� sdspo dsjdspe dkfodkhgfdsh fdsjhgiepy kdslkfds ewiuyfe fkjfdsohew dskjhfd fduew dsk di e sd lkjd dslkfd oiew sao cdk�wq acslkg� sdspo dsjdspe dkfodkhgfdsh fdsjhgiepy kdslkfds ewiuyfe fkjfdsohew dskjhfd fduew dsk di e sd lkjd dslkfd oiew sao cdk�wq acslkg� sdspo dsjdspe dkfodkhgfdsh fdsjhgiepy kdslkfds ewiuyfe fkjfdsohew dskjhfd fduew dsk di e sd lkjd dslkfd oiew sao cdk�wq acslkg� sdspo dsjdspe dkfodkhgfdsh fdsjhgiepy kdslkfds ewiuyfe fkjfdsohew dskjhfd fduew dsk di e sd lkjd dslkfd oiew sao cdk�wq acslkg� sdspo dsjdspe dkfodkhgfdsh fdsjhgiepy kdslkfds ewiuyfe fkjfdsohew dskjhfd fduew dsk di e sd lkjd dslkfd oiew sao cdk�wq acslkg� sdspo dsjdspe dkfodkhgfdsh fdsjhgiepy kdslkfds ewiuyfe fkjfdsohew dskjhfd fduew dsk di e sd lkjd dslkfd oiew sao cdk�wq acslkg� sdspo dsjdspe dkfodkhgfdsh fdsjhgiepy kdslkfds ewiuyfe fkjfdsohew dskjhfd fduew dsk di e sd lkjd dslkfd oiew sao cdk�wq acslkg� sdspo dsjdspe dkfo
dkhgfdsh fdsjhgiepy kdslkfds ewiuyfe fkjfdsohew dskjhfd fduew dsk di e sd lkjd dslkfd oiew sao cdk�wq acslkg� sdspo dsjdspe dkfodkhgfdsh fdsjhgiepy kdslkfds ewiuyfe fkjfdsohew dskjhfd fduew dsk di e sd lkjd dslkfd oiew sao cdk�wq acslkg� sdspo dsjdspe dkfo



\makebib
%\end{bibunit}

%%%%%%%%%%%%%%%%%%%%%%%%%%%%%%%%%%%%%%%%%%%%%%%%%%%%%%%%%%%%%%%%%%%%
%% Begin Part II - Collection of papers
%%%%%%%%%%%%%%%%%%%%%%%%%%%%%%%%%%%%%%%%%%%%%%%%%%%%%%%%%%%%%%%%%%%%

\makepartpage{Part II}%
\startpapers

%-------------------------------------------------------------------
\def\paperheader{Paper A}
\def\papertitle{Finite Element solution of the fiber/matrix interface crack problem: convergence properties and mode mixity of the Virtual Crack Closure Technique}
\def\paperauthorstring{Luca Di Stasio and Zoubir Ayadi}
\def\referencestring{Finite Elements in Analysis and Design, 2019.}
\def\copyrightstring{2019, The Authors.}

% The definitions above could just as well be put directly into the function
% call below, but were explicitly defined to more clearly illustrate the
% use of the function \makepaper.

\newrefsection
\makepapersubmitted
  {\paperheader}
  {\papertitle}
  {\paperauthorstring}
  {\referencestring}
  {\copyrightstring}

% The actual contents is imported by un-commenting the \input line below.
% Make sure the file exist.
\input{paperA/paper1.tex}


%-------------------------------------------------------------------
\def\paperheader{Paper B}
\def\papertitle{Energy release rate of the fiber/matrix interface crack in UD composites under transverse loading: effect of the fiber volume fraction and of the distance to the free surface and to non-adjacent debonds}
\def\paperauthorstring{Luca Di Stasio, Janis Varna and Zoubir Ayadi}
\def\referencestring{Theoretical and Applied Fracture Mechanics, Volume 103, October 2019, 102251.}
\def\copyrightstring{2019, The Publisher, Reprinted with permission.}

% The definitions above could just as well be put directly into the function
% call below, but were explicitly defined to more clearly illustrate the
% use of the function \makepaper.

\newrefsection
\makepaper
  {\paperheader}
  {\papertitle}
  {\paperauthorstring}
  {\referencestring}
  {\copyrightstring}

% The actual contents is imported by un-commenting the \input line below.
% Make sure the file exist.
\begin{bibunit}
\thispagestyle{plain}
\begin{center}
\Large\textbf{\papertitle}\\
\vspace{10mm}
\normalsize John Doe and Jane Doe\\
\vspace{15mm}
\textbf{Abstract}\\
\end{center}
Abstract text of the paper...

\section{Introduction}
The text of this article is imported from the file
\textit{paper2.tex}. The title and abstract part above are typeset
manually (see file for code template).

Be sure NOT to have any $\backslash$begin\{document\} or
$\backslash$end\{document\} tags in the imported files.

Some references here too, just to show the use of bibunits
\nocite{*}.

\paperappendix
\section{First appendix of paper B}
Some appendix text.
\subsection{A subsection of the appendix}
%
\begin{equation}
    X(\omega) = \int_{-\infty}^{\infty} x(t)e^{-j\omega t}\,dt.
\end{equation}
%
\subsection{Another subsection of the appendix}

\section{Another appendix}
Some text in the second appendix

%%% Put references here
\putbib
\end{bibunit}


%-------------------------------------------------------------------
\def\paperheader{Paper C}
\def\papertitle{Effect of the proximity to the $\mathbf{0^{\circ}/90^{\circ}}$ interface on Energy Release Rate of fiber/matrix interface crack growth in the  $\mathbf{90^{\circ}}$-ply of a cross-ply laminate under tensile loading}
\def\paperauthorstring{Luca Di Stasio, Janis Varna and Zoubir Ayadi}
\def\referencestring{Journal of Composite Materials, 2019.}
\def\copyrightstring{2019, The Authors.}

% The definitions above could just as well be put directly into the function
% call below, but were explicitly defined to more clearly illustrate the
% use of the function \makepaper.

\makepapersubmitted
  {\paperheader}
  {\papertitle}
  {\paperauthorstring}
  {\referencestring}
  {\copyrightstring}

% The actual contents is imported by un-commenting the \input line below.
% Make sure the file exist.
\begin{bibunit}
\thispagestyle{plain}
\begin{center}
\Large\textbf{\papertitle}\\
\vspace{10mm}
\normalsize Dr.~C\\
\vspace{15mm}
\textbf{Abstract}\\
\end{center}
Abstract text of the paper...

\section{Introduction}
The text of this article is imported from the file
\textit{paper3.tex}. The title and abstract part above are typeset
manually (see file for code template).

Be sure NOT to have any $\backslash$begin\{document\} or
$\backslash$end\{document\} tags in the imported files.

Some references here too, just to show the use of bibunits
\nocite{*}.


%%% Put references here
\putbib
\end{bibunit}


%-------------------------------------------------------------------
\def\paperheader{Paper D}
\def\papertitle{An example of a yet-to-be-submitted paper}
\def\paperauthorstring{Dr.\ C}

% The definitions above could just as well be put directly into the function
% call below, but were explicitly defined to more clearly illustrate the
% use of the function \makepaper.

\makepapertobesubmitted
  {\paperheader}
  {\papertitle}
  {\paperauthorstring}

% The actual contents is imported by un-commenting the \input line below.
% Make sure the file exist.
%\begin{bibunit}
\thispagestyle{plain}
\begin{center}
\Large\textbf{\papertitle}\\
\vspace{10mm}
\normalsize Dr.~C\\
\vspace{15mm}
\textbf{Abstract}\\
\end{center}
Abstract text of the paper...

\section{Introduction}
The text of this article is imported from the file
\textit{paper3.tex}. The title and abstract part above are typeset
manually (see file for code template).

Be sure NOT to have any $\backslash$begin\{document\} or
$\backslash$end\{document\} tags in the imported files.

Some references here too, just to show the use of bibunits
\nocite{*}.


%%% Put references here
\putbib
\end{bibunit}


\end{document}
