%------------------------------------------------------------------------------------------%
%------------------------------------------------------------------------------------------%
%------------------------------------------------------------------------------------------%
%                                      FILE BEGINS
%------------------------------------------------------------------------------------------%
%------------------------------------------------------------------------------------------%
%------------------------------------------------------------------------------------------%

%------------------------------------------------------------------------------------------%
%------------------------------------------------------------------------------------------%
%                                    DOCUMENT CLASS
%------------------------------------------------------------------------------------------%
%------------------------------------------------------------------------------------------%
\documentclass[a4paper]{jpconf}

%------------------------------------------------------------------------------------------%
%------------------------------------------------------------------------------------------%
%                                       PACKAGES
%------------------------------------------------------------------------------------------%
%------------------------------------------------------------------------------------------%
\usepackage{amsmath}
\usepackage{booktabs}
\usepackage{cite}
\usepackage{float}
\usepackage{graphicx}
\usepackage[caption=false]{subfig}
\usepackage[makeroom]{cancel}

%------------------------------------------------------------------------------------------%
%------------------------------------------------------------------------------------------%
%                                    DOCUMENT BEGINS
%------------------------------------------------------------------------------------------%
%------------------------------------------------------------------------------------------%
\begin{document}

%------------------------------------------------------------------------------------------%
%                                        HEADER
%------------------------------------------------------------------------------------------%

\title{A rule-base approach to the categorisation and automated generation of Representative Unit Cells for the mechanics of damage in fiber reinforced laminates}

\author{Luca Di Stasio$^{1,2,a}$ , Janis Varna$^{2,b}$ and Zoubir Ayadi$^{1,c}$ }

\address{$^{1}$SI2M, IJL, EEIGM, Universit\'e de Lorraine, 6 Rue Bastien Lepage, F-54010 Nancy, France\\$^{2}$Division of Polymer Engineering, Lule\aa\ University of Technology, SE-97187 Lule\aa , Sweden }

{\vspace*{5pt}\address{E-mail: $^{a}$luca.di-stasio@univ-lorraine.fr, $^{b}$janis.varna@ltu.se, $^{c}$zoubir.ayadi@univ-lorraine.fr}}
%\address{$^{a}$luca.di-stasio@univ-lorraine.fr}, \address{$^{b}$janis.varna@ltu.se}, \address{$^{c}$zoubir.ayadi@univ-lorraine.fr}

%------------------------------------------------------------------------------------------%
%                                       ABSTRACT
%------------------------------------------------------------------------------------------%

\begin{abstract}

\end{abstract}

%------------------------------------------------------------------------------------------%
%                                   List of acronyms
%------------------------------------------------------------------------------------------%

\section*{List of acronyms}

\begin{tabular}{ll}
VCCT &  Virtual Crack Closure Technique\\
BEM &  Boundary Element Method\\
FEM &  Finite Element Method\\
\end{tabular}
%\end{table}

%------------------------------------------------------------------------------------------%
%                                   List of symbols
%------------------------------------------------------------------------------------------%

\section*{List of symbols}

%\begin{table}[!h]
\begin{tabular}{lcl}

\end{tabular}
%\end{table}

\clearpage
%------------------------------------------------------------------------------------------%
%                        FEM formulation with quadrilateral elements
%------------------------------------------------------------------------------------------%


%\clearpage
%------------------------------------------------------------------------------------------%
%                             CONCLUSIONS AND PERSPECTIVES
%------------------------------------------------------------------------------------------%

%\section{Conclusions}


%------------------------------------------------------------------------------------------%
%                                   ACKNOWLEDGEMENTS
%------------------------------------------------------------------------------------------%
%\ack
%Luca Di Stasio gratefully acknowledges the support of the European School of Materials (EUSMAT) through the DocMASE Doctoral Programme and the European Commission through the Erasmus Mundus Programme.
%\newpage
%------------------------------------------------------------------------------------------%
%                                      REFERENCES
%------------------------------------------------------------------------------------------%
\section*{References}
\begin{thebibliography}{9}
%\bibitem{author:year}author surname author initials (up to 10) year title {\it Journal} {\bf vol} (issue) pages
% FEM
\bibitem{Griffiths:1994}Griffiths  R. 1994 Stiffness matrix of the four-node quadrilateral element in closed form {\it Int. J. Numer. Meth. Eng.} {\bf 57} (2) 109--143
% LEFM
\bibitem{Krueger:2004}Krueger R. 2004 Virtual crack closure technique: History, approach, and applications {\it Appl. Mech. Rev.} {\bf 57} (2) 109--143
\bibitem{abaqus:2016} ABAQUS 2016 ABAQUS 2016 Analysis User's Manual {\it Online Documentation Help: Dassault Syst\'emes} 
\bibitem{Rice:1968}Rice J. R. 1968 A Path Independent Integral and the Approximate Analysis of Strain Concentration by Notches and Cracks {\it J. Appl. Mech.} {\bf 35} 379--386

%\bibitem{author:year}
\end{thebibliography}

%------------------------------------------------------------------------------------------%
%------------------------------------------------------------------------------------------%
%                                    DOCUMENT ENDS
%------------------------------------------------------------------------------------------%
%------------------------------------------------------------------------------------------%
\end{document}

%------------------------------------------------------------------------------------------%
%------------------------------------------------------------------------------------------%
%------------------------------------------------------------------------------------------%
%                                      FILE ENDS
%------------------------------------------------------------------------------------------%
%------------------------------------------------------------------------------------------%
%------------------------------------------------------------------------------------------%