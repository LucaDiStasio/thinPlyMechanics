% Created on jeu. 27 juil. 2017 12:34:30 CEST
% Created on jeu. 27 juil. 2017 12:34:29 CEST
% Created on jeu. 27 juil. 2017 12:34:29 CEST
% Created on jeu. 27 juil. 2017 12:34:29 CEST
% Created on jeu. 27 juil. 2017 12:34:29 CEST
% Created on jeu. 27 juil. 2017 12:34:28 CEST
% Created on jeu. 27 juil. 2017 12:34:28 CEST
% Created on jeu. 27 juil. 2017 12:34:28 CEST
% Created on jeu. 27 juil. 2017 12:34:28 CEST
% Created on jeu. 27 juil. 2017 12:34:27 CEST
%\documentclass[first,firstsupp,handout,compress,notes,navigation]{ETHclass} 
%\documentclass[first,firstsupp,handout,lastsupp]{ETHclass} 
\documentclass[first,firstsupp,lastsupp,handout,last,hyperref,table]{ETHclass} 
%\documentclass[first,firstsupp]{ETHclass}
\usepackage{etex}

\usepackage{adjustbox}
\usepackage{amsmath}
\usepackage{amssymb}
\usepackage{animate}
\usepackage{booktabs}
\usepackage{charter}
\usepackage{enumitem}
\usepackage{etoolbox}
\usepackage{ifthen}
\usepackage{longtable}
\usepackage{mathrsfs}
\usepackage{multicol}
\usepackage{pgf}
\usepackage{pgfplots}
\usepackage{pifont}
\usepackage{ragged2e}
\usepackage{standalone}
\usepackage[caption=false]{subfig}
\usepackage{tabularx}
\usepackage{tikz}
\usepackage{verbatim}
\usepackage{xcolor}
\usepackage{hyperref}

\pgfplotsset{compat=1.7}

\setbeamertemplate{navigation symbols}{}
\usetikzlibrary{arrows,decorations.pathreplacing,positioning,shapes,shadows}

%\usepackage[style=numeric-comp]{biblatex}

%\usepackage{lipsum}

%\usetikzlibrary{fit}
\usetikzlibrary{arrows}
\usetikzlibrary{trees}

% Options for beamer:
%
% 9,10,11,12,13,14,17pt  Fontsizes
% 
% compress: navigation bar becomes smaller
% t       : place contents of frames on top (alternative: b,c)
% handout : handoutversion
% notes   : show notes
% notes=onlyslideswithnotes
%
%hyperref={bookmarksopen,bookmarksnumbered} : Needed for menues in
%                                             acrobat. Also need
%                                             pdftex as option or 
%                                             compile with
% pdflatex '\PassOptionsToPackage{pdftex,bookmarksopen,bookmarksnumbered}{hyperref} \input{file}'

%\usepackage{beamerseminar}
%\usepackage[accumulated]{beamerseminar}
                                % remove ``accumulated'' option
                                % for original behaviour
%\usepackage{beamerbasenotes}
%\setbeamertemplate{note page}[plain] 
%\setbeameroption{notes on second screen}

%\setbeamertemplate{note page}[plain] 
\setbeamertemplate{note page}{\ \\[.3cm]
\textbf{\color{blue}Notes:}\\%[0.1cm]
{\footnotesize %\tiny
\insertnote}}
%\setbeameroption{notes on second screen}


%\setbeamertemplate{navigation symbols}{} % suppresses all navigation symbols:
 \setbeamertemplate{navigation symbols}[horizontal] % Organizes the navigation symbols horizontally.
% \setbeamertemplate{navigation symbols}[vertical] % Organizes the navigation symbols vertically.
% \setbeamertemplate{navigation symbols}[only frame symbol] % Shows only the navigational symbol for navigating frames.

\setlayoutscale{0.5}
\setparametertextfont{\scriptsize}
\setlabelfont{\scriptsize}

% \useoutertheme[subsection=false]{miniframes}
% \usepackage{etoolbox}
% \makeatletter
% \patchcmd{\slideentry}{\advance\beamer@xpos by1\relax}{}{}{}
% \def\beamer@subsectionentry#1#2#3#4#5{\advance\beamer@xpos by1\relax}%
% \makeatother

% \makeatletter
%     \newenvironment{withoutheadline}{
%        \setbeamertemplate{headline}{%
% \vspace{15pt}
% }
%     }{}
% \makeatother

\makeatletter
    \newenvironment{withoutheadline}{
         \setbeamertemplate{headline}{%
\vspace{35pt}
}
        %\def\beamer@entrycode{\vspace*{-1.5\headheight}}
    }{}
\makeatother

\newcommand{\Cross}{$\mathbin{\tikz [x=1.4ex,y=1.4ex,line width=.2ex, red] \draw (0,0) -- (1,1) (0,1) -- (1,0);}$}%

\newcommand{\Checkmark}{$\color{green}\checkmark$}

\setbeamerfont{subsection in toc}{size=\tiny}

\makeatletter
\patchcmd{\beamer@sectionintoc}
  {\vfill}
  {\vskip1.5\itemsep}
  {}
  {}
\makeatother  

\setbeamertemplate{frametitle continuation}{}

\setbeamertemplate{bibliography entry title}{}
\setbeamertemplate{bibliography entry author}{}
\setbeamertemplate{bibliography entry location}{}
\setbeamertemplate{bibliography entry note}{}

\setbeamercolor*{bibliography entry title}{fg=black}
\setbeamercolor*{bibliography entry author}{fg=black}
\setbeamercolor*{bibliography entry location}{fg=black}
\setbeamercolor*{bibliography entry note}{fg=black}
% and kill the abominable icon
%\setbeamertemplate{bibliography item}{\color{forestgreen}$\blacktriangleright$}
\setbeamertemplate{bibliography item}{\insertbiblabel}
%\setbeamertemplate{bibliography item}{\theenumiv}

\newcommand{\highlightred}[1]{%
  \colorbox{red!50}{$\displaystyle#1$}}
  
\newcommand{\highlightyellow}[1]{%
  \colorbox{yellow!50}{$\displaystyle#1$}}
  
\newcommand{\highlightgreen}[1]{%
  \colorbox{green!50}{$\displaystyle#1$}}

\AtBeginSection[]{
  \begin{frame}
  \vfill
  \centering
  \begin{beamercolorbox}[sep=8pt,center,shadow=true,rounded=true]{title}
    \usebeamerfont{frametitle}\includegraphics[width=2ex]{freccia_trasparente_verde_foresta.png}\hspace{.5ex}~{\LARGE \textsc{\bfseries \insertsectionhead}}\par%
  \end{beamercolorbox}
  \vfill
  \end{frame}
}

\hyphenpenalty=5000
\tolerance=1000

\graphicspath{{figures/}}

\newenvironment{system}{\left\lbrace\begin{array}{@{}l@{}}}{\end{array}\right.}

\newenvironment{subsystem}{\left\lgroup\begin{array}{@{}l@{}}}{\end{array}\right.}

\defbeamertemplate*{title page}{customized}[1][]
{
\usebeamerfont{subtitle}
\usebeamercolor[fg]{subtitle}

\vspace{-1.75cm}

{\center
 \usebeamerfont{title}{\inserttitle}\par
}
\vspace{-.25cm}
{\flushleft
 \usebeamerfont{subtitle}{\small \insertsubtitle} \par
}

%\vspace{-.5cm}

{\center
\setbeamercolor{author}{bg=white,fg=Red}
\usebeamerfont{author}{\footnotesize \insertauthor} \par}

\vspace{-.2cm}

{\center
\usebeamerfont{institute}{\tiny \insertinstitute}\par }

\vspace{.2cm}

{\center
\usebeamerfont{date}{\scriptsize \insertdate} \par }

\vspace{0.2in}
}


\begin{document}
\setbeamertemplate{caption}{\raggedright\insertcaption\par}

\title{\textsc{Update 2017-07-31}}
\author{ L. Di Stasio$^{1,2}$, Z. Ayadi$^{1}$, J. Varna$^{2}$}
%\institute{ Science et Ing\'enierie des Mat\'eriaux et M\'etallurgie (SI2M), Institut Jean Lamour, Nancy, France\\Department of Engineering Sciences and Mathematics, Division of Materials Science, Lule\aa\ University of Technology, Lule\aa, Sweden}
\institute{$^{1}$EEIGM, Universit\'e de Lorraine, Nancy, France\\$^{2}$Division of Materials Science, Lule\aa\ University of Technology, Lule\aa, Sweden}
\date{July 31, 2017}

\begin{frame}[plain]
    \titlepage
\end{frame}

\begin{withoutheadline}
\begin{frame}
\frametitle{Outline}
\justifying
\vspace*{-0.5cm}
% \tableofcontents[hidesubsections]
% \begin{multicols}{2}
% \tableofcontents[hidesubsections]
% \end{multicols}
% \begin{columns}[t]
%         \begin{column}{.5\textwidth}
%             \tableofcontents[sections={1-2}]
%         \end{column}
%         \begin{column}{.5\textwidth}
%             \tableofcontents[sections={3-6}]
%         \end{column}
%     \end{columns}
% \end{frame}
\tableofcontents[hidesubsections]
\end{frame}
\end{withoutheadline}

%\note{}

%\begin{frame}
%\pagediagram
%\end{frame}
%% \note{}

\section{Symbols, Models, Equations \& Reference Data}

\subsection{Symbols}

\begin{frame}
\frametitle{\small Symbols}
\vspace{-0.25cm}
\footnotesize
\centering
\captionsetup[figure]{font=scriptsize,labelfont=scriptsize}
\begin{table}[htbp]

  \centering
  %\caption{Single phase properties summary.}
    \begin{tabularx}{\textwidth}{ccX}
    \textbf{Symbol}&\textbf{Unit} & \textbf{Description} \\[3pt]
    \midrule\\[12pt]
	$\theta$ & $\left[^{\circ}\right]$ & Debond angular position with respect to the center of the arc defined by the debond itself\\[1.5pt]
	$\Delta\theta$ & $\left[^{\circ}\right]$ & Debond semi-angular aperture\\[4pt]
	$\delta$ & $\left[^{\circ}\right]$ & Angle subtended by a single element at the fiber/matrix interface\\[3pt]
	$VF_{f}$ & $\left[-\right]$ & Fiber volume fraction\\[1.5pt]
	$l$ & $\left[\mu m\right]$ & Ply's half-length, equal to RVE's half-length (square element)\\[3pt]
	$u$ & $\left[\mu m\right]$ & Displacement along x\\[1.5pt]
	$w$ & $\left[\mu m\right]$ & Displacement along z\\
    \end{tabularx}%
  \label{tab:phaseprop}%
\end{table}%
\end{frame}

\begin{frame}
\frametitle{\small Symbols}
\vspace{-0.25cm}
\footnotesize
\centering
\captionsetup[figure]{font=scriptsize,labelfont=scriptsize}
\begin{table}[htbp]

  \centering
  %\caption{Single phase properties summary.}
    \begin{tabularx}{\textwidth}{ccX}
    \textbf{Symbol}&\textbf{Unit} & \textbf{Description} \\[3pt]
    \midrule\\[12pt]
	$\Gamma_{1}$ & $\left[-\right]$ & Bonded part of fiber surface\\[1.5pt]
	$\Gamma_{2}$ & $\left[-\right]$ & Free (debonded) part of fiber surface\\[1.5pt]
	$\Gamma_{3}$ & $\left[-\right]$ & Bonded part of matrix surface\\[1.5pt]
	$\Gamma_{4}$ & $\left[-\right]$ & Free (debonded) part of matrix surface\\[1.5pt]
    \end{tabularx}%
  \label{tab:phaseprop}%
\end{table}%
\end{frame}

\subsection{Reference Models}

\begin{frame}
\frametitle{\small Reference Models}
\vspace{-0.25cm}
\centering
\begin{figure}
\centering
\includegraphics[height=0.7\textheight]{LEFM2DsRVEsFsDfreeBCULappAxialDispLR.pdf}
\caption{\scriptsize Simple RVE, BC: free.}
\label{fig:singleRVE-rigid}
\end{figure}
\end{frame}

\begin{frame}
\frametitle{\small Reference Models}
\vspace{-0.25cm}
\centering
\begin{figure}
\centering
\includegraphics[height=0.7\textheight]{LEFM2DsRVEsFsDdepverdispBCULappAxialDispLR.pdf}
\caption{\scriptsize Simple RVE, BC: fixed vertical displacement.}
\label{fig:singleRVE-rigid}
\end{figure}
\end{frame}

\begin{frame}
\frametitle{\small Reference Models}
\vspace{-0.25cm}
\centering
\begin{figure}
\centering
\includegraphics[height=0.7\textheight]{LEFM2DsRVEsFsDhomoBCULappAxialDispLR.pdf}
\caption{\scriptsize Simple RVE, BC: fixed vertical and homogeneous horizontal displacement.}
\label{fig:singleRVE-homo}
\end{figure}
\end{frame}

\subsection{Angular discretization}

\begin{frame}
\frametitle{\small Angular discretization}
\vspace{-0.7cm}
\centering
\captionsetup[figure]{font=scriptsize,labelfont=scriptsize}
\begin{figure}[!h]
\centering
\includegraphics[height=0.7\textheight]{mesh-disc-at-interface.pdf}
  \caption{\scriptsize Angular discretization at fiber/matrix interface: $\delta=\frac{360^{\circ}}{4N_{\alpha}}$.}
  \label{fig:angu-discr-def}
\end{figure}
\end{frame}

\subsection{Material properties}

\begin{frame}
\frametitle{\small Material properties}
\vspace{-0.7cm}
\footnotesize
\centering
\captionsetup[figure]{font=scriptsize,labelfont=scriptsize}
\begin{table}[htbp]

  \centering
  %\caption{Single phase properties summary.}
    \begin{tabular}{cccc}

    \textbf{Material} & \textbf{$E\left[GPa\right]$}\ & \textbf{$G\left[GPa\right]$} & \textbf{$\nu\left[-\right]$} \\[3pt]
    \midrule\\[12pt]
    Glass fiber    & 70,0  & 29,2   & 0,2  \\[16pt]
    Epoxy    & 3,5    & 1,25   & 0,4  

    \end{tabular}%
  \label{tab:phaseprop}%
\end{table}%
\end{frame}

\subsection{Evaluation of $G_{0}$}
\begin{frame}
\frametitle{\small Evaluation of $G_{0}$}
\vspace{-0.7cm}
\footnotesize
\centering
\captionsetup[figure]{font=scriptsize,labelfont=scriptsize}
\begin{equation}
G_{0}=\pi R_{f}\sigma^{2}_{0}\frac{1+k_{m}}{8G_{m}}
\end{equation}
\begin{equation}
k_{m}=3-4\nu_{m}
\end{equation}
\begin{equation}
\sigma_{0}^{undamaged}=\frac{E_{m}}{1-\nu^{2}_{m}}\varepsilon_{xx}
\end{equation}%
\end{frame}

\subsection{VCCT}
\begin{frame}
\frametitle{\small Virtual Crack Closure Technique (Nodal Forces at Crack Tip)}
\vspace{-0.5cm}
\tiny
\centering

\begin{equation}
\Delta u =\left|\Delta u^{matrix}_{\text{1 element before crack tip}}-\Delta u^{fiber}_{\text{1 element before crack tip}}\right|
\end{equation}

\begin{equation}
\Delta w =\left|\Delta w^{matrix}_{\text{1 element before crack tip}}-\Delta w^{fiber}_{\text{1 element before crack tip}}\right|
\end{equation}



\begin{equation}
\beta=\arctan{\left(\frac{z^{matrix, undef}_{\text{crack tip}}}{x^{matrix, undef}_{\text{crack tip}}}\right)}
\end{equation}

\begin{equation}
\Delta_{r}=\cos{\left(\beta\right)}\Delta u+\sin{\left(\beta\right)}\Delta w\qquad\Delta_{\theta}=-\sin{\left(\beta\right)}\Delta u+\cos{\left(\beta\right)}\Delta w
\end{equation}

\begin{equation}
F_{r}=\cos{\left(\beta\right)}F^{reaction}_{x}+\sin{\left(\beta\right)}F^{reaction}_{z}\qquad F_{\theta}=-\sin{\left(\beta\right)}F^{reaction}_{x}+\cos{\left(\beta\right)}F^{reaction}_{z}
\end{equation}

\begin{equation}
G_{I}=\frac{1}{2}\frac{F_{r}\Delta_{r}}{R_{f}\delta}\qquad G_{II}=\frac{1}{2}\frac{F_{\theta}\Delta_{\theta}}{R_{f}\delta}\qquad b=1.0\leftrightarrow\Delta A = bR_{f}\delta
\end{equation}
\end{frame}

\subsection{VCCI}

\begin{frame}
\frametitle{\small Virtual Crack Closure Integral (Stress at Surface Nodes)}
\vspace{-0.7cm}
\tiny
\centering
\captionsetup[figure]{font=scriptsize,labelfont=scriptsize}

\begin{equation}
G_{I}=\frac{1}{2\Delta A}\int_{0}^{\Delta c}\sigma_{n}\left(s\right)\delta u_{n}\left(s-\Delta c\right)ds\quad G_{II}=\frac{1}{2\Delta A}\int_{0}^{\Delta c}\tau_{sn}\left(s\right)\delta u_{s}\left(s-\Delta c\right)ds
\end{equation}

with the reference frame centered on the crack tip and rotated according to the orientation of the crack tip.

\begin{equation}
\beta=\arctan{\left(\frac{z^{matrix, undef}_{\text{crack tip}}}{x^{matrix, undef}_{\text{crack tip}}}\right)}
\end{equation}

\begin{equation}
\Delta u^{i} =\left|\Delta u^{matrix}_{\text{i elements before crack tip}}-\Delta u^{fiber}_{\text{i elements before crack tip}}\right|
\end{equation}

\begin{equation}
\Delta w^{i} =\left|\Delta w^{matrix}_{\text{i elements before crack tip}}-\Delta w^{fiber}_{\text{i elements before crack tip}}\right|
\end{equation}

\begin{equation}
\Delta_{r}^{i}=\cos{\left(\beta\right)}\Delta u^{i}+\sin{\left(\beta\right)}\Delta w^{i}\qquad\Delta_{\theta}^{i}=-\sin{\left(\beta\right)}\Delta u^{i}+\cos{\left(\beta\right)}\Delta w^{i}
\end{equation}

\end{frame}

\begin{frame}
\frametitle{\small Virtual Crack Closure Integral (Stress at Surface Nodes)}
\vspace{-0.7cm}
\tiny
\centering
\captionsetup[figure]{font=scriptsize,labelfont=scriptsize}

\begin{equation}
\sigma_{rr}^{m,i} = \sigma_{xx}^{m,\text{i elements after c.t.}}\cos^{2}{\beta}+\sigma_{zz}^{m,\text{i elements after c.t.}}\sin^{2}{\beta}+2\tau_{xz}^{m,\text{i elements after c.t.}}\sin{\beta}\cos{\beta}
\end{equation}

\begin{equation}
\tau_{r\theta}^{m,i} = \left(\sigma_{zz}^{m,\text{i elements after crack tip}}-\sigma_{xx}^{m,\text{i elements after c.t.}}\right)\sin{\beta}\cos{\beta}+\tau_{xz}^{m,\text{i elements after c.t.}}\left(\cos^{2}{\beta}-\sin^{2}{\beta}\right)
\end{equation}

where $m$ stands for material, i.e. stresses can be extracted either on the fiber or the matrix surface.

\begin{equation}
G_{I}^{m}=\frac{1}{2 R_{f}\delta b}\sum_{i=1}^{N\ Int\ El}\frac{1}{2} R_{f}\delta\left(\sigma_{rr}^{m,i}\Delta_{r}^{i}+\sigma_{rr}^{m,i-1}\Delta_{r}^{i-1}\right)\qquad G_{II}^{m}=\frac{1}{2 R_{f}\delta b}\sum_{i=1}^{N\ Int\ El}\frac{1}{2} R_{f}\delta\left(\tau_{r\theta}^{m,i}\Delta_{\theta}^{i}+\tau_{r\theta}^{m,i-1}\Delta_{\theta}^{i-1}\right)
\end{equation}

remembering $b=1$, i.e. unit depth in the out-of-plane direction, they simplify to

\begin{equation}
G_{I}^{m}=\frac{1}{4}\left(\sigma_{rr}^{m,0}\Delta_{r}^{0}+\sum_{i=1}^{N\ Int\ El -\ 1}\left(2\sigma_{rr}^{m,i}\Delta_{r}^{i}\right)+\sigma_{rr}^{m,N\ Int\ El}\Delta_{r}^{N\ Int\ El}\right)
\end{equation}

\begin{equation}
G_{II}^{m}=\frac{1}{4}\left(\tau_{r\theta}^{m,0}\Delta_{\theta}^{0}+\sum_{i=1}^{N\ Int\ El\ -\ 1}\left(2\tau_{r\theta}^{m,i}\Delta_{\theta}^{i}\right)+\tau_{r\theta}^{m,N\ Int\ El}\Delta_{\theta}^{N\ Int\ El}\right)
\end{equation}

\end{frame}

\section{Results}

\subsection{Model Data}
\begin{frame}
\frametitle{Model Data}
\vspace{-0.5cm}
\centering

\begin{table}[htbp]

  \centering
  %\caption{Single phase properties summary.}
    \begin{tabular}{cc}
    \textbf{Quantity}&\textbf{Value} \\[3pt]
    \midrule\\[12pt]
	$\theta \left[^{\circ}\right]$ & $0 $\\[1.5pt]
	$\Delta\theta \left[^{\circ}\right]$ & $\in\left[10,150\right] $\\[4pt]
	$\delta \left[^{\circ}\right]$ & $\in\left[1,0.2\right] $\\[3pt]
	$VF_{f} \left[-\right]$ & $7.9\cdot 10^{-5}$ \\[1.5pt]
	$\frac{L}{R_{f}} \left[-\right]$ & $\sim 100$ \\[3pt]
	$R_{f} \left[\mu m\right]$ & $1 $ \\[1.5pt]
    \end{tabular}%
  \label{tab:phaseprop}%
\end{table}%

\end{frame}

% ====== > 1.0°

\subsection{$\delta=1.0^{\circ}$}

\begin{frame}
\frametitle{\small $\sigma_{0}$, $\delta=1.0^{\circ}$}
\vspace{-0.5cm}
\centering
\captionsetup[figure]{font=scriptsize,labelfont=scriptsize}
\begin{figure}[!h]
\centering
\includegraphics[height=0.7\textheight]{2017-07-10_AbqRunSummary_SmallStrainD10_sigma-inf_Summary.pdf}
  \caption{\scriptsize In red small strain FEM, in black analytical plain strain value.}
  \label{fig:res1}
\end{figure}
\end{frame}

\begin{frame}
\frametitle{\small $G_{0}$, $\delta=1.0^{\circ}$}
\vspace{-0.5cm}
\centering
\captionsetup[figure]{font=scriptsize,labelfont=scriptsize}
\begin{figure}[!h]
\centering
\includegraphics[height=0.7\textheight]{2017-07-10_AbqRunSummary_SmallStrainD10_G0_Summary.pdf}
  \caption{\scriptsize In red small strain FEM, in black analytical plain strain value.}
  \label{fig:res1}
\end{figure}
\end{frame}

\begin{frame}
\frametitle{\small J-Integral (Abaqus built-in routine), $\delta=1.0^{\circ}$}
\vspace{-0.5cm}
\centering
\captionsetup[figure]{font=scriptsize,labelfont=scriptsize}
\begin{figure}[!h]
\centering
\includegraphics[height=0.7\textheight]{2017-07-10_AbqRunSummary_SmallStrainD10_J-INT_Summary.pdf}
  \caption{\scriptsize Fading from blue to red for contours further from the crack tip, FEM results; in black BEM results.}
  \label{fig:res1}
\end{figure}
\end{frame}

\begin{frame}
\frametitle{\small VCCT in forces (in-house Python routine), $\delta=1.0^{\circ}$}
\vspace{-0.5cm}
\centering
\captionsetup[figure]{font=scriptsize,labelfont=scriptsize}
\begin{figure}[!h]
\centering
\includegraphics[height=0.7\textheight]{2017-07-10_AbqRunSummary_SmallStrainD10_M-F-VCCT_Summary.pdf}
  \caption{\scriptsize In green VCCT from FEM results, in black BEM results; positions of maxima highlighted by dashed lines.}
  \label{fig:res1}
\end{figure}
\end{frame}

\begin{frame}
\frametitle{\small J-Integral and VCCT in forces, $\delta=1.0^{\circ}$}
\vspace{-0.5cm}
\centering
\captionsetup[figure]{font=scriptsize,labelfont=scriptsize}
\begin{figure}[!h]
\centering
\includegraphics[height=0.7\textheight]{2017-07-10_AbqRunSummary_SmallStrainD10_F-VCCT-JINT_Summary.pdf}
  \caption{\scriptsize Fading from blue to red for contours further from the crack tip, J-Integral from FEM results; in green VCCT from FEM results; in black BEM results.}
  \label{fig:res1}
\end{figure}
\end{frame}

\input{2017-07-25_AbqRunSummary_SmallStrain_D10_VCCI.tex}
%%%
% ====== > 0.9°

\subsection{$\delta=0.9^{\circ}$}

\begin{frame}
\frametitle{\small $\sigma_{0}$, $\delta=0.9^{\circ}$}
\vspace{-0.5cm}
\centering
\captionsetup[figure]{font=scriptsize,labelfont=scriptsize}
\begin{figure}[!h]
\centering
\includegraphics[height=0.7\textheight]{2017-07-10_AbqRunSummary_SmallStrainD09_sigma-inf_Summary.pdf}
  \caption{\scriptsize In red small strain FEM, in black analytical plain strain value.}
  \label{fig:res1}
\end{figure}
\end{frame}

\begin{frame}
\frametitle{\small $G_{0}$, $\delta=0.9^{\circ}$}
\vspace{-0.5cm}
\centering
\captionsetup[figure]{font=scriptsize,labelfont=scriptsize}
\begin{figure}[!h]
\centering
\includegraphics[height=0.7\textheight]{2017-07-10_AbqRunSummary_SmallStrainD09_G0_Summary.pdf}
  \caption{\scriptsize In red small strain FEM, in black analytical plain strain value.}
  \label{fig:res1}
\end{figure}
\end{frame}

\begin{frame}
\frametitle{\small J-Integral (Abaqus built-in routine), $\delta=0.9^{\circ}$}
\vspace{-0.5cm}
\centering
\captionsetup[figure]{font=scriptsize,labelfont=scriptsize}
\begin{figure}[!h]
\centering
\includegraphics[height=0.7\textheight]{2017-07-10_AbqRunSummary_SmallStrainD09_J-INT_Summary.pdf}
  \caption{\scriptsize Fading from blue to red for contours further from the crack tip, FEM results; in black BEM results.}
  \label{fig:res1}
\end{figure}
\end{frame}

\begin{frame}
\frametitle{\small VCCT in forces (in-house Python routine), $\delta=0.9^{\circ}$}
\vspace{-0.5cm}
\centering
\captionsetup[figure]{font=scriptsize,labelfont=scriptsize}
\begin{figure}[!h]
\centering
\includegraphics[height=0.7\textheight]{2017-07-10_AbqRunSummary_SmallStrainD09_M-F-VCCT_Summary.pdf}
  \caption{\scriptsize In green VCCT from FEM results, in black BEM results; positions of maxima highlighted by dashed lines.}
  \label{fig:res1}
\end{figure}
\end{frame}

\begin{frame}
\frametitle{\small J-Integral and VCCT in forces, $\delta=0.9^{\circ}$}
\vspace{-0.5cm}
\centering
\captionsetup[figure]{font=scriptsize,labelfont=scriptsize}
\begin{figure}[!h]
\centering
\includegraphics[height=0.7\textheight]{2017-07-10_AbqRunSummary_SmallStrainD09_F-VCCT-JINT_Summary.pdf}
  \caption{\scriptsize Fading from blue to red for contours further from the crack tip, J-Integral from FEM results; in green VCCT from FEM results; in black BEM results.}
  \label{fig:res1}
\end{figure}
\end{frame}

\input{2017-07-25_AbqRunSummary_SmallStrain_D09_VCCI.tex}
%%%
% ====== > 0.8°

\subsection{$\delta=0.8^{\circ}$}

\begin{frame}
\frametitle{\small $\sigma_{0}$, $\delta=0.8^{\circ}$}
\vspace{-0.5cm}
\centering
\captionsetup[figure]{font=scriptsize,labelfont=scriptsize}
\begin{figure}[!h]
\centering
\includegraphics[height=0.7\textheight]{2017-07-10_AbqRunSummary_SmallStrainD08_sigma-inf_Summary.pdf}
  \caption{\scriptsize In red small strain FEM, in black analytical plain strain value.}
  \label{fig:res1}
\end{figure}
\end{frame}

\begin{frame}
\frametitle{\small $G_{0}$, $\delta=0.8^{\circ}$}
\vspace{-0.5cm}
\centering
\captionsetup[figure]{font=scriptsize,labelfont=scriptsize}
\begin{figure}[!h]
\centering
\includegraphics[height=0.7\textheight]{2017-07-10_AbqRunSummary_SmallStrainD08_G0_Summary.pdf}
  \caption{\scriptsize In red small strain FEM, in black analytical plain strain value.}
  \label{fig:res1}
\end{figure}
\end{frame}

\begin{frame}
\frametitle{\small J-Integral (Abaqus built-in routine), $\delta=0.8^{\circ}$}
\vspace{-0.5cm}
\centering
\captionsetup[figure]{font=scriptsize,labelfont=scriptsize}
\begin{figure}[!h]
\centering
\includegraphics[height=0.7\textheight]{2017-07-10_AbqRunSummary_SmallStrainD08_J-INT_Summary.pdf}
  \caption{\scriptsize Fading from blue to red for contours further from the crack tip, FEM results; in black BEM results.}
  \label{fig:res1}
\end{figure}
\end{frame}

\begin{frame}
\frametitle{\small VCCT in forces (in-house Python routine), $\delta=0.8^{\circ}$}
\vspace{-0.5cm}
\centering
\captionsetup[figure]{font=scriptsize,labelfont=scriptsize}
\begin{figure}[!h]
\centering
\includegraphics[height=0.7\textheight]{2017-07-10_AbqRunSummary_SmallStrainD08_M-F-VCCT_Summary.pdf}
  \caption{\scriptsize In green VCCT from FEM results, in black BEM results; positions of maxima highlighted by dashed lines.}
  \label{fig:res1}
\end{figure}
\end{frame}

\begin{frame}
\frametitle{\small J-Integral and VCCT in forces, $\delta=0.8^{\circ}$}
\vspace{-0.5cm}
\centering
\captionsetup[figure]{font=scriptsize,labelfont=scriptsize}
\begin{figure}[!h]
\centering
\includegraphics[height=0.7\textheight]{2017-07-10_AbqRunSummary_SmallStrainD08_F-VCCT-JINT_Summary.pdf}
  \caption{\scriptsize Fading from blue to red for contours further from the crack tip, J-Integral from FEM results; in green VCCT from FEM results; in black BEM results.}
  \label{fig:res1}
\end{figure}
\end{frame}

\input{2017-07-25_AbqRunSummary_SmallStrain_D08_VCCI.tex}
%%%
% ====== > 0.7°

\subsection{$\delta=0.7^{\circ}$}

\begin{frame}
\frametitle{\small $\sigma_{0}$, $\delta=0.7^{\circ}$}
\vspace{-0.5cm}
\centering
\captionsetup[figure]{font=scriptsize,labelfont=scriptsize}
\begin{figure}[!h]
\centering
\includegraphics[height=0.7\textheight]{2017-07-10_AbqRunSummary_SmallStrainD07_sigma-inf_Summary.pdf}
  \caption{\scriptsize In red small strain FEM, in black analytical plain strain value.}
  \label{fig:res1}
\end{figure}
\end{frame}

\begin{frame}
\frametitle{\small $G_{0}$, $\delta=0.7^{\circ}$}
\vspace{-0.5cm}
\centering
\captionsetup[figure]{font=scriptsize,labelfont=scriptsize}
\begin{figure}[!h]
\centering
\includegraphics[height=0.7\textheight]{2017-07-10_AbqRunSummary_SmallStrainD07_G0_Summary.pdf}
  \caption{\scriptsize In red small strain FEM, in black analytical plain strain value.}
  \label{fig:res1}
\end{figure}
\end{frame}

\begin{frame}
\frametitle{\small J-Integral (Abaqus built-in routine), $\delta=0.7^{\circ}$}
\vspace{-0.5cm}
\centering
\captionsetup[figure]{font=scriptsize,labelfont=scriptsize}
\begin{figure}[!h]
\centering
\includegraphics[height=0.7\textheight]{2017-07-10_AbqRunSummary_SmallStrainD07_J-INT_Summary.pdf}
  \caption{\scriptsize Fading from blue to red for contours further from the crack tip, FEM results; in black BEM results.}
  \label{fig:res1}
\end{figure}
\end{frame}

\begin{frame}
\frametitle{\small VCCT in forces (in-house Python routine), $\delta=0.7^{\circ}$}
\vspace{-0.5cm}
\centering
\captionsetup[figure]{font=scriptsize,labelfont=scriptsize}
\begin{figure}[!h]
\centering
\includegraphics[height=0.7\textheight]{2017-07-10_AbqRunSummary_SmallStrainD07_M-F-VCCT_Summary.pdf}
  \caption{\scriptsize In green VCCT from FEM results, in black BEM results; positions of maxima highlighted by dashed lines.}
  \label{fig:res1}
\end{figure}
\end{frame}

\begin{frame}
\frametitle{\small J-Integral and VCCT in forces, $\delta=0.7^{\circ}$}
\vspace{-0.5cm}
\centering
\captionsetup[figure]{font=scriptsize,labelfont=scriptsize}
\begin{figure}[!h]
\centering
\includegraphics[height=0.7\textheight]{2017-07-10_AbqRunSummary_SmallStrainD07_F-VCCT-JINT_Summary.pdf}
  \caption{\scriptsize Fading from blue to red for contours further from the crack tip, J-Integral from FEM results; in green VCCT from FEM results; in black BEM results.}
  \label{fig:res1}
\end{figure}
\end{frame}

\input{2017-07-25_AbqRunSummary_SmallStrain_D07_VCCI.tex}
%%%
% ====== > 0.6°

\subsection{$\delta=0.6^{\circ}$}

\begin{frame}
\frametitle{\small $\sigma_{0}$, $\delta=0.6^{\circ}$}
\vspace{-0.5cm}
\centering
\captionsetup[figure]{font=scriptsize,labelfont=scriptsize}
\begin{figure}[!h]
\centering
\includegraphics[height=0.7\textheight]{2017-07-10_AbqRunSummary_SmallStrainD06_sigma-inf_Summary.pdf}
  \caption{\scriptsize In red small strain FEM, in black analytical plain strain value.}
  \label{fig:res1}
\end{figure}
\end{frame}

\begin{frame}
\frametitle{\small $G_{0}$, $\delta=0.6^{\circ}$}
\vspace{-0.5cm}
\centering
\captionsetup[figure]{font=scriptsize,labelfont=scriptsize}
\begin{figure}[!h]
\centering
\includegraphics[height=0.7\textheight]{2017-07-10_AbqRunSummary_SmallStrainD06_G0_Summary.pdf}
  \caption{\scriptsize In red small strain FEM, in black analytical plain strain value.}
  \label{fig:res1}
\end{figure}
\end{frame}

\begin{frame}
\frametitle{\small J-Integral (Abaqus built-in routine), $\delta=0.6^{\circ}$}
\vspace{-0.5cm}
\centering
\captionsetup[figure]{font=scriptsize,labelfont=scriptsize}
\begin{figure}[!h]
\centering
\includegraphics[height=0.7\textheight]{2017-07-10_AbqRunSummary_SmallStrainD06_J-INT_Summary.pdf}
  \caption{\scriptsize Fading from blue to red for contours further from the crack tip, FEM results; in black BEM results.}
  \label{fig:res1}
\end{figure}
\end{frame}

\begin{frame}
\frametitle{\small VCCT in forces (in-house Python routine), $\delta=0.6^{\circ}$}
\vspace{-0.5cm}
\centering
\captionsetup[figure]{font=scriptsize,labelfont=scriptsize}
\begin{figure}[!h]
\centering
\includegraphics[height=0.7\textheight]{2017-07-10_AbqRunSummary_SmallStrainD06_M-F-VCCT_Summary.pdf}
  \caption{\scriptsize In green VCCT from FEM results, in black BEM results; positions of maxima highlighted by dashed lines.}
  \label{fig:res1}
\end{figure}
\end{frame}

\begin{frame}
\frametitle{\small J-Integral and VCCT in forces, $\delta=0.6^{\circ}$}
\vspace{-0.5cm}
\centering
\captionsetup[figure]{font=scriptsize,labelfont=scriptsize}
\begin{figure}[!h]
\centering
\includegraphics[height=0.7\textheight]{2017-07-10_AbqRunSummary_SmallStrainD06_F-VCCT-JINT_Summary.pdf}
  \caption{\scriptsize Fading from blue to red for contours further from the crack tip, J-Integral from FEM results; in green VCCT from FEM results; in black BEM results.}
  \label{fig:res1}
\end{figure}
\end{frame}

\input{2017-07-25_AbqRunSummary_SmallStrain_D06_VCCI.tex}
%%%
% ====== > 0.5°

\subsection{$\delta=0.5^{\circ}$}

\begin{frame}
\frametitle{\small $\sigma_{0}$, $\delta=0.5^{\circ}$}
\vspace{-0.5cm}
\centering
\captionsetup[figure]{font=scriptsize,labelfont=scriptsize}
\begin{figure}[!h]
\centering
\includegraphics[height=0.7\textheight]{2017-07-10_AbqRunSummary_SmallStrainD05_sigma-inf_Summary.pdf}
  \caption{\scriptsize In red small strain FEM, in black analytical plain strain value.}
  \label{fig:res1}
\end{figure}
\end{frame}

\begin{frame}
\frametitle{\small $G_{0}$, $\delta=0.5^{\circ}$}
\vspace{-0.5cm}
\centering
\captionsetup[figure]{font=scriptsize,labelfont=scriptsize}
\begin{figure}[!h]
\centering
\includegraphics[height=0.7\textheight]{2017-07-10_AbqRunSummary_SmallStrainD05_G0_Summary.pdf}
  \caption{\scriptsize In red small strain FEM, in black analytical plain strain value.}
  \label{fig:res1}
\end{figure}
\end{frame}

\begin{frame}
\frametitle{\small J-Integral (Abaqus built-in routine), $\delta=0.5^{\circ}$}
\vspace{-0.5cm}
\centering
\captionsetup[figure]{font=scriptsize,labelfont=scriptsize}
\begin{figure}[!h]
\centering
\includegraphics[height=0.7\textheight]{2017-07-10_AbqRunSummary_SmallStrainD05_J-INT_Summary.pdf}
  \caption{\scriptsize Fading from blue to red for contours further from the crack tip, FEM results; in black BEM results.}
  \label{fig:res1}
\end{figure}
\end{frame}

\begin{frame}
\frametitle{\small VCCT in forces (in-house Python routine), $\delta=0.5^{\circ}$}
\vspace{-0.5cm}
\centering
\captionsetup[figure]{font=scriptsize,labelfont=scriptsize}
\begin{figure}[!h]
\centering
\includegraphics[height=0.7\textheight]{2017-07-10_AbqRunSummary_SmallStrainD05_M-F-VCCT_Summary.pdf}
  \caption{\scriptsize In green VCCT from FEM results, in black BEM results; positions of maxima highlighted by dashed lines.}
  \label{fig:res1}
\end{figure}
\end{frame}

\begin{frame}
\frametitle{\small J-Integral and VCCT in forces, $\delta=0.5^{\circ}$}
\vspace{-0.5cm}
\centering
\captionsetup[figure]{font=scriptsize,labelfont=scriptsize}
\begin{figure}[!h]
\centering
\includegraphics[height=0.7\textheight]{2017-07-10_AbqRunSummary_SmallStrainD05_F-VCCT-JINT_Summary.pdf}
  \caption{\scriptsize Fading from blue to red for contours further from the crack tip, J-Integral from FEM results; in green VCCT from FEM results; in black BEM results.}
  \label{fig:res1}
\end{figure}
\end{frame}

\input{2017-07-25_AbqRunSummary_SmallStrain_D05_VCCI.tex}
%%%
% ====== > 0.4°

\subsection{$\delta=0.4^{\circ}$}

\begin{frame}
\frametitle{\small $\sigma_{0}$, $\delta=0.4^{\circ}$}
\vspace{-0.5cm}
\centering
\captionsetup[figure]{font=scriptsize,labelfont=scriptsize}
\begin{figure}[!h]
\centering
\includegraphics[height=0.7\textheight]{2017-07-10_AbqRunSummary_SmallStrainD04_sigma-inf_Summary.pdf}
  \caption{\scriptsize In red small strain FEM, in black analytical plain strain value.}
  \label{fig:res1}
\end{figure}
\end{frame}

\begin{frame}
\frametitle{\small $G_{0}$, $\delta=0.4^{\circ}$}
\vspace{-0.5cm}
\centering
\captionsetup[figure]{font=scriptsize,labelfont=scriptsize}
\begin{figure}[!h]
\centering
\includegraphics[height=0.7\textheight]{2017-07-10_AbqRunSummary_SmallStrainD04_G0_Summary.pdf}
  \caption{\scriptsize In red small strain FEM, in black analytical plain strain value.}
  \label{fig:res1}
\end{figure}
\end{frame}

\begin{frame}
\frametitle{\small J-Integral (Abaqus built-in routine), $\delta=0.4^{\circ}$}
\vspace{-0.5cm}
\centering
\captionsetup[figure]{font=scriptsize,labelfont=scriptsize}
\begin{figure}[!h]
\centering
\includegraphics[height=0.7\textheight]{2017-07-10_AbqRunSummary_SmallStrainD04_J-INT_Summary.pdf}
  \caption{\scriptsize Fading from blue to red for contours further from the crack tip, FEM results; in black BEM results.}
  \label{fig:res1}
\end{figure}
\end{frame}

\begin{frame}
\frametitle{\small VCCT in forces (in-house Python routine), $\delta=0.4^{\circ}$}
\vspace{-0.5cm}
\centering
\captionsetup[figure]{font=scriptsize,labelfont=scriptsize}
\begin{figure}[!h]
\centering
\includegraphics[height=0.7\textheight]{2017-07-10_AbqRunSummary_SmallStrainD04_M-F-VCCT_Summary.pdf}
  \caption{\scriptsize In green VCCT from FEM results, in black BEM results; positions of maxima highlighted by dashed lines.}
  \label{fig:res1}
\end{figure}
\end{frame}

\begin{frame}
\frametitle{\small VCCT, percentual error on BEM, $\delta=0.4^{\circ}$}
\vspace{-0.5cm}
\scriptsize
\centering
\captionsetup[figure]{font=scriptsize,labelfont=scriptsize}
\begin{table}[htbp]
  \centering
  %\caption{Single phase properties summary.}
    \begin{tabular}{cccc}
    $\Delta\theta \left[^{\circ}\right]$&$\frac{\frac{G_{I}}{G_{0}}|_{FEM}-\frac{G_{I}}{G_{0}}|_{BEM}}{\frac{G_{I}}{G_{0}}|_{BEM}}$&$\frac{\frac{G_{II}}{G_{0}}|_{FEM}-\frac{G_{II}}{G_{0}}|_{BEM}}{\frac{G_{II}}{G_{0}}|_{BEM}}$&$\frac{\frac{G_{TOT}}{G_{0}}|_{FEM}-\frac{G_{TOT}}{G_{0}}|_{BEM}}{\frac{G_{TOT}}{G_{0}}|_{BEM}}$\\
    \midrule\\
	$10$&$11.84\%$&$-45.09\%$&$0.06\%$\\
	$20$&$26.79\%$&$-28.36\%$&$2.95\%$\\
	$30$&$48.73\%$&$-19.81\%$&$4.31\%$\\
	$40$&$77.24\%$&$-12.93\%$&$2.20\%$\\
	$50$&$181.34\%$&$-6.75\%$&$2.04\%$\\
	$60$&$1084.50\%$&$0.68\%$&$2.78\%$\\
	$70$&&$3.99\%$&$3.93\%$\\
	$80$&&$4.79\%$&$4.61\%$\\
	$90$&&$5.62\%$&$6.07\%$\\
	$100$&&$6.18\%$&$6.59\%$\\
	$110$&&$3.83\%$&$5.37\%$\\
	$120$&&$1.31\%$&$-2.40\%$\\
	$130$&&$-4.97\%$&$-9.28\%$\\
	$140$&&$-30.42\%$&$-29.99\%$\\
	$150$&&$-61.36\%$&$-14.84\%$\\
    \end{tabular}%
  \label{tab:phaseprop}%
\end{table}%
\end{frame}

\begin{frame}
\frametitle{\small J-Integral and VCCT in forces, $\delta=0.4^{\circ}$}
\vspace{-0.5cm}
\centering
\captionsetup[figure]{font=scriptsize,labelfont=scriptsize}
\begin{figure}[!h]
\centering
\includegraphics[height=0.7\textheight]{2017-07-10_AbqRunSummary_SmallStrainD04_F-VCCT-JINT_Summary.pdf}
  \caption{\scriptsize Fading from blue to red for contours further from the crack tip, J-Integral from FEM results; in green VCCT from FEM results; in black BEM results.}
  \label{fig:res1}
\end{figure}
\end{frame}

\input{2017-07-25_AbqRunSummary_SmallStrain_D04_VCCI.tex}
%%%
% ====== > 0.3°

\subsection{$\delta=0.3^{\circ}$}

\begin{frame}
\frametitle{\small $\sigma_{0}$, $\delta=0.3^{\circ}$}
\vspace{-0.5cm}
\centering
\captionsetup[figure]{font=scriptsize,labelfont=scriptsize}
\begin{figure}[!h]
\centering
\includegraphics[height=0.7\textheight]{2017-07-10_AbqRunSummary_SmallStrainD03_sigma-inf_Summary.pdf}
  \caption{\scriptsize In red small strain FEM, in black analytical plain strain value.}
  \label{fig:res1}
\end{figure}
\end{frame}

\begin{frame}
\frametitle{\small $G_{0}$, $\delta=0.3^{\circ}$}
\vspace{-0.5cm}
\centering
\captionsetup[figure]{font=scriptsize,labelfont=scriptsize}
\begin{figure}[!h]
\centering
\includegraphics[height=0.7\textheight]{2017-07-10_AbqRunSummary_SmallStrainD03_G0_Summary.pdf}
  \caption{\scriptsize In red small strain FEM, in black analytical plain strain value.}
  \label{fig:res1}
\end{figure}
\end{frame}

\begin{frame}
\frametitle{\small J-Integral (Abaqus built-in routine), $\delta=0.3^{\circ}$}
\vspace{-0.5cm}
\centering
\captionsetup[figure]{font=scriptsize,labelfont=scriptsize}
\begin{figure}[!h]
\centering
\includegraphics[height=0.7\textheight]{2017-07-10_AbqRunSummary_SmallStrainD03_J-INT_Summary.pdf}
  \caption{\scriptsize Fading from blue to red for contours further from the crack tip, FEM results; in black BEM results.}
  \label{fig:res1}
\end{figure}
\end{frame}

\begin{frame}
\frametitle{\small VCCT in forces (in-house Python routine), $\delta=0.3^{\circ}$}
\vspace{-0.5cm}
\centering
\captionsetup[figure]{font=scriptsize,labelfont=scriptsize}
\begin{figure}[!h]
\centering
\includegraphics[height=0.7\textheight]{2017-07-10_AbqRunSummary_SmallStrainD03_M-F-VCCT_Summary.pdf}
  \caption{\scriptsize In green VCCT from FEM results, in black BEM results; positions of maxima highlighted by dashed lines.}
  \label{fig:res1}
\end{figure}
\end{frame}

\begin{frame}
\frametitle{\small VCCT, percentual error on BEM, $\delta=0.3^{\circ}$}
\vspace{-0.5cm}
\scriptsize
\centering
\captionsetup[figure]{font=scriptsize,labelfont=scriptsize}
\begin{table}[htbp]
  \centering
  %\caption{Single phase properties summary.}
    \begin{tabular}{cccc}
    $\Delta\theta \left[^{\circ}\right]$&$\frac{\frac{G_{I}}{G_{0}}|_{FEM}-\frac{G_{I}}{G_{0}}|_{BEM}}{\frac{G_{I}}{G_{0}}|_{BEM}}$&$\frac{\frac{G_{II}}{G_{0}}|_{FEM}-\frac{G_{II}}{G_{0}}|_{BEM}}{\frac{G_{II}}{G_{0}}|_{BEM}}$&$\frac{\frac{G_{TOT}}{G_{0}}|_{FEM}-\frac{G_{TOT}}{G_{0}}|_{BEM}}{\frac{G_{TOT}}{G_{0}}|_{BEM}}$\\
    \midrule\\
	$10$&$11.91\%$&$-39.86\%$&$1.23\%$\\
	$20$&$24.60\%$&$-26.37\%$&$2.56\%$\\
	$30$&$44.60\%$&$-16.55\%$&$4.97\%$\\
	$40$&$73.27\%$&$-12.54\%$&$1.86\%$\\
	$50$&$154.60\%$&$-5.10\%$&$2.35\%$\\
	$60$&$955.70\%$&$1.01\%$&$2.87\%$\\
	$70$&&$3.99\%$&$4.01\%$\\
	$80$&&$4.79\%$&$4.83\%$\\
	$90$&&$5.62\%$&$5.66\%$\\
	$100$&&$6.18\%$&$6.22\%$\\
	$110$&&$3.83\%$&$3.38\%$\\
	$120$&&$1.31\%$&$-4.08\%$\\
	$130$&&$-4.97\%$&$-10.17\%$\\
	$140$&&$-30.42\%$&$-34.90\%$\\
	$150$&&$-61.36\%$&$-0.52\%$\\
    \end{tabular}%
  \label{tab:phaseprop}%
\end{table}%
\end{frame}

\begin{frame}
\frametitle{\small J-Integral and VCCT in forces, $\delta=0.3^{\circ}$}
\vspace{-0.5cm}
\centering
\captionsetup[figure]{font=scriptsize,labelfont=scriptsize}
\begin{figure}[!h]
\centering
\includegraphics[height=0.7\textheight]{2017-07-10_AbqRunSummary_SmallStrainD03_F-VCCT-JINT_Summary.pdf}
  \caption{\scriptsize Fading from blue to red for contours further from the crack tip, J-Integral from FEM results; in green VCCT from FEM results; in black BEM results.}
  \label{fig:res1}
\end{figure}
\end{frame}

\begin{frame}
\frametitle{\small $G_{I}$ from VCCI, stresses extracted on fiber surface, $\delta=3.0^{\circ}$}
\vspace{-0.75cm}
\centering
\captionsetup[figure]{font=scriptsize,labelfont=scriptsize}
\begin{figure}[!h]
\centering
\includegraphics[height=0.7\textheight]{2017-07-25_AbqRunSummary_SmallStrain_D03/pdf/2017-07-25_AbqRunSummary_SmallStrain_D03_F-SoF-VCCT_GI.pdf}
  \caption{\scriptsize Fading from blue to red for increasing number of integration elements, Virtual Crack Closure Integral (VCCI) from FEM results; in green VCCT from FEM results; in black BEM results.}
  \label{fig:res1}
\end{figure}
\end{frame}
%%%%%%%%%%%%%%%%%%%%%%%%%%%%%%%%%%%%%
\begin{frame}
\frametitle{\small $G_{II}$ from VCCI, stresses extracted on fiber surface, $\delta=3.0^{\circ}$}
\vspace{-0.75cm}
\centering
\captionsetup[figure]{font=scriptsize,labelfont=scriptsize}
\begin{figure}[!h]
\centering
\includegraphics[height=0.7\textheight]{2017-07-25_AbqRunSummary_SmallStrain_D03/pdf/2017-07-25_AbqRunSummary_SmallStrain_D03_F-SoF-VCCT_GII.pdf}
  \caption{\scriptsize Fading from blue to red for increasing number of integration elements, Virtual Crack Closure Integral (VCCI) from FEM results; in green VCCT from FEM results; in black BEM results.}
  \label{fig:res1}
\end{figure}
\end{frame}
%%%%%%%%%%%%%%%%%%%%%%%%%%%%%%%%%%%%%
\begin{frame}
\frametitle{\small $G_{TOT}$ from VCCI, stresses extracted on fiber surface, $\delta=3.0^{\circ}$}
\vspace{-0.75cm}
\centering
\captionsetup[figure]{font=scriptsize,labelfont=scriptsize}
\begin{figure}[!h]
\centering
\includegraphics[height=0.7\textheight]{2017-07-25_AbqRunSummary_SmallStrain_D03/pdf/2017-07-25_AbqRunSummary_SmallStrain_D03_F-SoF-VCCT_GTOT.pdf}
  \caption{\scriptsize Fading from blue to red for increasing number of integration elements, Virtual Crack Closure Integral (VCCI) from FEM results; in green VCCT from FEM results; in black BEM results.}
  \label{fig:res1}
\end{figure}
\end{frame}
%%%%%%%%%%%%%%%%%%%%%%%%%%%%%%%%%%%%%
\begin{frame}
\frametitle{\small Summary of $G_{\left(\cdot\cdot\right)}$ from VCCI, stresses extracted on fiber surface, $\delta=3.0^{\circ}$}
\vspace{-0.75cm}
\centering
\captionsetup[figure]{font=scriptsize,labelfont=scriptsize}
\begin{figure}[!h]
\centering
\includegraphics[height=0.7\textheight]{2017-07-25_AbqRunSummary_SmallStrain_D03/pdf/2017-07-25_AbqRunSummary_SmallStrain_D03_F-SoF-VCCT_Summary.pdf}
  \caption{\scriptsize Fading from blue to red for increasing number of integration elements, Virtual Crack Closure Integral (VCCI) from FEM results; in green VCCT from FEM results; in black BEM results.}
  \label{fig:res1}
\end{figure}
\end{frame}
%%%%%%%%%%%%%%%%%%%%%%%%%%%%%%%%%%%%%
\begin{frame}
\frametitle{\small $G_{I}$ from VCCI, stresses extracted on matrix surface, $\delta=3.0^{\circ}$}
\vspace{-0.75cm}
\centering
\captionsetup[figure]{font=scriptsize,labelfont=scriptsize}
\begin{figure}[!h]
\centering
\includegraphics[height=0.7\textheight]{2017-07-25_AbqRunSummary_SmallStrain_D03/pdf/2017-07-25_AbqRunSummary_SmallStrain_D03_F-SoM-VCCT_GI.pdf}
  \caption{\scriptsize Fading from blue to red for increasing number of integration elements, Virtual Crack Closure Integral (VCCI) from FEM results; in green VCCT from FEM results; in black BEM results.}
  \label{fig:res1}
\end{figure}
\end{frame}
%%%%%%%%%%%%%%%%%%%%%%%%%%%%%%%%%%%%%
\begin{frame}
\frametitle{\small $G_{II}$ from VCCI, stresses extracted on matrix surface, $\delta=3.0^{\circ}$}
\vspace{-0.75cm}
\centering
\captionsetup[figure]{font=scriptsize,labelfont=scriptsize}
\begin{figure}[!h]
\centering
\includegraphics[height=0.7\textheight]{2017-07-25_AbqRunSummary_SmallStrain_D03/pdf/2017-07-25_AbqRunSummary_SmallStrain_D03_F-SoM-VCCT_GII.pdf}
  \caption{\scriptsize Fading from blue to red for increasing number of integration elements, Virtual Crack Closure Integral (VCCI) from FEM results; in green VCCT from FEM results; in black BEM results.}
  \label{fig:res1}
\end{figure}
\end{frame}
%%%%%%%%%%%%%%%%%%%%%%%%%%%%%%%%%%%%%
\begin{frame}
\frametitle{\small $G_{TOT}$ from VCCI, stresses extracted on matrix surface, $\delta=3.0^{\circ}$}
\vspace{-0.75cm}
\centering
\captionsetup[figure]{font=scriptsize,labelfont=scriptsize}
\begin{figure}[!h]
\centering
\includegraphics[height=0.7\textheight]{2017-07-25_AbqRunSummary_SmallStrain_D03/pdf/2017-07-25_AbqRunSummary_SmallStrain_D03_F-SoM-VCCT_GTOT.pdf}
  \caption{\scriptsize Fading from blue to red for increasing number of integration elements, Virtual Crack Closure Integral (VCCI) from FEM results; in green VCCT from FEM results; in black BEM results.}
  \label{fig:res1}
\end{figure}
\end{frame}
%%%%%%%%%%%%%%%%%%%%%%%%%%%%%%%%%%%%%
\begin{frame}
\frametitle{\small Summary of $G_{\left(\cdot\cdot\right)}$ from VCCI, stresses extracted on matrix surface, $\delta=3.0^{\circ}$}
\vspace{-0.75cm}
\centering
\captionsetup[figure]{font=scriptsize,labelfont=scriptsize}
\begin{figure}[!h]
\centering
\includegraphics[height=0.7\textheight]{2017-07-25_AbqRunSummary_SmallStrain_D03/pdf/2017-07-25_AbqRunSummary_SmallStrain_D03_F-SoM-VCCT_Summary.pdf}
  \caption{\scriptsize Fading from blue to red for increasing number of integration elements, Virtual Crack Closure Integral (VCCI) from FEM results; in green VCCT from FEM results; in black BEM results.}
  \label{fig:res1}
\end{figure}
\end{frame}
%%%%%%%%%%%%%%%%%%%%%%%%%%%%%%%%%%%%%

%%%
% ====== > 0.2°

\subsection{$\delta=0.2^{\circ}$}

\begin{frame}
\frametitle{\small $\sigma_{0}$, $\delta=0.2^{\circ}$}
\vspace{-0.5cm}
\centering
\captionsetup[figure]{font=scriptsize,labelfont=scriptsize}
\begin{figure}[!h]
\centering
\includegraphics[height=0.7\textheight]{2017-07-10_AbqRunSummary_SmallStrainD02_sigma-inf_Summary.pdf}
  \caption{\scriptsize In red small strain FEM, in black analytical plain strain value.}
  \label{fig:res1}
\end{figure}
\end{frame}

\begin{frame}
\frametitle{\small $G_{0}$, $\delta=0.2^{\circ}$}
\vspace{-0.5cm}
\centering
\captionsetup[figure]{font=scriptsize,labelfont=scriptsize}
\begin{figure}[!h]
\centering
\includegraphics[height=0.7\textheight]{2017-07-10_AbqRunSummary_SmallStrainD02_G0_Summary.pdf}
  \caption{\scriptsize In red small strain FEM, in black analytical plain strain value.}
  \label{fig:res1}
\end{figure}
\end{frame}

%\begin{frame}
%\frametitle{\small J-Integral (Abaqus built-in routine), $\delta=0.2^{\circ}$}
%\vspace{-0.5cm}
%\centering
%\captionsetup[figure]{font=scriptsize,labelfont=scriptsize}
%\begin{figure}[!h]
%\centering
%\includegraphics[height=0.7\textheight]{2017-07-10_AbqRunSummary_SmallStrainD02_J-INT_Summary.pdf}
%  \caption{\scriptsize Fading from blue to red for contours further from the crack tip, FEM results; in black BEM results.}
%  \label{fig:res1}
%\end{figure}
%\end{frame}

\begin{frame}
\frametitle{\small VCCT in forces (in-house Python routine), $\delta=0.2^{\circ}$}
\vspace{-0.5cm}
\centering
\captionsetup[figure]{font=scriptsize,labelfont=scriptsize}
\begin{figure}[!h]
\centering
\includegraphics[height=0.7\textheight]{2017-07-10_AbqRunSummary_SmallStrainD02_M-F-VCCT_Summary.pdf}
  \caption{\scriptsize In green VCCT from FEM results, in black BEM results; positions of maxima highlighted by dashed lines.}
  \label{fig:res1}
\end{figure}
\end{frame}

\begin{frame}
\frametitle{\small VCCT, percentual error on BEM, $\delta=0.2^{\circ}$}
\vspace{-0.5cm}
\scriptsize
\centering
\captionsetup[figure]{font=scriptsize,labelfont=scriptsize}
\begin{table}[htbp]
  \centering
  %\caption{Single phase properties summary.}
    \begin{tabular}{cccc}
    $\Delta\theta \left[^{\circ}\right]$&$\frac{\frac{G_{I}}{G_{0}}|_{FEM}-\frac{G_{I}}{G_{0}}|_{BEM}}{\frac{G_{I}}{G_{0}}|_{BEM}}$&$\frac{\frac{G_{II}}{G_{0}}|_{FEM}-\frac{G_{II}}{G_{0}}|_{BEM}}{\frac{G_{II}}{G_{0}}|_{BEM}}$&$\frac{\frac{G_{TOT}}{G_{0}}|_{FEM}-\frac{G_{TOT}}{G_{0}}|_{BEM}}{\frac{G_{TOT}}{G_{0}}|_{BEM}}$\\
    \midrule\\
	$10$&$11.41\%$&$-32.33\%$&$2.44\%$\\
	$20$&$21.51\%$&$-23.07\%$&$2.24\%$\\
	$30$&$39.40\%$&$-13.34\%$&$5.22\%$\\
	$40$&$62.43\%$&$-10.19\%$&$1.99\%$\\
	$50$&$131.80\%$&$-4.28\%$&$2.06\%$\\
	$60$&$712.56\%$&$1.64\%$&$3.03\%$\\
	$70$&&$4.05\%$&$4.06\%$\\
	$80$&&$5.02\%$&$5.05\%$\\
	$90$&&$5.69\%$&$5.71\%$\\
	$100$&&$5.83\%$&$5.86\%$\\
	$110$&&$4.45\%$&$3.97\%$\\
	$120$&&$1.55\%$&$-3.88\%$\\
	$130$&&$-6.42\%$&$-11.60\%$\\
	$140$&&$-28.46\%$&$-33.11\%$\\
	$150$&&$-67.88\%$&$-0.52\%$\\
    \end{tabular}%
  \label{tab:phaseprop}%
\end{table}%
\end{frame}

%\begin{frame}
%\frametitle{\small J-Integral and VCCT in forces, $\delta=0.2^{\circ}$}
%\vspace{-0.5cm}
%\centering
%\captionsetup[figure]{font=scriptsize,labelfont=scriptsize}
%\begin{figure}[!h]
%\centering
%\includegraphics[height=0.7\textheight]{2017-07-10_AbqRunSummary_SmallStrainD02_F-VCCT-JINT_Summary.pdf}
%  \caption{\scriptsize Fading from blue to red for contours further from the crack tip, J-Integral from FEM results; in green VCCT from FEM results; in black BEM results.}
%  \label{fig:res1}
%\end{figure}
%\end{frame}

% =======> Summary

\subsection{Summary}

\begin{frame}
\frametitle{\small $G_{I}$, VCCT in forces}
\vspace{-0.5cm}
\centering
\captionsetup[figure]{font=scriptsize,labelfont=scriptsize}
\begin{figure}[!h]
\centering
\includegraphics[height=0.7\textheight]{2017-07-10_AbqRunSummary_SmallStrain_M-F-VCCT_GI.pdf}
  \caption{\scriptsize Fading from red to blue for decreasing size of elements at the interface, VCCT from FEM results; in black BEM results.}
  \label{fig:res1}
\end{figure}
\end{frame}

\begin{frame}
\frametitle{\small $G_{I}$ Error with respect to BEM, VCCT in forces}
\vspace{-0.5cm}
\centering
\captionsetup[figure]{font=scriptsize,labelfont=scriptsize}
\begin{figure}[!h]
\centering
\includegraphics[height=0.7\textheight]{2017-07-10_AbqRunSummary_SmallStrain_M-F-VCCT_GI_ERR.pdf}
  \caption{\scriptsize Fading from red to blue for decreasing size of elements at the interface, VCCT from FEM results.}
  \label{fig:res1}
\end{figure}
\end{frame}

\begin{frame}
\frametitle{\small $G_{II}$, VCCT in forces}
\vspace{-0.5cm}
\centering
\captionsetup[figure]{font=scriptsize,labelfont=scriptsize}
\begin{figure}[!h]
\centering
\includegraphics[height=0.7\textheight]{2017-07-10_AbqRunSummary_SmallStrain_M-F-VCCT_GII.pdf}
  \caption{\scriptsize Fading from red to blue for decreasing size of elements at the interface, VCCT from FEM results; in black BEM results.}
  \label{fig:res1}
\end{figure}
\end{frame}

\begin{frame}
\frametitle{\small $G_{II}$ Error with respect to BEM, VCCT in forces}
\vspace{-0.5cm}
\centering
\captionsetup[figure]{font=scriptsize,labelfont=scriptsize}
\begin{figure}[!h]
\centering
\includegraphics[height=0.7\textheight]{2017-07-10_AbqRunSummary_SmallStrain_M-F-VCCT_GII_ERR.pdf}
  \caption{\scriptsize Fading from red to blue for decreasing size of elements at the interface, VCCT from FEM results.}
  \label{fig:res1}
\end{figure}
\end{frame}

\begin{frame}
\frametitle{\small $G_{TOT}$, VCCT in forces}
\vspace{-0.5cm}
\centering
\captionsetup[figure]{font=scriptsize,labelfont=scriptsize}
\begin{figure}[!h]
\centering
\includegraphics[height=0.7\textheight]{2017-07-10_AbqRunSummary_SmallStrain_M-F-VCCT_GTOT.pdf}
  \caption{\scriptsize Fading from red to blue for decreasing size of elements at the interface, VCCT from FEM results; in black BEM results.}
  \label{fig:res1}
\end{figure}
\end{frame}

\begin{frame}
\frametitle{\small $G_{TOT}$ Error with respect to BEM, VCCT in forces}
\vspace{-0.5cm}
\centering
\captionsetup[figure]{font=scriptsize,labelfont=scriptsize}
\begin{figure}[!h]
\centering
\includegraphics[height=0.7\textheight]{2017-07-10_AbqRunSummary_SmallStrain_M-F-VCCT_GTOT_ERR.pdf}
  \caption{\scriptsize Fading from red to blue for decreasing size of elements at the interface, VCCT from FEM results.}
  \label{fig:res1}
\end{figure}
\end{frame}

\section{Summary \& Conclusion}

\begin{frame}
\frametitle{\small Summary}
\vspace{-0.5cm}
\centering
\begin{list}{\Large\textcolor{green}{$\mathbf{\checkmark}$}}{}  
\item Implemented Virtual Crack Closure Integral (VCCI) method for the calculation of $G_{I}$ and $G_{II}$
\item Analysis of free infinite RVE ($\frac{L}{R_{f}}\sim 100$) for several mesh refinements $\delta\in\left[1.0^{\circ},0.2^{\circ}\right]$
\item $G_{I}$,$G_{II}$ and $G_{TOT}$ calculated using Abaqus built-in J-Integral routine, in-house implemented VCCT and VCCI routines
\end{list}
\end{frame}

\begin{frame}
\frametitle{\small Conclusion}
\vspace{-0.5cm}
\centering
\begin{list}{\Large\textcolor{green}{$\mathbf{\checkmark}$}}{}  
\item Good agreement of J-Integral results with $G_{TOT}$ from BEM
\item J-Integral convergence improves refining the mesh
\item For $\delta=0.4^{\circ},0.3^{\circ},0.2^{\circ}$ maxima are at the right angle ($20^{\circ}$ for $G_{I}$, $60^{\circ}$ for $G_{II}$ and $G_{TOT}$) with in-house VCCT
\item $G_{TOT}$ relative errors of VCCT over BEM are small ($\sim 5\%$ or less) for every $\Delta\theta$
\item $G_{II}$ relative errors of VCCT over BEM are small ($\sim 5\%$ or less) for $\Delta\theta > 40^{\circ}$
\item Results tend to converge to BEM values as the mesh is refined
\end{list}
\end{frame}

\begin{frame}
\frametitle{\small Conclusion}
\vspace{-0.5cm}
\centering
\begin{list}{\Huge\textcolor{red}{$\mathbf{\times}$}}{}  
\item $G_{I}$ relative errors of VCCT over BEM are high ($>10\%$)
\item $G_{II}$ relative errors of VCCT over BEM are high ($>10\%$) for $\Delta\theta \leq 40^{\circ}$
\item $G_{I}$ of VCCI has correct functional form but values are overestimated (except for very small integration lengths)
\item $G_{II}$ (and consequently $G_{TOT}$) of VCCI provides strange results
\end{list}
\end{frame}

\begin{frame}
\frametitle{\small Next steps}
\vspace{-0.5cm}
\centering
\begin{itemize}[label=\ding{212}]
\item New analysis with $\delta=0.05^{\circ}$ to see if a better agreement of $G_{I}$ values is attained
\end{itemize}
\end{frame}

%\section{Appendices \& References}
%
%\subsection{Appendices}
%
%
%%\end{frame}
%
%\subsection{References}
%
%\begin{frame}[allowframebreaks]
%  \frametitle{References}
%    
%  \begin{thebibliography}{10}
%    
%%  \beamertemplatebookbibitems
%%  % Start with overview books.
%%
%%  \bibitem{Author1990}
%%    A.~Author.
%%    \newblock {\em Handbook of Everything}.
%%    \newblock Some Press, 1990.
% 
%    
%  \beamertemplatearticlebibitems
%  % Followed by interesting articles. Keep the list short. 
%
%\bibitem{DonaldL.Flaggs1982}
%Donald L. Flaggs, Murat H. Kural;
%\newblock {\em Experimental Determination of the In Situ Transverse Lamina Strength in Graphite/Epoxy Laminates.}
%\newblock Journal of Composite Materials, vol. 16, n. 2, 1982.
%
%\bibitem{Parvizi1978}
%Parvizi A., Bailey J.E;
%\newblock {\em On multiple transverse cracking in glass fibre epoxy cross-ply laminates.}
%\newblock Journal of Materials Science, 1978; 13:2131-2136.
%
%\bibitem{herraez2015}
%Miguel Herr\'aez, Diego Mora, Fernando Naya, Claudio S. Lopes, Carlos Gonz\'alez, Javier LLorca;
%\newblock {\em Transverse cracking of cross-ply laminates: A computational micromechanics perspective.}
%\newblock Composites Science and Technology, 2015; 110:196-204.
%
%\bibitem{Canal2012}
%Luis Pablo Canal, Carlos Gonz\'alez, Javier Segurado, Javier LLorca;
%\newblock {\em Intraply fracture of fiber-reinforced composites: Microscopic mechanisms and modeling.}
%\newblock Composites Science and Technology, 2012; 72(11):1223-1232.
%
%\bibitem{StephenW.Tsai2005}
%Stephen W. Tsai;
%\newblock {\em Thin ply composites.}
%\newblock JEC Magazine 18, 2005.
%
%
%\bibitem{ZnedekP.Bazant2002}
%Znedek P. Bazant;
%\newblock {\em Size Effect Theory and its Application to Fracture of Fiber Composites and Sandwich Plates.} 
%\newblock in Continuum Damage Mechanics of Materials and Structures, eds. O. Allix and F. Hild, 2002.
%
%
%\bibitem{RobinAmacherWayneSmithClemensDransfeldJohnBotsis2014}
%Robin Amacher, Wayne Smith, Clemens Dransfeld, John Botsis, Jo\"el Cugnoni;
%\newblock {\em Thin Ply: from Size-Effect Characterization to Real Life Design}
%\newblock CAMX 2014, 2014
%
%\bibitem{RalfCuntze}
%Ralf Cuntze;
%\newblock {\em The  World-Wide-Failure-Exercises -I  and - II for UD-materials.}
%
%
%\bibitem{Pinho}
%Pinho, S. T. and Pimenta, S.;
%\newblock {\em Size Effects on the Strength and Toughness of Fibre-Reinforced Composites.}
%
%\bibitem{PedroP.CamanhoCarlosG.DavilaSilvestreT.PinhoLorenzoIannucci2006}
%Pedro P. Camanho, Carlos G. D\'avila, Silvestre T. Pinho, Lorenzo Iannucci, Paul Robinson;
%\newblock {\em Prediction of in situ strengths and matrix cracking in composites under transverse tension and in-plane shear.}
%\newblock Composites Part A: Applied Science and Manufacturing, vol. 37, n. 2, 2006.
%
%\bibitem{P.P.CamanhoP.Maimi2007}
%P.P. Camanho, P. Maim\'i, C.G. D\'avila;
%\newblock {\em Prediction of size effects in notched laminates using continuum damage mechanics.}
%\newblock Composites Science and Technology, vol. 67, n. 13, 2007.
%
%\bibitem{Nairn1992}
%J. A. Nairn;
%\newblock {\em The Initiation and Growth of Delaminations Induced by Matrix Microcracks in Laminated Composites.}
%\newblock International Journal of Fracture, vol. 57, 1992.
%
%\bibitem{JoelCugnoniRobinAmacher2013}
%Joel Cugnoni , Robin Amacher, John Botsis;
%\newblock {\em Thin ply technology advantages. An overview of the TPT-TECA project.}
%\newblock 2014.
%
%
%  \end{thebibliography}
%\end{frame}

\begin{frame}[plain]
\frametitle{}
\end{frame}

\end{document}

