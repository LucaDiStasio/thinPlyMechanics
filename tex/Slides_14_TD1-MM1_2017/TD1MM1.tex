%\documentclass[first,firstsupp,handout,compress,notes,navigation]{ETHclass}
%\documentclass[first,firstsupp,handout,lastsupp]{ETHclass}
\documentclass[first,firstsupp,lastsupp,handout,last,hyperref,table]{ETHclass}
%\documentclass[first,firstsupp]{ETHclass}
\usepackage{etex}

\usepackage{adjustbox}
\usepackage{amsmath}
\usepackage{amssymb}
\usepackage{animate}
%\usepackage[frenchb]{babel}
\usepackage{booktabs}
\usepackage{charter}
\usepackage{enumitem}
\usepackage{etoolbox}
\usepackage{ifthen}
\usepackage{longtable}
\usepackage{mathrsfs}
\usepackage{multicol}
\usepackage{pgf}
\usepackage{pgfplots}
\usepackage{pifont}
\usepackage{ragged2e}
\usepackage{standalone}
\usepackage[caption=false]{subfig}
\usepackage{tabularx}
\usepackage{tikz}
\usepackage{verbatim}
\usepackage{xcolor}
\usepackage{hyperref}

\pgfplotsset{compat=1.7}

\setbeamertemplate{navigation symbols}{}
\usetikzlibrary{arrows,decorations.pathreplacing,positioning,shapes,shadows}

%\usepackage[style=numeric-comp]{biblatex}

%\usepackage{lipsum}

%\usetikzlibrary{fit}
\usetikzlibrary{arrows}
\usetikzlibrary{trees}

% Options for beamer:
%
% 9,10,11,12,13,14,17pt  Fontsizes
%
% compress: navigation bar becomes smaller
% t       : place contents of frames on top (alternative: b,c)
% handout : handoutversion
% notes   : show notes
% notes=onlyslideswithnotes
%
%hyperref={bookmarksopen,bookmarksnumbered} : Needed for menues in
%                                             acrobat. Also need
%                                             pdftex as option or
%                                             compile with
% pdflatex '\PassOptionsToPackage{pdftex,bookmarksopen,bookmarksnumbered}{hyperref} \input{file}'

%\usepackage{beamerseminar}
%\usepackage[accumulated]{beamerseminar}
                                % remove ``accumulated'' option
                                % for original behaviour
%\usepackage{beamerbasenotes}
%\setbeamertemplate{note page}[plain]
%\setbeameroption{notes on second screen}

%\setbeamertemplate{note page}[plain]
\setbeamertemplate{note page}{\ \\[.3cm]
\textbf{\color{blue}Notes:}\\%[0.1cm]
{\footnotesize %\tiny
\insertnote}}
%\setbeameroption{notes on second screen}


%\setbeamertemplate{navigation symbols}{} % suppresses all navigation symbols:
 \setbeamertemplate{navigation symbols}[horizontal] % Organizes the navigation symbols horizontally.
% \setbeamertemplate{navigation symbols}[vertical] % Organizes the navigation symbols vertically.
% \setbeamertemplate{navigation symbols}[only frame symbol] % Shows only the navigational symbol for navigating frames.

\setlayoutscale{0.5}
\setparametertextfont{\scriptsize}
\setlabelfont{\scriptsize}

% \useoutertheme[subsection=false]{miniframes}
% \usepackage{etoolbox}
% \makeatletter
% \patchcmd{\slideentry}{\advance\beamer@xpos by1\relax}{}{}{}
% \def\beamer@subsectionentry#1#2#3#4#5{\advance\beamer@xpos by1\relax}%
% \makeatother

% \makeatletter
%     \newenvironment{withoutheadline}{
%        \setbeamertemplate{headline}{%
% \vspace{15pt}
% }
%     }{}
% \makeatother

\makeatletter
    \newenvironment{withoutheadline}{
         \setbeamertemplate{headline}{%
\vspace{35pt}
}
        %\def\beamer@entrycode{\vspace*{-1.5\headheight}}
    }{}
\makeatother

\newcommand{\Cross}{$\mathbin{\tikz [x=1.4ex,y=1.4ex,line width=.2ex, red] \draw (0,0) -- (1,1) (0,1) -- (1,0);}$}%

\newcommand{\Checkmark}{$\color{green}\checkmark$}

\setbeamerfont{subsection in toc}{size=\tiny}

\makeatletter
\patchcmd{\beamer@sectionintoc}
  {\vfill}
  {\vskip1.5\itemsep}
  {}
  {}
\makeatother

\setbeamertemplate{frametitle continuation}{}

\setbeamertemplate{bibliography entry title}{}
\setbeamertemplate{bibliography entry author}{}
\setbeamertemplate{bibliography entry location}{}
\setbeamertemplate{bibliography entry note}{}

\setbeamercolor*{bibliography entry title}{fg=black}
\setbeamercolor*{bibliography entry author}{fg=black}
\setbeamercolor*{bibliography entry location}{fg=black}
\setbeamercolor*{bibliography entry note}{fg=black}
% and kill the abominable icon
%\setbeamertemplate{bibliography item}{\color{forestgreen}$\blacktriangleright$}
\setbeamertemplate{bibliography item}{\insertbiblabel}
%\setbeamertemplate{bibliography item}{\theenumiv}

\newcommand{\highlightred}[1]{%
  \colorbox{red!50}{$\displaystyle#1$}}

\newcommand{\highlightyellow}[1]{%
  \colorbox{yellow!50}{$\displaystyle#1$}}

\newcommand{\highlightgreen}[1]{%
  \colorbox{green!50}{$\displaystyle#1$}}

\AtBeginSection[]{
  \begin{frame}
  \vfill
  \centering
  \begin{beamercolorbox}[sep=8pt,center,shadow=true,rounded=true]{title}
    \usebeamerfont{frametitle}\includegraphics[width=2ex]{freccia_trasparente_verde_foresta.png}\hspace{.5ex}~{\LARGE \textsc{\bfseries \insertsectionhead}}\par%
  \end{beamercolorbox}
  \vfill
  \end{frame}
}

\hyphenpenalty=5000
\tolerance=1000

\graphicspath{{figures/}}

\newenvironment{system}{\left\lbrace\begin{array}{@{}l@{}}}{\end{array}\right.}

\newenvironment{subsystem}{\left\lgroup\begin{array}{@{}l@{}}}{\end{array}\right.}

\defbeamertemplate*{title page}{customized}[1][]
{
\usebeamerfont{subtitle}
\usebeamercolor[fg]{subtitle}

\vspace{-1.75cm}

{\flushleft
 \usebeamerfont{title}{\inserttitle}\par
}
\vspace{-.25cm}
{\flushleft
 \usebeamerfont{subtitle}{\small \insertsubtitle} \par
}

%\vspace{-.5cm}

{\flushright
\setbeamercolor{author}{bg=white,fg=Red}
\usebeamerfont{author}{\footnotesize \insertauthor} \par}

\vspace{-.2cm}

{\flushright
\usebeamerfont{institute}{\tiny \insertinstitute}\par }

\vspace{.2cm}

{\centering
\usebeamerfont{date}{\scriptsize \insertdate} \par }

\vspace{0.2in}
}


\begin{document}
\setbeamertemplate{caption}{\raggedright\insertcaption\par}

\title[\textsc{Principe de Boltzmann}]{\textsc{D\'etermination de la loi de comportement par l'application du principe de Boltzmann}}
\author{ L. Di Stasio }
%\institute{ Science et Ing\'enierie des Mat\'eriaux et M\'etallurgie (SI2M), Institut Jean Lamour, Nancy, France\\Department of Engineering Sciences and Mathematics, Division of Materials Science, Lule\aa\ University of Technology, Lule\aa, Sweden}
\institute{EEIGM, Universit\'e de Lorraine, Nancy, France}
\date{Travaux Dirig\'es - M\'ecanique des Mat\'eriaux 1}

\begin{frame}[plain]
    \titlepage
\end{frame}

\begin{withoutheadline}
\begin{frame}
\frametitle{Table des mati\`eres}
\justifying
\vspace*{-0.5cm}
% \tableofcontents[hidesubsections]
% \begin{multicols}{2}
% \tableofcontents[hidesubsections]
% \end{multicols}
% \begin{columns}[t]
%         \begin{column}{.5\textwidth}
%             \tableofcontents[sections={1-2}]
%         \end{column}
%         \begin{column}{.5\textwidth}
%             \tableofcontents[sections={3-6}]
%         \end{column}
%     \end{columns}
% \end{frame}
\tableofcontents[hidesubsections]
\end{frame}
\end{withoutheadline}

%\note{}

%\begin{frame}
%\pagediagram
%\end{frame}
%% \note{}

\section{Rappels d'organisation}

\subsection{Contacts}

\begin{frame}
%\vspace*{-1cm}
\frametitle{\small Contacts}
\vspace{-0.75cm}
\centering
\captionsetup[subfigure]{labelfont=footnotesize}
\begin{itemize}
\item luca.di-stasio@univ-lorraine.fr
\item Bureau des th\`esards, Salle 0.5, EEIGM
\end{itemize}
\end{frame}

\subsection{Pr\'esence aux TDs}

\begin{frame}
%\vspace{-0.5cm}
\frametitle{\small Pr\'esence aux TDs}
\vspace{-0.25cm}
\centering
\scriptsize
\begin{itemize}[label=\ding{212}]
\item La pr\'esence aux s\'eances des TDs est obligatoire
\end{itemize}
\end{frame}

\section{Probl\`eme 1}

\subsection{Enonce}

\begin{frame}
\frametitle{Enonce}
%\vspace{-1cm}
\centering
On dispose des r\'esultats exp\'erimentaux donnant l'\'evolution de la complaisance en fonction du temps.
\begin{table}
\begin{tabular}{cccccc}
{\footnotesize Temps $\left[h\right]$}                      &{\footnotesize 0}&{\footnotesize 100} &{\footnotesize 200} &{\footnotesize 300}&{\footnotesize 400}\\
{\footnotesize Complaisance J$\left[10^{-3}MPa^{-1}\right]$}&{\footnotesize 9}&{\footnotesize 10.6}&{\footnotesize 10.9}&{\footnotesize 11.1}&{\footnotesize 11.2}\\
\end{tabular}
\end{table}
\begin{table}
\begin{tabular}{cccccc}
{\footnotesize Temps $\left[h\right]$}                      &{\footnotesize 500}&{\footnotesize 600}&{\footnotesize 700}&{\footnotesize 800}&{\footnotesize 900}\\
{\footnotesize Complaisance J$\left[10^{-3}MPa^{-1}\right]$}&{\footnotesize 11.4}&{\footnotesize 11.5}&{\footnotesize 11.6}&{\footnotesize 11.7}&{\footnotesize 11.7}\\
\end{tabular}
\end{table}
Ces donn\'ees ont \'et\'e obtenues lors d'un essai sur une \'eprouvette en polym\'ere \'a temp\`erature ambiante sous une contrainte donn\'ee.
\end{frame}

\begin{frame}
\frametitle{Enonce}
%\vspace{-1cm}
\centering
On cherche \'a determiner l'\'evolution de la d\'eformation en fonction du temps pour l'histoire de chargement d\'ecrite sur le tableau en utilisant le principe de superposition de Boltzmann.
%\begin{table}
\begin{tabular}{ccccccc}
{\footnotesize Contrainte appliqu\'ee $\sigma_{T}$ $\left[MPa\right]$}&{\footnotesize 10}&{\footnotesize 5} &{\footnotesize 10} &{\footnotesize 0}&{\footnotesize 5}&{\footnotesize 7.5}\\
{\footnotesize Dur\'ee $\Delta t\left[h\right]$}                     &{\footnotesize 100}&{\footnotesize 200}&{\footnotesize 200}&{\footnotesize 100}&{\footnotesize 200}&{\footnotesize 100}\\
\end{tabular}
\end{frame}

\begin{frame}[plain]
\frametitle{}
\end{frame}

\end{document}
